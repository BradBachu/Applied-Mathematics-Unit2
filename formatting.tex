%%%%%%%%%%%%%%%%%%%%%%%%%%%%%%%%%%%%%%%%%%%%%%%%%%%%%%%%%%%%%%%%%%%%%%%%%%%%%%%%%%%%%%%%%%%
% Do not edit this file
% Copy and paste the contents into another file, and save it under another name eg Topic_2

% This template shows general structures that you should try to stick to
% Try to use 'tab' to indent your code. It doesnt not affect the compiler but it makes it easier for others to understand
% Do not try to build this code
% Go back to Main, make sure you \include{thisfile} and then build in main
% We can add more tools here as necessary
%%%%%%%%%%%%%%%%%%%%%%%%%%%%%%%%%%%%%%%%%%%%%%%%%%%%%%%%%%%%%%%%%%%%%%%%%%%%%%%%%%%%%%%%%%%
% Use this to create a section
\section{Section}

	% you can also create subsections
	\subsection{Subsections}

		% Or subsubsections
		\subsubsection{Subsubsection}

			This is how you enter math mode in text $x = 1$. Alternatively, you can center it by doubling the dollar sign $$x=y$$ but I would try to void this.

			The recommended thing is to used 
			\begin{equation}
				x = y
			\end{equation}
			or add an asterisx if you want to remove the number on the side
			\begin{equation*}
				x=y
			\end{equation*}

			When you have a series of equations, its better to use the align enviroment
			\begin{align}
				x &= y \\
				x &= 2\\
				\therefore y &= 2 \
			\end{align}
			You can add the asterisk to remove the number with the align enviroment also
			\begin{align*}
				x &= y \\
				x &= 2\\
				\therefore y &= 2 \
			\end{align*}

	\subsection{Cross Referencing}\label{subsection: Cross_Referencing}
		To cross reference simply a the label tag as above. Then you can reference it as Subsection \ref{subsection: Cross_Referencing}. You can also do this for equations, tables, images, figures....

		Example:
		\begin{equation} \label{Eq:Straight_Line}
			x=y
		\end{equation}
		Then you can get the refernece as Eq.\ref{Eq:Straight_Line}. You might want to make it more pretty by doing Eq.~[\ref{Eq:Straight_Line}].
