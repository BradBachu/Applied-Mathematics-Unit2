%------------------------------------------------------------------------------
% Author(s):
%  Brad Bachu
%
% Copyright:
%  Copyright (C) 2020 Brad Bachu, Arjun Mohammed, Nicholas Sammy, Kerry Singh
%
%  This file is part of Applied-Mathematics-Unit2 and is distributed under the
%  terms of the MIT License. See the LICENSE file for details.
%
%  Description:
%     How to make tables for the Hungarian algorithm 
%------------------------------------------------------------------------------

We will use \texttt{minipage} enviroment to place the tables.
In \texttt{solutionmannual.cls} we have defined \texttt{\textbackslash hhs}
\texttt{\textbackslash hhe},\texttt{\textbackslash hvs} and \texttt{\textbackslash hve}
to assist with drawing lines on the matrix. Please see that document for details.

\begin{table}[!hbt]
   % \centering
   \begin{minipage}{0.3\textwidth}
      \centering
      \begin{tabular}{cccc}
          0 & 0 & 0 & 0 \\
          0 & 0 & 1 & 3 \\
          0 & 1 & 4 & 5 \\
          0 & 2 & 4 & 8 \\
      \end{tabular}
      \caption*{Step 1} % it might be convinent to get rid of all the 'Table 1.2' stuff. Use the * next to caption.If you do this you need to get rid of the label as well otherwise it will not work.
   \end{minipage}
   \begin{minipage}{0.3\textwidth}
      \centering
      \begin{tabular}{cccc}
         \hhs{h1}0 & 0 & 0 & 0 \hhe[red]{h1} \\
                 0 & 0 & 1 & 3 \\
                 0 & 1 & 4 & 5 \\
                 0 & 2 & 4 & 8\\
      \end{tabular}
      \caption{\label{20XX:hung1:step2} Step 2}
   \end{minipage}
   \begin{minipage}{0.3\textwidth}
      \centering
      \begin{tabular}{cccc}
         \hhs{h1}\hvs{v1}0 &       \hvs{v2}0 & 0 & 0 \hhe{h1} \\
                         0 &               0 & 1 & 3          \\
                         0 &               1 & 4 & 5          \\
            \hve[red]{v1}0 & \hve[red]{v2}2 & 4 & 8          \\
      \end{tabular}
      \caption{\label{20XX:hung1:step2} Step 3}
   \end{minipage}

   \vspace{20pt} % when you need to move to a new line give it some vertical space
   \begin{minipage}{0.3\textwidth}
      \centering
      \begin{tabular}{cccc}
         \hhs{h1}\hvs{v1}0 &       \hvs{v2}0 & 0 & 0 \hhe{h1} \\
                         0 &               0 & 1 & 3          \\
                         0 &               1 & 4 & 5          \\
            \hve[red]{v1}0 & \hve[blue]{v2}2 & 4 & 8          \\
      \end{tabular}
      \caption{\label{20XX:hung1:step2} Step 4}
   \end{minipage}
   \caption{\label{ex:20XX:hungarian_main} Here is an example of how to setup the Hungarian algorithm}
\end{table}

You might want to consider not labeling the individual tables but labeling the overall table, or vice-versa. Clearly doing both doesnt make sense since I didnt use the subcaption package.