%------------------------------------------------------------------------------
% Author(s):
% Varaun  Ramgoolie
% Copyright:
%  Copyright (C) 2020 Brad Bachu, Arjun Mohammed, Varaun Ramgoolie, Nicholas Sammy
%
%  This file is part of Applied-Mathematics-Unit2 and is distributed under the
%  terms of the MIT License. See the LICENSE file for details.
%
%  Description:
%     Year: 2015 
%     Module: 2
%     Question: 4
%------------------------------------------------------------------------------

\begin{subquestions}
	
%------------------------------------------------------------------------------
% q4 a
%------------------------------------------------------------------------------
\subquestion

\begin{subsubquestions}
	
\subsubquestion
The probability distribution function of a continuous random variable $X$ is given, as well as $F(0.5)=0.8.$ \\

From \rprop{mod2:prop:ContinuousRV:1}, we know that,
\begin{equation}
	 \int_{-\infty}^{\infty} f(x).dx. = 1
\end{equation}


Thus, we can formulate the equation,
\begin{align}
	\int_{0}^{1} (a+bx).dx = 1, \nn \\
	\left[ax + \frac{bx^2}{2}\right]_0^1 = 1, \nn \\
	\left[a(1) + \frac{b(1)^2}{2}\right] - \left[a(0) + \frac{b(0)^2}{2}\right]= 1, \nn \\
	a + \frac{b}{2} = 1. \label{2015:q4:eqn:crv1}
\end{align}

In order to solve for $a$ and $b$, we must formulate another equation using the cumulative distribution function of $X$ given as $F(x)$. From \rdef{mod2:defn:ContinuousRV:CDF}, 
\begin{align}
	F(x) & = \int_{-\infty}^{x} f(x).dx \nn \\
	     & = \int_{0}^{x} (a+bx)dx \nn \\
	     & = \left[ax + \frac{bx^2}{2}\right]_0^x \nn \\
	     & = \left[a(x) + \frac{b(x)^2}{2}\right] - \left[a(0) + \frac{b(0)^2}{2}\right] \nn \\
	     & = ax + \frac{bx^2}{2}.
\end{align}

Since we are given that $F(0.5)=0.8$, our second equation becomes,
\begin{align}
	F(0.5) & = a(0.5) + \frac{b(0.5)^2}{2} = 0.8, \nn \\
  \implies &  0.5 \times a + 0.25 \times \frac{b}{2} = 0.8, \nn \\
  \implies & \frac{a}{2} + \frac{b}{8} = \frac{4}{5}. \label{2015:q4:eqn:crv2}
\end{align}

We will manipulate \req{2015:q4:eqn:crv2} by multiplying by 4 to obtain,
\begin{equation}
	2a + \frac{b}{2} = \frac{16}{5}. \label{2015:q4:eqn:crv3}
\end{equation}

We can now solve for $a$ by subtracting \req{2015:q4:eqn:crv3} from \req{2015:q4:eqn:crv1}.
\begin{align}
	\left[2a + \frac{b}{2} \right] - \left[a + \frac{b}{2} \right] & = \left[\frac{16}{5} \right] - [1] \nn \\
	a & = \frac{11}{5} \nn \\
	  & = 2.2.
\end{align}

Finally, substituting $a=2.2$ into \req{2015:q4:eqn:crv1}, we can solve for $b$ as follows,
\begin{align}
	2.2 + \frac{b}{2} & = 1 \nn \\
	\frac{b}{2} & = 1 - 2.2 \nn \\
	            & = -1.2 \nn \\
	\implies  b & = -1.2 \times 2 \nn \\
	            & = -2.4
\end{align}

Thus, $a=2.2$ and $b=-2.4$.

%------------------------------------------------------------------------------

\subsubquestion

From \rdef{mod2:note:ContinuousRV:CDF2}, we can solve for $P(X>0.5)$ as follows,
\begin{align}
	P(X>0.5) & = 1 - F(0.5) \nn \\
	         & = 1 - 0.8 \nn \\
	         & = 0.2.
\end{align}

Thus, $P(X>0.5)=0.2$.
\end{subsubquestions}

%------------------------------------------------------------------------------
% q4 b
%------------------------------------------------------------------------------

\subquestion

\begin{subsubquestions}
	
\subsubquestion

It is given that $X$ is normally distributed as $X \sim N(9,1^2)$. The values of $m$ and $n$ can be given as,
\begin{align}
	m & = \text{Total} \times P(8 \leq X < 9) \nn \\
	  & = 400 \times P(8 \leq X < 9). \nn \\
    n & = \text{Total} \times P(X \geq 10) \nn \\
      & = 400 \times P(X \geq 10). \label{2015:q4:MNeqn}
\end{align}
	
To calculate $P(8 \leq X < 9)$, we must first standardize the random variable $X$, to the normal random variable $Z$. From Section ~\ref{mod2:section:Normal}, we standardize and calculate as follows,
\begin{align}
	P(8 \leq X < 9) & = P\left(\frac{8- \mu}{\sigma} \leq \frac{X-\mu}{\sigma} < \frac{9-\mu}{\sigma} \right) \nn \\
	                & = P\left(\frac{8- 9}{1} \leq Z < \frac{9-9}{1} \right) \nn \\
	                & = P\left(-1 \leq Z < 0 \right) \nn \\
	                & = \Phi(0) - \Phi(-1) \nn \\
	                & = \Phi(0) - (1 - \Phi(1)) \nn \\
	                & = \Phi(0) + \Phi(1) - 1 \nn \\
	                & = 0.5 + 0.841 - 1 \nn \\
	                & = 0.341 \,. \label{2015:q4:Normal1}
\end{align}

Similarly, in order to calculate $P(X \geq 10)$, we have to standardize in a similar manner as \req{2015:q4:Normal1}. This result is as follows,
\begin{align}
	P(X \geq 10) & = P\left(\frac{X-\mu}{\sigma} \geq \frac{10 - \mu}{\sigma} \right) \nn \\
	             & = P\left(Z \geq \frac{10-9}{1}\right) \nn \\
	             & = P\left(Z \geq \frac{1}{1}\right) \nn \\
	             & = P\left(Z \geq 1\right) \nn \\
	             & = 1 - \Phi(1) \nn \\
	             & = 1 - 0.841 \nn \\
	             & = 0.159 \,. \label{2015:q4:Normal2}
\end{align}

Using \req{2015:q4:Normal1} and \req{2015:q4:Normal2}, we can calculate the values of $m$ and $n$ from \req{2015:q4:MNeqn} as follows,
\begin{align}
	m & = 400 \times P(8 \leq X < 9) \nn \\
	  & = 400 \times 0.341 \nn \\
	  & = 136.5 \,. \\
    n & = 400 \times P(X \geq 10) \nn \\
      & = 400 \times 0.159 \nn \\
      & = 63.5 \,.
\end{align}

Therefore, $m=136.5$ and $n=63.5$. \\

However, we can also notice that $X$ is assumed to have a normal distribution. This means that the cumulative probabilities are symmetric about the mean of the distribution (which in this case is 9). Therefore, $P(8 \leq X < 9) = P(9 \leq X < 10)$ and $P(X < 8) = P(X \geq 10)$. This method spares us from the calculation above and provides the answer quickly.

%------------------------------------------------------------------------------

\subsubquestion

We can state 2 hypotheses:
\begin{itemize}
	\item $H_0$ : The data can be modeled by a normal distribution with mean 9 and standard deviation 1.
	\item $H_1$ : The data cannot be modeled by a normal distribution with mean 9 and standard deviation 1.
\end{itemize}

We are asked to perform a $\chi^2$ goodness-of-fit test at the 5\% significance level. From Section(placeholder for CHI Squared notes), we know that,
\begin{equation}
	\chi^2_{\text{calc.}} = \sum_{\forall k} \left( \frac{(O_k-E_k)^2}{E_k}\right) \,.
\end{equation}

We will formulate a table and obtain the relevant values. See \rtab{2015:q4:tab:Chi}.
\begin{table}[H]
	\centering
	\begin{tabular}{|c|c|c|c|c|c|}
		\hline 
		X & Observed, O & Expected, E & (O-E) & (O-E)$^2$ & $\frac{(O-E)^2}{E}$ \\
		\hline
		$x<8$ & 80 & 63.5 & 16.5 & 272.25 & 4.29 \\
		$8 \leq x < 9$ & 180 & 136.5 & 43.5 & 1892.25 & 13.86 \\
		$9 \leq x < 10$ & 50 & 136.5 & -86.5 & 7482.25 & 54.82 \\
		$x \geq 10$ & 90 & 63.5 & 26.5 & 702.25 & 11.06 \\
		\hline
		$\chi^2_{\text{calc.}}$ & & & & & 84.03 \\
		\hline
	\end{tabular}
	\caption{\label{2015:q4:tab:Chi} Observed and Expected Values.}
\end{table}

We will reject the null hypothesis, $H_0$, iff 
\begin{align}
	\chi^2_{calc.} > \chi^2_{\alpha} (\upsilon) \,, \nn 
\end{align}
where $\alpha$ is the significance level, and $\upsilon$ is the number of degrees of freedom. \\

From the question, we know $\alpha= 0.05$ and $\upsilon = 4-1=3$. Therefore, the value of the test statistic, $\chi^2_{0.05}(3)=7.815$.\\
Since we have calculated that $\chi^2_{\text{calc.}} = 84.03$, which is definitely greater than $\chi^2_{0.05}(3)=7.815$, we reject $H_0$. Thus, there is not sufficient evidence at the 5\% significance level to suggest that the data can be modeled by a normal distribution with mean 9 and standard deviation 1.

\end{subsubquestions}

\end{subquestions}

