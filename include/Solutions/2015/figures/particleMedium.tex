% The following was edited from https://texample.net/tikz/examples/free-body-diagrams/

\documentclass[crop,tikz]{standalone}
%\usetikzlibrary{...}% tikz package already loaded by 'tikz' option
\usetikzlibrary{scopes}
\begin{document}

\def\iangle{35} % Angle of the inclined plane
\def\down{-90}
\def\arcr{0.5cm} % Radius of the arc used to indicate angles

% Define some basic objects of the diagrams
\tikzstyle{m1}= [rectangle,draw=black,fill=lightgray,inner sep = 1pt,minimum size=0.5cm,thin,font=\small]
\tikzstyle{m2}= [rectangle,draw=black,fill=lightgray,minimum size=0.5cm,thin,font=\small]
\tikzstyle{plane}= [draw=black,fill=blue!10]
\tikzstyle{string}= [draw=red, thick]
\tikzstyle{pulley}= [thick]
\tikzstyle{particle}= [thick,circle,draw=black,fill=lightgray,inner sep = 1pt,minimum size=0.5cm,thin,font=\small]
% line styles
\tikzstyle{force}= [>=latex,draw=blue,fill=blue]
\tikzstyle{internal_axis}= [densely dashed,gray,font=\small]
\tikzstyle{external_axis}= [font=\small]

\tikzset
    {
        extaxis1/.pic = 
        {
            \draw[->] (0,0) -- (0,.7) node[left] {$+y_1$} ;
            \draw[->] (0,0) --(.7,0) node[right] {$+x_1$};
        }
}

\tikzset
    {
        extaxis2/.pic = 
        {
            \draw[->] (0,0) -- (0,.7) node[left] {$+y_2$} ;
            \draw[->] (0,0) --(.7,0) node[right] {$+x_2$};
        }
}

\tikzset
    {
        extaxis3/.pic = 
        {
            \draw[->] (0,0) -- (0,.7) node[left] {$+y_3$} ;
            \draw[->] (0,0) --(.7,0) node[right] {$+x_3$};
        }
}

\begin{tikzpicture}

% Free body diagram
\draw[particle] circle (0.25cm);

% Internal Axis
{[internal_axis,->]
    \draw(0,0) -- (0,.5) node[left] {$+y_1$};   
    \draw(0,0) -- (1,0) node[below] {$+x_1$};   
}

% External Axes
{[external_axis,transform shape] \draw (-2,-1)  pic {extaxis1};}

% Insert forces
{[force,->]
    \draw(0,-.25) -- (0,-2) node[left] {$F_r + mg$};
}



\end{tikzpicture}

\end{document}