% The following was edited from https://texample.net/tikz/examples/free-body-diagrams/

\documentclass[crop,tikz]{standalone}
%\usetikzlibrary{...}% tikz package already loaded by 'tikz' option
\usetikzlibrary{scopes}
\begin{document}

\def\iangle{35} % Angle of the inclined plane
\def\down{-90}
\def\arcr{0.5cm} % Radius of the arc used to indicate angles

% Define some basic objects of the diagrams
\tikzstyle{m1}= [rectangle,draw,fill=lightgray,minimum size=0.5cm,thin]
\tikzstyle{m2}= [rectangle,draw=black,fill=lightgray,minimum size=0.3cm,thin]
\tikzstyle{plane}= [draw=black,fill=blue!10]
\tikzstyle{string}= [draw=red, thick]
\tikzstyle{pulley}= [thick]
% line styles
\tikzstyle{force}= [>=latex,draw=blue,fill=blue]
\tikzstyle{internal_axis}= [densely dashed,gray,font=\small]
\tikzstyle{external_axis}= [font=\small]

\tikzset
    {
        extaxis/.pic = 
        {
            \draw[->] (0,0) -- (0,.7) node[left] {$+y$} ;
            \draw[->] (0,0) --(.7,0) node[right] {$+x$};
        }
}

\begin{tikzpicture}

\matrix[column sep=1cm] { % draws the objects in 3 columns 

    %% Sketch
    
    % grid for ease of drawing
    % \draw[step = 0.5cm, very thin, gray] (-1,-2) grid (4,2); % comment out for final figure

    % draws the triangle
    % use 'coordinate' when defining shapes
    \draw[plane] (0,-1) coordinate (base)
                     -- coordinate[pos=0.5] (mid) ++(\iangle:3) coordinate (top)
                     |- (base) -- cycle;

    \path (mid) node[m1,rotate=\iangle,yshift=0.25cm] (m1) {};
    \draw[pulley] (top) -- ++(\iangle:0.25) circle (0.25cm)
                   ++ (90-\iangle:0.5) coordinate (pulley);
    \draw[string] (m1.east) -- ++(\iangle:1.5cm) arc (90+\iangle:0:0.25)
                  -- ++(0,-1) node[m2] {};

    \draw[->] (base)++(\arcr,0) arc (0:\iangle:\arcr);
    \path (base)++(\iangle*0.5:\arcr+5pt) node {$\alpha$};
    %%
&
    %% Free body diagram of M
    \begin{scope}[rotate=\iangle]
        \node[m1,transform shape] (m1) {}; %transform shape causes the current “external” transformation matrix to be applied to the shape

        % Draw axes and help lines
        {[external_axis,transform shape] \draw (-2,-2)  pic {extaxis};}
        
        {[internal_axis,->]
            \draw (0,-1) -- (0,2) node[right] {$+y$};
            \draw (m1) -- ++(2,0) node[right] {$+x$};
            % Indicate angle. The code is a bit awkward.

            \draw[solid,shorten >=0.5pt] (\down-\iangle:\arcr)
                arc(\down-\iangle:\down:\arcr);
            \node at (\down-0.5*\iangle:1.3*\arcr) {$\alpha$};
        }

        % Forces
        {[force,->]
            % Assuming that Mg = 1. The normal force will therefore be cos(alpha)
            \draw (m1.center) -- ++(0,{cos(\iangle)}) node[above right] {$N$};
            \draw (m1.west) -- ++(-1,0) node[left] {$f_R$};
            \draw (m1.east) -- ++(1,0) node[above] {$T$};
        }

    \end{scope}
    % Draw gravity force. The code is put outside the rotated
    % scope for simplicity. No need to do any angle calculations. 
    \draw[force,->] (m1.center) -- ++(0,-1) node[below] {$Mg$};
    %%

&
    %%%
    % Free body diagram of m
    \node[m2] (m) {};
    \draw[internal_axis,->] (m) -- ++(0,-2) node[left] {$+$};
    {[force,->]
        \draw (m.north) -- ++(0,1) node[above] {$T'$};axis
        \draw (m.south) -- ++(0,-1) node[right] {$mg$};
    }

\\
};
\end{tikzpicture}

\end{document}