%------------------------------------------------------------------------------
% Author(s):
% Brad Bachu
%
% Copyright:
%  Copyright (C) 2020 Brad Bachu, Arjun Mohammed, Nicholas Sammy, Kerry Singh
%
%  This file is part of Applied-Mathematics-Unit2 and is distributed under the
%  terms of the MIT License. See the LICENSE file for details.
%
%  Description:
%     Year: 2015 
%     Module: 3
%     Question: 5
%------------------------------------------------------------------------------

\begin{subquestions}

%------------------------------------------------------------------------
% 5 (a)
%------------------------------------------------------------------------
\subquestion 

We are given that $m_1 = 5$ kg, $m_2 = 6$ kg, $\theta = \arcsin\left(\frac{1}{10}\right)$ and $\mu = \frac{1}{4}$. Representing the positions of $m_1$ and $m_2$ with the coordinates $(x_1,y_1)$ and $(x_2,y_2)$ respectively, we summarize this information in Fig.~\ref{2015:q5:fig:force1}

\begin{figure}
\begin{center}
\includegraphics{figures/particleOnPlane}
\caption{\label{2015:q5:fig:force1} A figure caption. The figure captions are
automatically numbered.}
\end{center}
\end{figure}


%------------------------------------------------------------------------
% 5 (a) (i)
%------------------------------------------------------------------------
\item See Fig.~\ref{2015:q5:fig:force1}

%------------------------------------------------------------------------
% 5 (a) (ii)
%------------------------------------------------------------------------
\item We would like to find the tension and acceleration of the particles.


The next step is to represent mathematically.
Now we apply Newton's 2nd law in component form for each component.
For $m_1$, since it is stationary in the $\hat{y}_1$ component throughout
its motion, have
\begin{align}
   \hat{y}_1 &: m_1 \ddot{y}_1 = N - m_1 g \cos\theta = 0  \,,\nn \\
   \hat{x}_1 &:m_1 \ddot{x}_1 = T - f_R \,.
\end{align}
For $m_2$, we only have to consider one component,
\begin{align}
   \hat{y}_2 &: m_2 \ddot{y}_2 = T^\prime - m_2g \,.
\end{align}
Since the string is light $m_s=0$, we have for any position along the string $x_s$,
\begin{align}
   \hat{x}_s: 0 = T - T^\prime \,.
\end{align}
Since the string is inelastic, the two masses move with the same speed and acceleration,
\begin{align}
   \ddot{x}_1 &= -\ddot{y}_2 \nn  \, ,\\
   \dot{x}_1 &= -\dot{y}_2  \, . 
\end{align}

Next, we solve and evaluate. We see that
\begin{align}
   T &= \frac{m_1m_2g(1 - \cos\theta)}{m_1 + m_2} \nn \\
   \ddot{x}_1 = - \ddot{y}_2 &= \frac{g(m_2 - m_1\cos\theta)}{m_1+m_2}
\end{align}

Thus, summarizing the results, we have the accelerations of each particle
\begin{align}
   \vec{a}_1 &=  () \hat{x}_1 + () \hat{y}_1 \nn \\
   \vec{a}_2 &=  () \hat{x}_2 + () \hat{y}_2  
\end{align}
with the tension, following the same format
\begin{align}
   \vec{T}_1 &= () \hat{x}_1 + () \hat{y}_1 \nn \\
   \vec{T}_2 &= () \hat{x}_2 + () \hat{y}_2 
\end{align}

%------------------------------------------------------------------------
% 1 (a) (iii)
%------------------------------------------------------------------------
\item



%------------------------------------------------------------------------
% 1 (b)
%------------------------------------------------------------------------
\subquestion We are given that $\vec{x}(t=0) = \vec{0}$ and $\vec{v}(t) = 5t^2 \hat{i} - (t-4)\hat{j}$
%------------------------------------------------------------------------
% 1 (b) (i)
%------------------------------------------------------------------------
\item We want to find $\vec{a}(t=3)$.
\begin{align}
   \vec{a}(t) & = \frac{d^2 \vec{x}}{dt^2} = \frac{d\vec{v}}{dt} \nn \\
   &= 10 t \hat{i}  - \hat{j}  \nn \\
   \vec{a} (t=3) &= 30 \hat{i} - \hat{j}
\end{align}

\item We want to find $\vec{x}(t=3)$.
\begin{align}
   \vec{v} &= \frac{d\vec{x}}{dt} \nn \\
   \int_{t_1}^{t_2} \vec{v} dt &=  \int_{t_1}^{t_2} \frac{d\vec{x}}{dt}  dt\nn \\
         &=\vec{x}(t_2) - \vec{x}(t_1)
\end{align}

Solving the integral, 
\begin{align}
   \vec{x}(t=3) - \vec{x}(t=0) &= \left[ \frac{5}{3}t^3 \hat{i} - \left(\frac{t^2}{2} - 4t\right)\hat{j}\right]^{3}_0 \nn \\
   &= 45 \hat{i} + 7.5 \hat{j}
\end{align}


%------------------------------------------------------------------------
% 1 (c)
%------------------------------------------------------------------------
\subquestion We are given that $\vec{v}(t=0) = 0$ (at rest) and $\vec{F}_r \propto -\vec{v}$ (resistance to motion proportional to $\vec{v}$). 


\begin{figure}
\begin{center}
\includegraphics{figures/particleMedium.pdf}
\caption{\label{2015:q5:fig:particle1} A figure caption. The figure captions are
automatically numbered.}
\end{center}
\end{figure}

Look at Fig.~\ref{2015:q5:fig:particle1}
\begin{align}
   m \vec{a} &= \sum \vec{F} \nn\\
    &= \vec{F}_w + F_r \nn \\
   &= -m g - c \vec{v} \nn \\
   m \frac{d\vec{v}}{dt} &= -mg - c \vec{v} \nn \\
    \frac{d\vec{v}}{dt} &= -g - \frac{c}{m} \vec{v}
\end{align}
where $c$ is some undetermined constant of proportionality. This answer is completely correct once the coordinate system is specified as in the figure.

An important point here is that if our coordinate system, our answer would change slightly. We could have picked the system shown in Fig.~\ref{2015:q5:fig:particle2}
\begin{figure}
\begin{center}
\includegraphics{figures/particleMedium2.pdf}
\caption{\label{2015:q5:fig:particle2} A figure caption. The figure captions are
automatically numbered.}
\end{center}
\end{figure}
Then Newton's 2nd law would read,
\begin{align}
   m \vec{a} &= \sum \vec{F} \nn\\
   &= \vec{F}_w + F_r \nn \\
   &= m g - c \vec{v} \nn \\
   m \frac{d\vec{v}}{dt} &= mg - c \vec{v} \nn \\
   \frac{d\vec{v}}{dt} &= g - \frac{c}{m} \vec{v} 
\end{align}


\end{subquestions}

