%------------------------------------------------------------------------------
% Author(s): Varaun Ramgoolie
%
% Copyright:
%  Copyright (C) 2020 Brad Bachu, Arjun Mohammed, Nicholas Sammy, Kerry Singh
%
%  This file is part of Applied-Mathematics-Unit2 and is distributed under the
%  terms of the MIT License. See the LICENSE file for details.
%
%  Description:
%     Year: 2015
%     Module: 2
%     Question: 3
%------------------------------------------------------------------------------

\begin{subquestions}
	
	%------------------------------------------------------------------------
	% 3 (a)
	%------------------------------------------------------------------------
	
\subquestion

It is given that 20\% of bolts manufactured at a factory are defective. Therefore, we shall say the probability a bolt is defective, $P(D)= 0.2$. 

\begin{subsubquestions}
	
	
	%-------------------------------------------------------------------------------------
	% 3 (a) i
	%------------------------------------------------------------------------------------
\subsubquestion	

In a random sample of 15 bolts, we want to find the probability that more than 1 bolt is defective. From Def \ref{mod2:defn:Binomial}, we see that this can be modelled using a Binomial Distribution.\\

Let X be "the number of defective bolts in some sample" \\
Therefore, 
\begin{equation}
	 X \sim \text{Bin}(n,0.2). 	\\ \nn
\end{equation}



We want to find the probability that, in a sample of 15 bolts, more than 1 is defective. Mathematically, we want to find $P(X>1)$.

Using the Binomial distribution and setting $n=15$, we get that
\begin{equation}
	P(X = x) = { 15 \choose x} 0.2^x (1-0.2)^{15-x} . 
\end{equation}

To solve this question, we can either find the sum of the different probabilities for defective bolts from 2 to 15 OR we can find the probability that the number of defective bolts is $\leq$ 1. This is because of the fact that $P(X>1)=1-P(X\leq1)$.  \\

Using the latter, we see that it is much simpler to calculate $P(X\leq1) $   by noticing that 
\begin{equation}
	P(X \leq 1)  =  P(X=1)+P(X=0). 
\end{equation}

Solving this, we get
\begin{align}
	P(X=0)&={ 15 \choose 0} 0.2^0 (1-0.2)^{15-0} \nn \\
	&=1\times1\times0.8^{15} \nn \\
	&=0.8^{15}.\nn \\ \nn \\
	P(X=1)&={ 15 \choose 1} 0.2^1 (1-0.2)^{15-1} \nn \\
	&=15 \times 0.2 \times 0.8^{14} \nn \\
	&=3\times 0.8^{14}. 
\end{align}

Evaluating the probability,
\begin{align}
	P(X>1)&=1-P(X\leq1) \nn \\
	&=1-(P(X=0)+P(X=1)) \nn \\
	&=1-(0.8^{15}+3\times0.8^{14}) \nn \\
	&=1- 0.167 \nn \\
	&=0.833. \\ \nn
\end{align}


%------------------------------------------------------------------------
% 3 a ii
%--------------------------------------------------------------------------

\subsubquestion
We now want to find the smallest value of $n$ that satisfies the following criteria: \\The ratio of the Standard Deviation of X to the Mean of X is less than 0.1.\\

Mathematically, we can express this as
\begin{equation}
	\frac{\sqrt{Var[X]}}{E[X]}< 0.1.  \\ 
\end{equation}

From Equations \ref{mod2:eq:Binomial:Mean} and \ref{mod2:eq:Binomial:Variance}, we can simplify this expression to
\begin{equation}
	\frac{\sqrt{np(1-p)}}{np}<0.1 \implies \sqrt{\frac{np(1-p)}{n^2p^2}}<0.1.  \\ 
\end{equation}

Simplifying this expression yields,
\begin{equation}
	\sqrt{\frac{1-p}{np}}<0.1.
\end{equation}

Now, we evaluate the expression and solve for $n$,
\begin{align}
		&\implies \sqrt{\frac{0.8}{0.2n}}<0.1 \nn \\
		&\implies \sqrt{\frac{4}{n}}<0.1 \nn \\
		&\implies \frac{2}{\sqrt{n}}<0.1 \nn \\
		&\implies \frac{2}{0.1}<\sqrt{n} \nn \\
		&\implies 20<\sqrt{n}.  \\ \nn
\end{align}

Lastly, we square the expression to give
\begin{equation}
	400<n.  
\end{equation}

We get that $n$ must be an integer that is greater than $400$. \\
Therefore, the smallest value of $n$ that satisfies our given criteria is $n=401$. \\

\end{subsubquestions}  

%---------------------
% 3 b
%----------------------------------

\subquestion

We are given that $X \sim \text{Bin} (500,0.005)$. In order to find $P(X\leq2)$, we can use a Poisson Approximation to the Binomial Distribution. \\ \\
From Def \ref{mod2:defn:PoissonApproxBinomial}, we know that certain conditions must hold for a Binomial Distribution to be approximated by a Poisson Distribution. \\ \\
From the given distribution, we can see that $n=500>50$ and $np=2.5<5$. \\

Therefore, by Def \ref{mod2:defn:PoissonApproxBinomial} again, we see that X can be apporximated as $X \sim \text{Pois} (2.5)$.\\

Now, we have $X \sim \text{Pois} (2.5)$ and we want to find $P(X\leq2)$. \\

We can notice that, for a Poisson Distribution, 
\begin{align}
	P(X \leq 2) = P(0) + P(1) + P(2). \\ \nn
\end{align}

By using Eq. \ref{mod2:eq:PoissonDist}, we can evaluate that, 

\begin{align}
	P(X = 0) & =\frac{ 2.5 ^ 0 \times e^{-2.5}}{0!} \nn \\
  & =\frac{1 \times e^{-2.5}}{1} \nn \\
  & =e^{-2.5} \nn \\
  & = 0.082.
\end{align}

\begin{align}
	P(X = 1) & =\frac{ 2.5 ^ 1 \times e^{-2.5}}{1!} \nn \\
	& =\frac{2.5 \times e^{-2.5}}{1} \nn \\
	& =2.5 \times e^{-2.5} \nn \\
	& = 0.205. 
\end{align}

\begin{align}
	P(X = 2) & =\frac{ 2.5 ^ 2 \times e^{-2.5}}{2!} \nn \\
	& =\frac{2.5^2 \times e^{-2.5}}{2} \nn \\
	& = 0.257. \\ \nn
\end{align}
	
	Finally, solving for $P(X \leq 2)$, we get 
	
\begin{align}
		P(X \leq 2) & = 0.082 + 0.205 + 0.257 \nn \\
		& = 0.544 \\ \nn
\end{align}


%%%%%%%%%%%%%%%%%%%%%%%%%%%%%%%%%%%%%%%%%%%%
%%% 3 c %%%%%%%%%%%%%%%%%%%%%%%%%%%%%%%%%%
%%%%%%%%%%%%%%%%%%%%%%%%%%%%%%%%%%%%%%%%%%%%%%
\subquestion

We are given that $X \sim \text{Pois} (20)$. In order to find, $P(X\leq 25)$, we must use a Normal Approximation to the Poisson Distribution as seen in Def \ref{mod2:def:NormalApproxToPois:Definition}.\\

Therefore, by Note \ref{mod2:note:NormalApproxtoPois:Continuity} and using a continuity correction, we will find the probabilty that $P(X \leq 25.5)$.

Standardizing using Property \ref{mod2:note:Normal:Property2} in Note \ref{mod2:note:Normal:Properties}, we evaluate the probability as follows,

\begin{align}
	P(X \leq 25.5) &= P(Z< \frac{25.5-20}{\sqrt{20}}) \nn \\
	&=P(Z<1.23) \nn \\ 
	&=0.891 \\ \nn 
\end{align}

%%%%%%%%%%%%%%%%%%%%%%%%%%%%%%%%%%%%%%%%%%%%%%%%%%%%%
%%%% 3 d %%%%%%%%%%%%%%%%%%%%%%%%%%%%%%%%%%%%%%%%%%%
%%%%%%%%%%%%%%%%%%%%%%%%%%%%%%%%%%%%%%%%%%%%%%%%%%

\subquestion

We are given that $X \sim \text{Geo} (0.3)$. \\
To find $P(X<4)$, we can either sum all the individual probabilites from $X=1$ to $3$ OR we can find $P(X>3)$ using the property of \ref{mod2:eq:Geometric:Prop}. This is because of the fact that $P(X<4)=1-P(X>3)$.

Evaluating this, we get,
\begin{align}
	P(X<4) &= 1-P(X>3) \nn \\
	&=1-0.7^3 \nn \\
	&=1-0.343 \nn \\
	&=0.657 \\ \nn		
\end{align}


\end{subquestions}
