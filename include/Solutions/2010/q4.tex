%------------------------------------------------------------------------------
% Author(s):
% Varaun Ramgoolie
% Copyright:
%  Copyright (C) 2020 Brad Bachu, Arjun Mohammed, Varaun Ramgoolie, Nicholas Sammy
%
%  This file is part of Applied-Mathematics-Unit2 and is distributed under the
%  terms of the MIT License. See the LICENSE file for details.
%
%  Description:
%     Year: 2010
%     Module: 2
%     Question: 4
%------------------------------------------------------------------------------

%------------------------------------------------------------------------------
% 4 a
%------------------------------------------------------------------------------

\begin{subquestions}

\subquestion

We will define $X$ as the continuous random variable representing the lengths of wood produced. We can therefore express $X$ as,
\begin{equation}
	X \sim N(3, \sigma^2) \,,
\end{equation}	
where $\sigma$ is the standard deviation of $X$.
	
\begin{subsubquestions}
	
\subsubquestion

We should note that we are given,
\begin{equation}
	P(X<2.75) = 0.15 \,.
\end{equation}
	
To make sense of this, we must first standardize the random variable $X$ to the normal random variable $Z$ as follows,
\begin{align}
	P(X<2.75) = P\left(\frac{X - \mu}{\sigma}<\frac{2.75-\mu}{\sigma}\right) & = 0.15 \nn \\
	            P\left(Z <\frac{2.75-3}{\sigma}\right) & = 0.15 \nn \\
	            P\left(Z <\frac{-0.25}{\sigma}\right) & = 0.15 \nn \\
	            \implies \Phi\left(\frac{-0.25}{\sigma}\right) & = 0.15 \,.
\end{align}

From our Z-tables, we see that $\Phi(-1.04)=0.15$. Thus, we get that,
\begin{align}
	\frac{-0.25}{\sigma} & = -1.04 \nn \\
	\implies \sigma & = \frac{-0.25}{-1.04} \nn \\
	                & \approx 0.24 \,.
\end{align}

%------------------------------------------------------------------------------

\subsubquestion

We have found that,
\begin{equation}
	X \sim N(3,0.24^2) \,.
\end{equation}

By standardizing from $X$ to $Z$, we get that,
\begin{align}
	P(X>3.2) & = P\left(\frac{X-\mu}{\sigma}>\frac{3.2-\mu}{\sigma} \right) \nn \\
	         & = P\left(Z>\frac{3.2-3}{0.24} \right) \nn \\
	         & = P\left(Z>\frac{0.2}{0.24} \right) \nn \\
	         & = 1 - \Phi\left(\frac{5}{6}\right) \nn \\
	         & = 1 - 0.798 \nn \\
	         & = 0.202 \,.
\end{align}

\end{subsubquestions}

%------------------------------------------------------------------------------
% 4 b
%------------------------------------------------------------------------------

\subquestion

\begin{subsubquestions}
	
\subsubquestion

The hypotheses for this test are,
\begin{itemize}
	\item $H_0$: These is no association between the number of employees working overtime and the distance from their home.
	\item $H_1$: These is an association between the number of employees working overtime and the distance from their home.
\end{itemize}
	
%------------------------------------------------------------------------------

\subsubquestion

We should note that, if there is no association between the number of employees working overtime and the distance from their home, the expected value for the number of employees working overtime would be 20 $(\frac{\text{Total}=100}{5})$. 

%------------------------------------------------------------------------------

\subsubquestion
We are asked to perform a $\chi^2$ goodness-of-fit test at the 5\% significance level. From Section(placeholder for CHI Squared notes), we know that,
\begin{equation}
	\chi^2_{\text{calc.}} = \sum_{\forall k} \left( \frac{(O_k-E_k)^2}{E_k}\right) \,.
\end{equation}

 See \rtab{2010:q4:ChiTab}.
\begin{table}[H]
	\centering
	\begin{tabular}{|c|c|c|c|}
		\hline
		Distances & Observed Number of Employees, O & Expected Number of Employees, E & $\frac{(O-E)^2}{E}$\\
		\hline 
		10-14 & 30 & 20 & 5 \\
		15-19 & 25 & 20 & 1.25 \\
		20-24 & 14 & 20 & 1.8 \\
		25-29 & 19 & 20 & 0.05 \\
		30-34 & 12 & 20 & 3.2 \\
		\hline
		Total, $\chi^2_{\text{calc.}}$ & & & 11.3 \\
		\hline
	\end{tabular}
	\caption{\label{2010:q4:ChiTab} $\chi^2$ Table.}
\end{table}	

We know that the number of degrees of freedom, $\nu$, is,
\begin{equation}
	\nu = \text{Number of classes}-1 = 5-1 = 4\,.
\end{equation}

At the 5\% significance level, we know that we will reject $H_0$ iff,
\begin{align}
	&\chi^2_{\text{calc.}} > \chi^2_{\alpha}(\nu) \nn \\
	\implies & \chi^2_{\text{calc.}} > \chi^2_{0.05}(4) \nn \\
	\implies & \chi^2_{\text{calc.}} > 9.488 \,.
\end{align} 

Therefore, from \rtab{2010:q4:ChiTab}, since $11.3>9.488$, we will reject $H_0$.
\end{subsubquestions}

\end{subquestions}
