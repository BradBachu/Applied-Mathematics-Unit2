%------------------------------------------------------------------------------
% Author(s):
% Varaun Ramgoolie
% Copyright:
%  Copyright (C) 2020 Brad Bachu, Arjun Mohammed, Varaun Ramgoolie, Nicholas Sammy
%
%  This file is part of Applied-Mathematics-Unit2 and is distributed under the
%  terms of the MIT License. See the LICENSE file for details.
%
%  Description:
%     Year: 2010
%     Module: 2
%     Question: 4
%------------------------------------------------------------------------------

%------------------------------------------------------------------------------
% 4 a
%------------------------------------------------------------------------------

\begin{subquestions}

\subquestion

Let $X$ be the continuous random variable representing the lengths of wood produced,
\begin{equation}
	X \sim N(3, \sigma^2) \,.
\end{equation}	

We are also given that,
\begin{equation}
	P(X<2.75) = 0.15 \,.
\end{equation}
	
\begin{subsubquestions}
	
\subsubquestion

We can use the information given to determine $\sigma$.	
We begin with standardizing the random variable $X$ to the normal random variable $Z$ as follows,
\begin{align}
	P(X<2.75) = P\left(\frac{X - \mu}{\sigma}<\frac{2.75-\mu}{\sigma}\right) & = 0.15 \nn \\
	            P\left(Z <\frac{2.75-3}{\sigma}\right) & = 0.15 \nn \\
	            P\left(Z <\frac{-0.25}{\sigma}\right) & = 0.15 \nn \\
	            \implies \Phi\left(\frac{-0.25}{\sigma}\right) & = 0.15 \,.
\end{align}

From our Z-tables, we see that $\Phi(-1.04)=0.15$. Thus, 
\begin{align}
	\frac{-0.25}{\sigma} & = -1.04 \nn \\
	\implies \sigma & = \frac{-0.25}{-1.04} \nn \\
	                & \approx 0.24 \,.
\end{align}

%------------------------------------------------------------------------------

\subsubquestion

We can update our distritbuion with the new information,
\begin{equation}
	X \sim N(3,0.24^2) \,.
\end{equation}

To determine $P(X>3.2)$, we can proceed by standardizing standardizing $X$ to $Z$ as follows,
\begin{align}
	P(X>3.2) & = P\left(\frac{X-\mu}{\sigma}>\frac{3.2-\mu}{\sigma} \right) \nn \\
	         & = P\left(Z>\frac{3.2-3}{0.24} \right) \nn \\
	         & = P\left(Z>\frac{0.2}{0.24} \right) \nn \\
	         & = 1 - \Phi\left(\frac{5}{6}\right) \nn \\
	         & = 1 - 0.798 \nn \\
	         & = 0.202 \,.
\end{align}

\end{subsubquestions}

%------------------------------------------------------------------------------
% 4 b
%------------------------------------------------------------------------------

\subquestion

\begin{subsubquestions}
	
\subsubquestion

The hypotheses for this test are,
\begin{itemize}
	\item $H_0$: These is no association between the number of employees working overtime and the distance from their home.
	\item $H_1$: These is an association between the number of employees working overtime and the distance from their home.
\end{itemize}
	
%------------------------------------------------------------------------------

\subsubquestion

We should note that, if there is no association between the number of employees working overtime and the distance from their home, the expected value for the number of employees working overtime would be 20 $(\frac{\text{Total}=100}{5})$. 

%------------------------------------------------------------------------------

\subsubquestion
We are asked to perform a $\chi^2$ goodness-of-fit test at the 5\% significance level. From Section \ref{Section:ChiSquareTest}, we know that,
\begin{equation}
	\chi^2_{\text{calc.}} = \sum_{\forall k} \left( \frac{(O_k-E_k)^2}{E_k}\right) \,.
\end{equation}

We summarize the details of this calculation in \rtab{2010:q4:ChiTab}.
\begin{table}[H]
	\centering
	\begin{tabular}{|c|c|c|c|}
		\hline
		Distances & Observed Employees, O & Expected Employees, E & $\frac{(O-E)^2}{E}$\\
		\hline 
		10-14 & 30 & 20 & 5 \\
		15-19 & 25 & 20 & 1.25 \\
		20-24 & 14 & 20 & 1.8 \\
		25-29 & 19 & 20 & 0.05 \\
		30-34 & 12 & 20 & 3.2 \\
		\hline
		Total, $\chi^2_{\text{calc.}}$ & & & 11.3 \\
		\hline
	\end{tabular}
	\caption{\label{2010:q4:ChiTab} $\chi^2$ Table.}
\end{table}	

The number of degrees of freedom, $\nu$, is,
\begin{equation}
	\nu = \text{Number of classes}-1 = 5-1 = 4\,.
\end{equation}

At the 5\% significance level, we that we can reject $H_0$ if and only if,
\begin{align}
	&\chi^2_{\text{calc.}} > \chi^2_{\alpha}(\nu) \nn \\
	\implies & \chi^2_{\text{calc.}} > \chi^2_{0.05}(4) \nn \\
	\implies & \chi^2_{\text{calc.}} > 9.488 \,.
\end{align} 

Therefore, from \rtab{2010:q4:ChiTab}, since $11.3>9.488$, we reject $H_0$.
\end{subsubquestions}

\end{subquestions}
