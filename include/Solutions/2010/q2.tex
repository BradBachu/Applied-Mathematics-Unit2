%------------------------------------------------------------------------------
% Author(s):
% Varaun Ramgoolie
% Copyright:
%  Copyright (C) 2020 Brad Bachu, Arjun Mohammed, Nicholas Sammy, Kerry Singh
%
%  This file is part of Applied-Mathematics-Unit2 and is distributed under the
%  terms of the MIT License. See the LICENSE file for details.
%
%  Description:
%     Year: 2010
%     Module: 1
%     Question: 2
%------------------------------------------------------------------------------

\begin{subquestions}

%------------------------------------------------------------------------------
% 1 a--------------------------------------------------------------------------
%------------------------------------------------------------------------------

\subquestion

See the switching circuit below.

\begin{circuitikz}
\draw [color=black, thin] (0,0) -- (2,0);
\draw [color=black, thin] (2,0) -- (2,2);
\draw [color=black, thin] (2,0) -- (2,-2);

\draw (2,2) to[normal open switch, *-*](6,2);
\draw (6,2) to[normal open switch, *-*](10,2);

\path (2,2) -- (6,2) node[pos=0.5,below]{A};
\path (6,2) -- (10,2) node[pos=0.5,below]{B};

\draw (2,-2) to[normal open switch, *-*](6,-2);
\draw (6,-2) to[normal open switch, *-*](10,-2);

\path (2,-2) -- (6,-2) node[pos=0.5,below]{B};
\path (6,-2) -- (10,-2) node[pos=0.5,below]{C};

\draw (2,0) to[normal open switch, *-*](6,0);
\draw (6,0) to[normal open switch, *-*](10,0);

\path (2,0) -- (6,0) node[pos=0.5,below]{A};
\path (6,0) -- (10,0) node[pos=0.5,below]{C};

\draw [color=black, thin] (10,2) -- (10,0);
\draw [color=black, thin] (10,-2) -- (10,0);

\draw [color=black, thin] (10,0) -- (12,0);

\end{circuitikz}

%------------------------------------------------------------------------------
% 1 b--------------------------------------------------------------------------
%------------------------------------------------------------------------------

\subquestion

From \rtab{2011:q2:tab:TruthTab1}, we can see that $(\sim p ~\lor \sim q) \implies (p ~\land \sim q)$ always takes the same truth value of $p$.

\begin{table}[ht]
	\centering
	\begin{tabular}{|c|c|c|c|c|c|c|}
		\hline
		p & q & $\sim$ p & $\sim$ q & ($\sim$ p $\lor$ $\sim$ q) & (p $\land$ $\sim$ q) & ($\sim$ p $\lor$ $\sim$ q) $\implies$ (p $\land$ $\sim$ q) \\
		\hline
		0 & 0 & 1 & 1 & 1 & 0 & 0 \\
		0 & 1 & 1 & 0 & 1 & 0 & 0 \\
		1 & 0 & 0 & 1 & 1 & 1 & 1 \\
		1 & 1 & 0 & 0 & 0 & 0 & 1 \\
		\hline
	\end{tabular}
	\caption{\label{2011:q2:tab:TruthTab1} Showing the truth values of $(\sim p ~\lor \sim q) \implies (p ~\land \sim q)$}\,.
\end{table}

%------------------------------------------------------------------------------
% 1 c--------------------------------------------------------------------------
%------------------------------------------------------------------------------

\subquestion

Since we are given that $a \implies b \equiv ~\sim a \lor b$\,, we need to draw the circuit for $\sim a \lor b$\,. \\
See \rfig{2010:q2:fig:Circuit1}.

\begin{figure}
	\begin{center}
		\includegraphics{../2010/figures/2010q2Circuit1}
		\caption{\label{2010:q2:fig:Circuit1} Showing the logic gate equivalent of $\sim a \lor b$.}
	\end{center}
\end{figure}

%------------------------------------------------------------------------------
% 1 d--------------------------------------------------------------------------
%------------------------------------------------------------------------------

\subquestion

\begin{subsubquestions}
	
\subsubquestion

See \rtab{2011:q2:tab:CritPath}.

\begin{table}[ht]
	\centering
	\begin{tabular}{|c|c|c|c|}
		\hline
		Activity & Earliest Start Time & Latest Start Time & Float Time \\
		\hline
		A & 0 & 0 & 0 \\
		B & 6 & 6 & 0 \\
		C & 16 & 16 & 0 \\
		D & 6 & 17 & 11 \\
		E & 18 & 19 & 1 \\
		F & 18 & 18 & 0 \\
		G & 22 & 22 & 0 \\
		\hline
	\end{tabular}
	\caption{\label{2011:q2:tab:CritPath} Showing the float time of the activities.}
\end{table}

\subsubquestion

From \rtab{2011:q2:tab:CritPath} and using \rdef{mod1:defn:FloatTime} and \rdef{mod1:defn:CritPath}, we can see that the crtical path is,

\begin{equation}
	\text{Start} \rightarrow A \rightarrow B \rightarrow C \rightarrow F \rightarrow G \rightarrow \text{Finish}\,.
\end{equation}

\end{subsubquestions}

\end{subquestions}


