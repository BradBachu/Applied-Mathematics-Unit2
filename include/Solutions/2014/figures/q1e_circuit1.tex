
\documentclass[crop,tikz]{standalone}

\usepackage{../../../../src/tikzappmath}

\usetikzlibrary{scopes}
\usetikzlibrary{arrows,shapes.gates.logic.US,shapes.gates.logic.IEC,calc}


\begin{document}
   \begin{tikzpicture}

      % make these defs here so that I can change all after
      % note that you do not have to keep them consistent
      % I just defined this to make initial coding easier
      \def\xspace{3} % set the horizonal distance between gates
      \def\yspace{2} % set the vertical distance between gates
      \def\shift{0.5cm} % sets some spacing to draw connecting lines

      % Draw gates at nodes
      % define first col of gates
      \node[and gate US,gatestyle] at (0,0) (x1) {};
      \node[or gate US, gatestyle] at (0,\yspace) (x2) {};
      \node[and gate US, gatestyle] at (0,2*\yspace) (x3) {};
      % define second col of gates
      % I do this to position it between the coordinates
      \node[or gate US, gatestyle] at (0.9*\xspace,-0.5*\yspace) (x4) {};
      \node[and gate US, gatestyle] at (0.9*\xspace,1.5*\yspace) (x5) {};
      % define third col of gates
      \node[or gate US, gatestyle] at (2*\xspace,-1*\yspace) (x6) {};
      % define fourth col of gates
      \node[or gate US, gatestyle] at (3.3*\xspace,1*\yspace) (x7) {};
      % Draw all wires

      % Connect all gates internally
      \draw (x1.output) -| ([xshift = -\shift]x4.input 1) node (l1) {} --++(\shift,0);
      \draw (x2.output) -| ([xshift = -\shift]x5.input 2) --++(\shift,0);
      \draw (x3.output) -| ([xshift = -\shift]x5.input 1) --++(\shift,0);
      \draw (x4.output) -| ([xshift = -\shift]x6.input 1) --++(\shift,0);
      \draw (x5.output) -| ([xshift = -\shift]x7.input 1) --++(\shift,0);
      \draw (x6.output) -| ([xshift = -\shift]x7.input 2) --++(\shift,0);
      % draw end piece
      \draw (x7.output) --++ (\shift,0);
      % input positions (some work goes into making the text separated)
      % inputs 1
      \draw (x1.input 1) -|++ (-\shift,\shift) --++(-\shift,0) node[left]{${p}$};
      \draw (x1.input 2) -|++ (-\shift,-\shift) --++(-\shift,0) node[left](in1){$r$};
      % inputs 2
      \draw (x2.input 1) -|++ (-\shift,\shift) --++(-\shift,0) node[left]{$p$};
      \draw (x2.input 2) -|++ (-\shift,-\shift) --++(-\shift,0) node[left]{$r$};
      % inputs 3
      \draw (x3.input 1) -|++ (-\shift,\shift) --++(-\shift,0) node[left]{$p$};
      \draw (x3.input 2) -|++ (-\shift,-\shift) --++(-\shift,0) node[left]{$q$};
      % hanging inputs
      \draw (x4.input 2) -|++ (-1.26*\xspace,0) node[left]{$q$};
      \draw (x6.input 2) -|++ (-2.36*\xspace,0) node[left]{$s$};

      % Insert labels
      \node[above] at ($(x3.output) + (\shift,0)$) (l3) {\gatetext$p \land q$};
      \node[above] at ($(x2.output) + (\shift,0)$) (l2) {\gatetext$p \lor r$};
      \node[above] at ($(x1.output) + (\shift,0)$) (l1) {\gatetext$p \land r$};
      \node[above] at ($(x4.output) + (1.5*\shift,0)$) (l4) {\gatetext$(p \land r)\lor q$};
      \node[above] at ($(x5.output) + (2.5*\shift,0)$) (l5) {\gatetext$(p\land q)\land (p\lor r)$};
      \node[above] at ($(x6.output) + (2.5*\shift,0)$) (l6) {\gatetext$(p\land r)\lor q\lor s$};
      \node[above] at ($(x7.output) + (\shift,0)$) (l7) {\gatetext$(x)$};
      \node[above] at ($(x7.output) + (\shift,0)$) (l7) {\gatetext$(x)$};
      \node[above] at ($(x6.output) - (4*\shift,3*\shift)$) (l8) {$(x) = p \land q \land (p \lor r) \land (q \lor (r \land p ) \lor s)$};

   \end{tikzpicture}

\end{document}