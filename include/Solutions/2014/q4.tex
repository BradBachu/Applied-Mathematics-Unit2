%------------------------------------------------------------------------------
% Author(s):
% Varaun Ramgoolie
% Copyright:
%  Copyright (C) 2020 Brad Bachu, Arjun Mohammed, Varaun Ramgoolie, Nicholas Sammy
%
%  This file is part of Applied-Mathematics-Unit2 and is distributed under the
%  terms of the MIT License. See the LICENSE file for details.
%
%  Description:
%     Year: 2014
%     Module: 2
%     Question: 4 
%------------------------------------------------------------------------------

\begin{subquestions}
%------------------------------------------------------------------------------
% 4 a
%------------------------------------------------------------------------------
\subquestion

\begin{subsubquestions}
	
\subsubquestion

Let the discrete random variable, $X$ be "the number of faults which occur in 15 meters of cloth". We can then say that $X$ follows a Poisson distribution with parameter, $\lambda = 3$. \\
We define the distribution of $X$ as
\begin{equation}
	X \sim \text{Pois}(3) \,.
\end{equation}
	
From Section ~\ref{mod2:section:Poisson}, we know that,
\begin{equation}
	P(X=x) = \frac{3^{x} \times e^{-3}}{x!} \,. \label{2014:q4:PoisEqn}
\end{equation}
	
Using \req{2014:q4:PoisEqn}, we can find,
\begin{align}
	P(X=4) & = \frac{3^{4} \times e^{-3}}{4!} \nn \\
	       & = \frac{81 \times e^{-3}}{24} \nn \\
	       & = 0.168 \,.
\end{align} 

%------------------------------------------------------------------------------

\subsubquestion

Since we know that the rate of faults in 15 meters of cloth is equal to 3, the rate of faults in 60 meters of cloth is $\frac{60}{15} \times 3 = 12$. \\
Let the discrete random variable, $Y$, be "the number of faults which occur in 60 meters of cloth". We can define the distribution of $Y$ as
\begin{equation}
	Y \sim \text{Pois}(12) \,.
\end{equation}

From Section ~\ref{mod2:section:Poisson},
\begin{equation}
	P(Y=y) = \frac{12^{y} \times e^{-12}}{y!} \,. \label{2014:q4:PoisEqn2}
\end{equation}

Since $Y$ is discrete, we know that,
\begin{align}
	P(Y \geq 2) & = 1 - P(Y<2) \nn \\
	            & = 1 - (P(Y=0) + P(Y=1)) \,.
\end{align}

Thus, using \req{2014:q4:PoisEqn2}, we can calculate,
\begin{align}
	P(Y \geq 2) & = 1 - (P(Y=0) + P(Y=1)) \nn \\
	            & = 1 - \left(\frac{12^{0} \times e^{-12}}{0!} + \frac{12^{1} \times e^{-12}}{1!} \right) \nn \\
	            & = 1 - \left(\frac{1 \times e^{-12}}{1} + \frac{12 \times e^{-12}}{1} \right) \nn \\
	            & = 1 - \left(13 \times e^{-12} \right) \nn \\
	            & = 1 - 0.00008 \nn \\
	            & = 0.99992
\end{align}

\end{subsubquestions}

%------------------------------------------------------------------------------
% 4 b
%------------------------------------------------------------------------------
	
\subquestion

We are given that the mass of oranges, $X$, can be modeled by a normal distribution. This can be expressed as follows,
\begin{equation}
	X \sim N(62.2, 3.6^2) \,.
\end{equation}

\begin{subsubquestions}
	
\subsubquestion

In order to find the probability that $X<60$, we must first standardize the random variable, $X$, to the normal random variable, $Z$. From Section ~\ref{mod2:section:Normal}, we standardize and calculate as follows,
\begin{align}
	P (X<60) & = P\left(\frac{X - \mu}{\sigma} < \frac{60 - \mu}{\sigma} \right) \nn \\
	         & = P\left(Z < \frac{60-62.2}{3.6} \right) \nn \\
	         & = P\left(Z < -\frac{2.2}{3.6} \right) \nn \\	
	         & = P\left(Z < -\frac{11}{18} \right) \nn \\
	         & = \Phi\left(-\frac{11}{18} \right) \nn \\
	         & = 1 - \Phi\left(\frac{11}{18} \right) \nn \\
	         & = 1 - 0.729 \nn \\
	         & = 0.271 \,.	
\end{align}
	
%------------------------------------------------------------------------------

\subsubquestion

Similar to \textbf{(b)(i)}, we standardize as follows,
\begin{align}
	P(61<X<64) & = P\left(\frac{61- \mu}{\sigma} < \frac{X - \mu}{\sigma} < \frac{64 - \mu}{\sigma} \right) \nn \\
	           & = P\left(\frac{61- 62.2}{3.6} < Z < \frac{64 - 62.2}{3.6} \right) \nn \\
	           & = P\left(-\frac{1.2}{3.6} < Z < \frac{1.8}{3.6} \right) \nn \\
	           & = P\left(-\frac{1}{3} < Z < \frac{1}{2} \right) \nn \\
	           & = \Phi\left(\frac{1}{2} \right) - \Phi\left(-\frac{1}{3}\right) \nn \\
	           & = \Phi\left(\frac{1}{2} \right) - \left(1 - \Phi\left(\frac{1}{3}\right) \right) \nn \\
	           & = 0.691 - (1 - 0.631) \nn \\
	           & = 0.322 \,.
\end{align}

\end{subsubquestions}

%------------------------------------------------------------------------------
% 4 c
%------------------------------------------------------------------------------

\subquestion

We are given the relevant probabilities of 2 independent random variables, $X$ and $Y$ (see \rtab{2014:q4:Tab1}).
\begin{table}[H]
	\centering
	\begin{tabular}{|c|c|c|c|}
		\hline
		$X$ & $P(X=x)$ & $Y$ & $P(Y=y)$ \\
		\hline
		0 & 0.2 & 0 & 0.2 \\
		1 & 0.3 & 1 & 0.1 \\
		2 & 0.5 & 2 & 0.3 \\
		  &     & 3 & 0.25 \\
		  &     & 4 & 0.15 \\
		  \hline
	\end{tabular}
	\caption{\label{2014:q4:Tab1} Probabilities of $X$ and $Y$.}
\end{table}

\begin{subsubquestions} 

\subsubquestion

$(X+Y=3)$ can happen in exactly 3 ways:
\begin{itemize}
	\item $X = 0$ and $Y = 3$
	\item $X = 1$ and $Y = 2$
	\item $X = 2$ and $Y = 1$
\end{itemize}

Since $X$ and $Y$ are independent, we know that
\begin{equation}
	P(X=x \land Y=y) = P(X=x) \times P(Y=y)
\end{equation}

We can therefore calculate $P(X+Y=3)$ as follows,
\begin{align}
	P(X+Y=3) & = P(X=0 ~\land~ Y=3) + P(X=1 ~\land~ Y=2) + P(X=2 ~\land~ Y=1) \nn \\
	         & = [P(X=0) \times P(Y=3)] + [P(X=1) \times P(Y=2)] + [P(X=2) \times P(Y=1)] \nn \\
	         & = [0.2 \times 0.25] + [0.3 \times 0.3] + [0.5 \times 0.1] \nn \\ 
	         & = 0.05 + 0.075+0.05 \nn \\
	         & = 0.175.
\end{align}

%------------------------------------------------------------------------------

\subsubquestion

\begin{subsubsubquestions}
	
\subsubsubquestion

We can use \rdef{mod2:defn:Discrete:Expectation} as follows,
\begin{align}
	E(X) & =  \sum_{\forall k} \left(x_k \times P(X=x_k) \right) \nn \\
	     & = (0 \times 0.2)+(1 \times 0.3)+(2 \times 0.5) \nn \\
	     & = 0 + 0.3 + 1 \nn \\
	     & = 1.3.
\end{align}
	
\subsubsubquestion

From \rdef{mod2:defn:Discrete:Variance}, we know,
\begin{equation}
	\var(X) = E(X^2) - (E(X))^2 \,.
\end{equation}

Using \rdef{mod2:defn:Discrete:SecondMoment}, we get that,
\begin{align}
	E(X^2) & = \sum_{\forall k} \left(x^2_k \times P(X=x_k) \right)  \nn \\
	       & = (0^2 \times 0.2) + (1^2 \times 0.3) + (2^2 \times 0.5)  \nn \\
	       & = (0 \times 0.2) + (1 \times 0.3) + (4 \times 0.5)  \nn \\
	       & = 0 + 0.3 + 2  \nn \\
	       & = 2.3 \,.
\end{align}

Thus, we can proceed by calculating,
	\begin{align}
		\var(X) & = E(X^2) - (E(X))^2 \nn \\
		& = 2.3 - 1.3^2 \nn \\
		& = 2.3 - 1.69 \nn \\
		& = 0.61.
	\end{align}
		
\subsubsubquestion

We can use \rdef{mod2:defn:Discrete:Expectation} as follows,
\begin{align}
	E(Y) & =  \sum_{\forall k} \left(y_k \times P(Y=y_k) \right) \nn \\
	& = (0 \times 0.2)+(1 \times 0.1)+(2 \times 0.3)+(3 \times 0.25)+(4 \times 0.15)\nn \\
	& = 0 + 0.1 + 0.6 + 0.75 + 0.6 \nn \\
	& = 2.05.
\end{align}

\subsubsubquestion

From \rdef{mod2:defn:Discrete:Variance}, we know,
\begin{equation}
	\var(Y) = E(Y^2) - (E(Y))^2 \,.
\end{equation}

Using \rdef{mod2:defn:Discrete:SecondMoment}, we get that,
\begin{align}
	E(Y^2) & =  \sum_{\forall k} \left(y^2_k \times P(Y=y_k) \right) \nn \\
	       & = (0^2 \times 0.2) + (1^2 \times 0.1) + (2^2 \times 0.3) + (3^2 \times 0.25) + (4^2 \times 0.15) \nn \\
	       & = (0 \times 0.2) + (1 \times 0.1) + (4 \times 0.3) + (9 \times 0.25) + (16 \times 0.15) \nn \\
	       & = 0 + 0.1 + 1.2 + 2.25 + 2.4 \nn \\
	       & = 5.95 \,.
\end{align}

Thus, we can proceed by calculating,
\begin{align}
	\var(Y) & = E(Y^2) - (E(Y))^2 \nn \\
	& = 5.95 - 2.05^2 \nn \\
	& = 5.95 - 4.2025 \nn \\
	& = 1.7475.
\end{align}

\end{subsubsubquestions}

%------------------------------------------------------------------------------

\subsubquestion

Using Section ~\ref{mod2:section:ExpectationVariance}, we can calculate the following,

\begin{subsubsubquestions}
	
\subsubsubquestion

\begin{align}
	E(3X-2Y) & = 3 \times E(X) - 2 \times E(Y) \nn \\
	         & = 3 \times 1.3 - 2 \times 2.05 \nn \\
	         & = 3.9 - 4.1 \nn \\
	         & = -0.2.
\end{align}

\subsubsubquestion

\begin{align}
	\var(3X-2Y) & = 3^2 \times \var(X) + 2^2 \times \var(Y) \nn \\
	           & = 9 \times 0.61 + 4 \times 1.7475 \nn \\
	           & = 5.49 + 6.99 \nn \\
	           & = 12.48.
\end{align}

\end{subsubsubquestions}

\end{subsubquestions}

\end{subquestions}

