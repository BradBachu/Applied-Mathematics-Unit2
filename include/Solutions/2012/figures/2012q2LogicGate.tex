%------------------------------------------------------------------------------
% Author(s):
% Varaun Ramgoolie
% Copyright:
%  Copyright (C) 2020 Brad Bachu, Arjun Mohammed, Varaun Ramgoolie, Nicholas Sammy
%
%  This file is part of Applied-Mathematics-Unit2 and is distributed under the
%  terms of the MIT License. See the LICENSE file for details.
%
%  Description:
%        Logic gate diagram for 2012 q2 e
%------------------------------------------------------------------------------

\documentclass[crop,tikz]{standalone}

\usepackage{../../../../src/tikzappmath}

\usetikzlibrary{scopes}
\usetikzlibrary{arrows,shapes.gates.logic.US,shapes.gates.logic.IEC,calc}


\begin{document}
	\begin{tikzpicture}
	
		\def\xspace{3} % set the horizonal distance between gates
		\def\yspace{2} % set the vertical distance between gates
		\def\shift{0.5cm} % sets some spacing to draw connecting lines
		
		% Drawing gates
		\node[not gate US,gatestyle] at (0,-2) (x1){};
		\node[not gate US,gatestyle] at (0,-4) (x2){};
		\node[and gate US,gatestyle] at (7,0) (x3) {};
		\node[or gate US,gatestyle] at (3,-3) (x4) {};
		
		% Connecting internals of gates
		\draw (x1.output) -| ([xshift = -\shift]x4.input 1) --++(\shift,0);
		\draw (x2.output) -| ([xshift = -\shift]x4.input 2) --++(\shift,0);
		\draw (x4.output) -| ([xshift = -\shift]x3.input 2) --++(\shift,0);
		
		% Setting the input names 
		\draw (x1.input) -|++ (-\shift,0) --++(-\shift,0) node[left,]{${p}$};
		\draw (x2.input) -|++ (-\shift,0) --++(-\shift,0) node[left,]{${q}$};
		\draw (x3.input 1) -|++ (-\shift,\shift) --++(-\shift,0) node[left,]{${p}$};
		
		% End piece
		\draw (x3.output) --++ (3*\shift,0);
		
		% Naming Gate Outputs
		\node[above] at ($(x1.output) + (\shift,0)$) (l1) {\gatetext${\sim p}$};
		\node[above] at ($(x2.output) + (\shift,0)$) (l1) {\gatetext${\sim q}$};
		\node[above] at ($(x3.output) + (2.5*\shift,0)$) (l1) {\gatetext${p \land (\sim p \, \lor \sim q)}$};
		\node[above] at ($(x4.output) + (\shift,0)$) (l1) {\gatetext${\sim p \, \lor \sim q}$};
		
	\end{tikzpicture}
	
\end{document}