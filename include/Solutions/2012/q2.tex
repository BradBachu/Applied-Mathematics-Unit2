%------------------------------------------------------------------------------
% Author(s):
% Varaun Ramgoolie
% Copyright:
%  Copyright (C) 2020 Brad Bachu, Arjun Mohammed, Varaun Ramgoolie, Nicholas Sammy
%
%  This file is part of Applied-Mathematics-Unit2 and is distributed under the
%  terms of the MIT License. See the LICENSE file for details.
%
%  Description:
%     Year: 2012
%     Module: 1
%     Question: 2
%------------------------------------------------------------------------------

\begin{subquestions}

%-----------------------------------------------------------------------------
% 2 a
%-----------------------------------------------------------------------------

\subquestion

\begin{subsubquestions}
	
\subsubquestion

From \rtab{2012:q2:TruthTab1}, we see that both $(p \implies r) \land (p \implies q)$ and $(p \implies (q \land r))$ are equivalent due to their corresponding truth values.
\begin{table}[ht]
	\centering
	\begin{tabular}{|c|c|c|c|c|c|c|c|}
		\hline
		p & q & r & p $\implies$ r & p $\implies$ q & (p $\implies$ r) $\land$ (p $\implies$ q) & q $\land$ r & (p $\implies$ (q $\land$ r)) \\
		\hline
		0 & 0 & 0 & 1 & 1 & 1 & 0 & 1 \\
		0 & 0 & 1 & 1 & 1 & 1 & 0 & 1 \\
		0 & 1 & 0 & 1 & 1 & 1 & 0 & 1 \\
		0 & 1 & 1 & 1 & 1 & 1 & 1 & 1 \\
		1 & 0 & 0 & 0 & 0 & 0 & 0 & 0 \\
		1 & 0 & 1 & 1 & 0 & 0 & 0 & 0  \\
		1 & 1 & 0 & 0 & 1 & 0 & 0 & 0 \\
		1 & 1 & 1 & 1 & 1 & 1 & 1 & 1 \\
		\hline
	\end{tabular}
\caption{\label{2012:q2:TruthTab1} Truth Table of the expressions.}
\end{table}

%-----------------------------------------------------------------------------

\subsubquestion

From \rtab{2012:q2:TruthTab2}, we see that the truth values of $(p \lor q) \implies p$ are not constant. Therefore, the expression is neither \textit{true} nor \textit{false}.
\begin{table}[ht]
	\centering
	\begin{tabular}{|c|c|c|c|}
		\hline
		p & q & p $\lor$ q & (p $\lor$ q) $\implies$ p \\
		\hline
		0 & 0 & 0 & 1 \\
		0 & 1 & 1 & 0 \\
		1 & 0 & 1 & 1 \\
		1 & 1 & 1 & 1 \\
		\hline
	\end{tabular}
	\caption{\label{2012:q2:TruthTab2} Showing the truth values of $(p \lor q) \implies p$\,.}
\end{table}
\end{subsubquestions}

%-----------------------------------------------------------------------------
% 2 b
%-----------------------------------------------------------------------------

\subquestion

Let $p$ $\equiv$ "2+3=5", $q$ $\equiv$ "2+2=4" and $r$ $\equiv$ "4+4=8". \\	
The expression in the question is a conditional statement. We can express it as, 
\begin{equation}
	p \implies \sim (q \land r)\,.
\end{equation}

From the statement, we know that $p$ is true, $q$ is true and $r$ is true. \\
Therefore, we get the truth value of,
\begin{align}
	T & \implies \sim(T \land T)\,, \nn \\
	\rightarrow T & \implies \sim(T)\,, \nn \\
	\rightarrow T & \implies F \equiv F\,.
\end{align}

%-----------------------------------------------------------------------------
% 2 c
%-----------------------------------------------------------------------------

\subquestion

\begin{subsubquestions}
	
\subsubquestion
Let $p$ $\equiv$ "It is hot" and $q$ $\equiv$ "It is sunny". \\

The statement in the question can be expressed as,

\begin{equation}
	\sim(\sim p \lor q)\,.
\end{equation}

%-----------------------------------------------------------------------------

\subsubquestion

Using \rdef{mod1:law:DeMorgan}, we can simplify the expression as follows,
\begin{equation}
		\sim(\sim p \lor q) \equiv p \, \land \sim q\,.
\end{equation}

\end{subsubquestions}

%-----------------------------------------------------------------------------
% 2 d
%-----------------------------------------------------------------------------

\subquestion

\begin{subsubquestions}

\subsubquestion
The Boolean expression for the switching circuit is,

\begin{equation}
	A \land (\sim B \lor C)\,.
\end{equation}

%-----------------------------------------------------------------------------

\subsubquestion

From \rtab{2012:q2:TruthTab3}, the 3 combinations that will put the circuit on are,
\begin{align}
	& A (\text{on}),\, B (\text{off}),\, C (\text{off})\,, \nn \\
	& A (\text{on}),\, B (\text{off}),\, C (\text{on})\,, \nn \\
	& A (\text{on}),\, B (\text{on}),\, C (\text{on})\,.
\end{align}

\begin{table}[ht]
	\centering
	\begin{tabular}{|c|c|c|c|c|c|}
		\hline
		A & B & C & $\sim$ B & $\sim$ B $\lor$ C & A $\land$ ($\sim$ B $\lor$ C) \\
		\hline
		0 & 0 & 0 & 1 & 1 & 0 \\
		0 & 0 & 1 & 1 & 1 & 0 \\
		0 & 1 & 0 & 0 & 0 & 0 \\
		0 & 1 & 1 & 0 & 1 & 0 \\
		1 & 0 & 0 & 1 & 1 & 1 \\
		1 & 0 & 1 & 1 & 1 & 1 \\
		1 & 1 & 0 & 0 & 0 & 0 \\
		1 & 1 & 1 & 0 & 1 & 1 \\
		\hline
	\end{tabular}
	\caption{\label{2012:q2:TruthTab3} Truth Table of $A \land (\sim B \lor C)$.}
\end{table}

\end{subsubquestions}

%-----------------------------------------------------------------------------
% 2 e
%-----------------------------------------------------------------------------

\subquestion

See \rfig{2012:q2:LogicGate}.
\begin{figure}
	\begin{center}
		\includegraphics{../2012/figures/2012q2LogicGate}
		\caption{\label{2012:q2:LogicGate} Showing logic gate representation of $p \land (\sim p \, \lor \sim q)$.}
	\end{center}
\end{figure}


\end{subquestions}
