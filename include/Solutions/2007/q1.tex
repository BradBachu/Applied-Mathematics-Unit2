%------------------------------------------------------------------------------
% Author(s):
% Varaun Ramgoolie
% Copyright:
%  Copyright (C) 2020 Brad Bachu, Arjun Mohammed, Varaun Ramgoolie, Nicholas Sammy
%
%  This file is part of Applied-Mathematics-Unit2 and is distributed under the
%  terms of the MIT License. See the LICENSE file for details.
%
%  Description:
%     Year: 2008 May
%     Module: 1
%     Question: 1 
%------------------------------------------------------------------------------

\begin{subquestions}
	
%------------------------------------------------------------------------------
% 1 a--------------------------------------------------------------------------
%------------------------------------------------------------------------------

\subquestion

Let $x$ be the number of newspaper advertisements and let $y$ be the number of television advertisements.	
The objective function $C$, the sale of cars, that we want to maximize can be expressed as,
\begin{equation}
	C =2x+5y \,.
\end{equation} 

The constraints of this linear programming model are as follows,
\begin{align}
	\text{Total budget:} \ & 1500x + 5000y \leq 5000 \,, \nn \\
	\text{Newspaper ad budget:} \ & 1500x \leq 30000 \,, \nn \\
	\text{TV ad budget:} \ & 5000y \geq 25000 \,, \nn \\
	\text{Relative number of advertisements:} \ & x \leq 2y \,.
\end{align}

%------------------------------------------------------------------------------
% 1 b--------------------------------------------------------------------------
%------------------------------------------------------------------------------

\subquestion

\begin{subsubquestions}

%-----------------------------------------------------------------------------
	
\subsubquestion

The feasible region is shaded in \rfig{2008M:q1:fig:Graph}.
\begin{figure}[H]
	\begin{center}
		\includegraphics[scale=1]{../2007/figures/2008Mq1Graph}
		\caption{\label{2008M:q1:fig:Graph} Linear Programming Graph.}
	\end{center}
\end{figure}

%-----------------------------------------------------------------------------------------

\subsubquestion

We can solve the problem by performing a Tour of the Vertices (\rdef{mod1:defn:TourOfVertices}),

\begin{table}[H]
	\centering
	\begin{tabular}{|c|c|}
		\hline
		Vertice & $P = 2x + 7y$ \\
		\hline
		(2, 2) & 18 \\
		(2, 3) & 25 \\
		(4, 1) & 15 \\
		(2.4, 1) & 11.8 \\
		\hline
	\end{tabular}
	\caption{\label{2007:q1:tab:P} Tour of Vertices}
\end{table}

Thus, the maximum value of $P$ is $25$, when $x=2$ and $y=3$.

\end{subsubquestions}

\end{subquestions}

