%------------------------------------------------------------------------------
% Author(s):
% Varaun Ramgoolie
% Copyright:
%  Copyright (C) 2020 Brad Bachu, Arjun Mohammed, Nicholas Sammy, Kerry Singh
%
%  This file is part of Applied-Mathematics-Unit2 and is distributed under the
%  terms of the MIT License. See the LICENSE file for details.
%
%  Description:
%     Year: 2008 May
%     Module: 1
%     Question: 2 
%------------------------------------------------------------------------------

\begin{subquestions}

%------------------------------------------------------------------------------
% 2 a--------------------------------------------------------------------------
%------------------------------------------------------------------------------

\subquestion

The Hungarian algorithm is shown in \rtab{2008M:q2:tab:HungAlgo}.

\begin{table}[!hbt]
	\begin{minipage}{0.3\textwidth}
		\centering
		\begin{tabular}{cccc}
			48 & 46 & 50 & 44  \\
			49 & 45 & 46 & 49  \\
			47 & 46 & 48 & 44  \\
			51 & 48 & 47 & 45  \\
		\end{tabular}
		\captionsetup{width=1.1\linewidth}
		\caption*{Matrix From question}
	\end{minipage}
	%-------------------------------------------------------------------------------
	\hspace{20pt}
	\begin{minipage}{0.3\textwidth}
		\centering
		\begin{tabular}{cccc}
			4 & 2 & 6 & 0  \\
			4 & 0 & 1 & 4  \\
			3 & 2 & 4 & 0  \\
			6 & 3 & 2 & 0  \\
		\end{tabular}
		\captionsetup{width=1.1\linewidth}
		\caption*{Matrix after Reducing Rows}
	\end{minipage}
	%-------------------------------------------------------------------------------
	\hspace{20pt}
	\begin{minipage}{0.3\textwidth}
		\centering
		\begin{tabular}{cccc}
			1 & 2 & 5 & 0  \\
			1 & 0 & 0 & 4  \\
			0 & 2 & 3 & 0  \\
			3 & 3 & 1 & 0  \\
		\end{tabular}
		\captionsetup{width=1.1\linewidth}
		\caption*{Matrix after Reducing Columns} 
	\end{minipage}
	%-------------------------------------------------------------------------------
	\vspace{20pt} 
	\begin{minipage}{0.3\textwidth}
		\centering
		\begin{tabular} {cccccc}
			&   &   &   & \hspace{-3.25mm} \hvs{v1} 	   &     			  \\
            & 1 & 2 & 5 & 0								   &   			      \\
   \hhs{h1} & 1 & 0 & 0 & 4 						 	   &    \hhe[blue]{h1}\\
   \hhs{h2}	& 0 & 2 & 3 & 0								   &    \hhe[blue]{h2}\\
            & 3 & 3 & 1 & 0							       &                  \\
			&   &   &   & \hspace{-3.25mm} \hve[blue]{v1}  &                  \\

		\end{tabular}
		\captionsetup{width=1.1\linewidth}
		\caption*{Shading 0's using the least \\ \centering number of lines}
	\end{minipage}
	%-------------------------------------------------------------------------------
	\hspace{20pt}
	\begin{minipage}{0.3\textwidth}
		\centering
		\begin{tabular}{cccc}
			  &   &   &      \\
			0 & 1 & 4 & 0   \\
			1 & 0 & 0 & 6   \\
			0 & 2 & 3 & 2   \\
			2 & 2 & 0 & 0   \\
			  &   &   &    \\
		\end{tabular}
		\captionsetup{width=1.1\linewidth}
		\caption*{Applying Step ~\ref{mod1:defn:HungAlgStep4} \\ \hspace{0pt}} 
	\end{minipage}
	%-------------------------------------------------------------------------------
	\hspace{20pt}
	\begin{minipage}{0.3\textwidth}
		\centering
		\begin{tabular}{cccccc}
						&     &   &   &   &    			     \\
			\hhs{h1}	&	0 & 1 & 4 & 0 &    \hhe[red]{h1} \\
			\hhs{h2}	&	1 & 0 & 0 & 6 &    \hhe[red]{h2} \\
			\hhs{h3}	&	0 & 2 & 3 & 2 &    \hhe[red]{h3} \\
			\hhs{h4}	&	2 & 2 & 0 & 0 &    \hhe[red]{h4} \\
						&	  &   &   &   &     			 \\
		
		\end{tabular}
		\captionsetup{width=1.1\linewidth}
		\caption*{Shading 0's using the least \\ \centering number of lines}
	\end{minipage}
	
	\caption{\label{2008M:q2:tab:HungAlgo} Showing the steps of the Hungarian Algorithm.}
\end{table}

From \rtab{2008M:q2:tab:HungAlgo}, the possible pairings of the runners and the parts of the race are as follows,

\begin{align}
	W & \rightarrow 1 ~\text{or} ~4 \,, \nn \\
	X & \rightarrow 2 ~\text{or} ~3 \,, \nn \\
	Y & \rightarrow 1 \,, \nn \\
	Z & \rightarrow 3 ~\text{or} ~4 \,.
\end{align}

Thus, the optimal pairing to minimize the total race time is,

\begin{align}
	W \rightarrow 4 \,, \nn \\
	X \rightarrow 2 \,, \nn \\
	Y \rightarrow 1 \,, \nn \\
	Z \rightarrow 3 \,.
\end{align}

%------------------------------------------------------------------------------
% 2 b--------------------------------------------------------------------------
%------------------------------------------------------------------------------

\subquestion

see \rfig{2008M:q2:fig:ActNet1}.

\begin{figure}
	\begin{center}
		\includegraphics{../2007/figures/2008M-ActNet1}
		\caption{\label{2008M:q2:fig:ActNet1} Activity Network of the project.}
	\end{center}
\end{figure}

%------------------------------------------------------------------------------
% 2 c--------------------------------------------------------------------------
%------------------------------------------------------------------------------

\subquestion 

\begin{subsubquestions}
	
\subsubquestion

See \rtab{2008M:q2:tab:ActNet2}

\begin{table}[ht]
	\centering
	\begin{tabular}{|c|c|c|c|}
		\hline
		Activity & Earliest Start Time & Latest Start Time & Float Time \\
		\hline
		A & 0 & 0 & 0 \\
		B & 6 & 7 & 1 \\
		C & 6 & 6 & 0 \\
		D & 15 & 15 & 0 \\
		E & 21 & 22 & 1 \\
		F & 21 & 21 & 0 \\
		G & 34 & 34 & 0 \\
		\hline
	\end{tabular}
	\caption{\label{2008M:q2:tab:ActNet2} Float times of the activities.}
\end{table} 

%-----------------------------------------------------------------------------

\subsubquestion

\begin{subsubsubquestions}

\subsubsubquestion

From \rtab{2008M:q2:tab:ActNet2} and \rdef{mod1:defn:CritPath}, we can see that the critical path is,

\begin{equation}
	\text{Start} \rightarrow A \rightarrow C \rightarrow D \rightarrow F \rightarrow G \rightarrow \text{Finish} \,.
\end{equation}	

%----------------------------------------------------------------------------

\subsubsubquestion

From \rtab{2008M:q2:tab:ActNet2}, the latest finish time of $E$ is $34$ hours.

\end{subsubsubquestions}

\end{subsubquestions}

\end{subquestions}

