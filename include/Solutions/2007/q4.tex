%------------------------------------------------------------------------------
% Author(s):
% Varaun Ramgoolie
% Copyright:
%  Copyright (C) 2020 Brad Bachu, Arjun Mohammed, Varaun Ramgoolie, Nicholas Sammy
%
%  This file is part of Applied-Mathematics-Unit2 and is distributed under the
%  terms of the MIT License. See the LICENSE file for details.
%
%  Description:
%     Year: 2008 May
%     Module: 2
%     Question: 4
%------------------------------------------------------------------------------

%------------------------------------------------------------------------------
% 4 a
%------------------------------------------------------------------------------

\begin{subquestions}
	
\subquestion

The probability density function, $f(x)$, of a continuous random variable $X$ is given.

We can use the property that probabilities must sum to one, \rprop{mod2:prop:ContinuousRV:1} for continuous random variables,
\begin{align}
	\int_{-\infty}^{\infty} f(x)\,\dd x  & = 1 \,.
\end{align}

Since $f(x)$ is only defined for $1 \leq x \leq 5$, the integral reduces to
\begin{align}
	\int_{-\infty}^{\infty} f(x)\,\dd x &= \int_{-\infty}^{1} f(x)\,\dd x+\int_{1}^{5} f(x)\,\dd x+\int_{5}^{\infty} f(x)\,\dd x \nn \\
	&= 0 + \int_{1}^{5} (ax)\,\dd x + 0 \,.
\end{align}

Evaluating and equating to $1$, we can solve for $a$,
\begin{align}
	a \times \left[\frac{x^2}{2}\right]^5_1 & = 1 \nn \\
	a \times \left(\frac{5^2}{2}- \frac{1^2}{2}\right) & = 1 \nn \\
	a \times \frac{25-1}{2} & = 1 \nn \\
	12a & = 1 \nn \\
	\implies a & = \frac{1}{12} \,.
\end{align}
	
%------------------------------------------------------------------------------
% 4 b
%------------------------------------------------------------------------------

\subquestion


From Note ~\ref{mod2:note:ContinuousRV:Note1}, we know that we must integrate the p.d.f over the appropriate region,
\begin{align}
	P(X<3) & = P(-\infty<X<3) \nn \\
	       & = \int_{-\infty}^{3} f(x)\, \dd x \,.
\end{align}

Again, splitting up the integral in the appropriate intervals,
\begin{align}
	P(X<3) & = \int_{-\infty}^{1} f(x)\,\dd x + \int_{1}^{3} f(x)\, \dd x \,,
\end{align}
allows us to substitute and evaluate as,
\begin{align}
	P(X<3) & = 0 + \int_{1}^{3}\frac{x}{12} \, \dd x \nn \\
	       & = \frac{1}{12} \times \left[\frac{x^2}{2}\right]^3_1 \nn \\
	       & = \frac{1}{12} \times \left(\frac{3^2}{2} - \frac{1^2}{2}\right) \nn \\
	       & = \frac{1}{12} \times \left(\frac{9}{2} - \frac{1}{2}\right) \nn \\
	       & = \frac{1}{12} \times \frac{8}{2} \nn \\
	       & = \frac{1}{3} \,. 
\end{align}
	
%------------------------------------------------------------------------------
% 4 c
%------------------------------------------------------------------------------

\subquestion

Using \rdef{mod2:defn:ContinuousRV:Expectation}, we know that $E(X)$ is given by,
\begin{align}
	E(X) & = \int_{-\infty}^{\infty} xf(x)\, \dd x \,.
\end{align}

Splitting up the integral over the appropriate intervals,
\begin{align}
	E(X) & = \int_{-\infty}^{1} xf(x)\,\dd x + \int_{1}^{5} xf(x)\,\dd x + \int_{5}^{\infty} xf(x)\,\dd x \,,
\end{align}
allows us to substitute and evaluate,
\begin{align}
	E(X)& = 0 + \int_{1}^{5} x\left( \frac{x}{12}\right)\,\dd x + 0\nn \\
		 & = \frac{1}{12} \times \int_{1}^{5} x^2\,\dd x \nn \\ 
		 & = \frac{1}{12} \times \left[\frac{x^3}{3}\right]^5_1 \nn \\
		 & = \frac{1}{12} \times \left(\frac{5^3}{3}-\frac{1^3}{3}\right) \nn \\
		 & = \frac{1}{12} \times \frac{124}{3} \nn \\
		 & = \frac{124}{36} \,.
\end{align}

To compute $\var(X)$, we use \rdef{mod2:defn:ContinuousRV:Variance},
\begin{equation}
	\var(X) = E(X^2) - E(X)^2 \,.\label{2008M:q4:CRVar}
\end{equation}

Using \req{mod2:eq:ContinuousRV:Variance}, we can calculate $E(X^2)$,
\begin{align}
	E(X^2) & = \int_{-\infty}^{\infty} x^2f(x)\,\dd x \,.
\end{align}

As before, we split up the integral,
\begin{align}
	E(X^2)& = \int_{-\infty}^{1} x^2f(x)\,\dd x+\int_{1}^{5} x^2f(x)\,\dd x+\int_{5}^{\infty} x^2f(x)\,\dd x \,,
\end{align}
substitute, and evaluate,
\begin{align}
	E(X^2) & = 0 + \int_{1}^{5} x^2\left( \frac{x}{12}\right)\mathrm{d}x + 0 \nn \\
	& = \frac{1}{12} \times \int_{1}^{5} x^3\mathrm{d}x \nn \\ 
	& = \frac{1}{12} \times \left[\frac{x^4}{4}\right]^5_1 \nn \\
	& = \frac{1}{12} \times \left(\frac{5^4}{4}-\frac{1^4}{4}\right) \nn \\
	& = \frac{1}{12} \times \frac{624}{4} \nn \\
	& = \frac{624}{48} \,. 
\end{align}

Then, substituting these values into \req{2008M:q4:CRVar}, we find,
\begin{align}
	\var(X) & = E(X^2) - E(X)^2 \nn \\
	& = \frac{624}{48} - \left(\frac{124}{36}\right)^2 \nn \\
	& = \frac{624}{48}-\frac{961}{81} \nn \\
	& = \frac{92}{81} \,.
\end{align}

%------------------------------------------------------------------------------
% 4 d
%------------------------------------------------------------------------------

\subquestion

From \rdef{mod2:defn:ContinuousRV:CDF}, the cumulative distribution function is given by,
\begin{align}
	F(x) & = \int_{-\infty}^{x} f(t)\, \dd t\,.
\end{align}

We can consider this for three possible values of $x$.\footnote{Technically the domain was already restricted to $1\leq x \leq 5$, so there is only one region. We decided to stick with $x\in \mathrm{R}$ for consistency. But the answer does not change with the way we have presented it.} 

For $x < 1$,
\begin{align}
	F(x) &= \int_{-\infty}^x f(t)\,\dd t \nn \\
			&= 0 \,.
\end{align}

For $1\leq x < 5$,
\begin{align}
	F(x) &= \int_{-\infty}^x f(t)\,\dd t \nn \\
		  &= \int_{-\infty}^1 f(t)\,\dd t + \int_{1}^x f(t)\,\dd t \nn \\
		  &= 0 + \int_{1}^{x} \left(\frac{t}{12} \right)\,\dd t \nn \\
	     & = \frac{1}{12} \times \left[ \frac{t^2}{2} \right]^a_1 \nn \\
	     & = \frac{1}{12} \times \left(\frac{x^2}{2} - \frac{1}{2}\right) \nn \\
	     & = \frac{x^2}{24} - \frac{1}{24} \,.
\end{align}

For $x>5$,
\begin{align}
	F(x) &= \int_{-\infty}^x f(t)\,\dd t \nn \\
		  &= \int_{-\infty}^1 f(t)\,\dd t + \int_{1}^5 f(t)\,\dd t+ \int_{5}^x f(t)\,\dd t \nn \\
		  &= 0 + 1 + 0 \,.
\end{align}

Therefore, the cumulative distribution function is given as,
\begin{equation}
F(x) = \begin{cases}
	 0, & x < 1 \\
	\frac{x^2}{24}-\frac{1}{24}, & 1\leq x < 5 \\
	1, & x\geq 5 \\ 
\end{cases}
\end{equation}

%------------------------------------------------------------------------------
% 4 e
%------------------------------------------------------------------------------

\subquestion

Recall from \rprop{mod2:prop:ContinuousRV:CDF}, that the median $M$, is the value of $x = M$ such that
\begin{align}
	F(M) & = \frac{1}{2} \,.
\end{align}

Since $0 \leq F(M) \leq 1$, by inspecting our definition of $F$ above, we know $1 \leq M < 5$. Thus, we can equate and solve as follows,
\begin{align}
	\frac{M^2}{24}-\frac{1}{24} & = \frac{1}{2} \nn \\
	M^2 - 1 & = 12 \nn \\
	M^2 & = 13 \nn \\
	\implies M & = \sqrt{13} \,.
\end{align}

%------------------------------------------------------------------------------
% 4 f
%------------------------------------------------------------------------------

\subquestion

From Section ~\ref{mod2:section:ExpectationVariance}, we know that,
\begin{align}
	E(3X+2) & = 3\, E(X) +2\,.
\end{align}

Substituting for $E(X)$, we have
\begin{align}
	E(3X+2)  & = 3 \times \frac{124}{36} + 2 \nn \\
	         & = \frac{37}{3} \,. 
\end{align}

Similarly, we also know that
\begin{align}
	\var(3X+2) & = 3^2 \, \var(X) \,.
\end{align}

Substituting for $\var(X)$, we have
\begin{align}
	\var(3X+2) & = 9  \times \frac{92}{81} \nn \\
	           & = \frac{92}{9} \,.
\end{align}

\end{subquestions}
