%------------------------------------------------------------------------------
% Author(s):
% Varaun Ramgoolie
% Copyright:
%  Copyright (C) 2020 Brad Bachu, Arjun Mohammed, Varaun Ramgoolie, Nicholas Sammy
%
%  This file is part of Applied-Mathematics-Unit2 and is distributed under the
%  terms of the MIT License. See the LICENSE file for details.
%
%  Description:
%     Year: 2008 May
%     Module: 2
%     Question: 4
%------------------------------------------------------------------------------

%------------------------------------------------------------------------------
% 4 a
%------------------------------------------------------------------------------

\begin{subquestions}
	
\subquestion

\TODO{Change 2007 to 2008 May}

The probability density function, $f(x)$, of a continuous random variable $X$ is given.
From \rprop{mod2:prop:ContinuousRV:1}, we can find that,
\begin{align}
	\int_{-\infty}^{\infty} f(x)\mathrm{d}x  & = 1 \nn \\
	\int_{-\infty}^{1} f(x)\mathrm{d}x+\int_{1}^{5} f(x)\mathrm{d}x+\int_{5}^{\infty} f(x)\mathrm{d}x & = 1 \nn \\
	0 + \int_{1}^{5} (ax)\mathrm{d}x + 0 & = 1 \nn \\
	a \times \left[\frac{x^2}{2}\right]^5_1 & = 1 \nn \\
	a \times \left(\frac{5^2}{2}- \frac{1^2}{2}\right) & = 1 \nn \\
	a \times \frac{25-1}{2} & = 1 \nn \\
	12a & = 1 \nn \\
	\implies a & = \frac{1}{12} \,.
\end{align}
	
%------------------------------------------------------------------------------
% 4 b
%------------------------------------------------------------------------------

\subquestion

Using Note ~\ref{mod2:note:ContinuousRV:Note1}, we get that,
\begin{align}
	P(X<3) & = P(-\infty<X<3) \nn \\
	       & = \int_{-\infty}^{3} f(x)\mathrm{d}x \nn \\
	       & = \int_{-\infty}^{1} f(x)\mathrm{d}x + \int_{1}^{3} f(x)\mathrm{d}x \nn \\
	       & = 0 + \int_{1}^{3}\frac{x}{12} \mathrm{d}x \nn \\
	       & = \frac{1}{12} \times \left[\frac{x^2}{2}\right]^3_1 \nn \\
	       & = \frac{1}{12} \times \left(\frac{3^2}{2} - \frac{1^2}{2}\right) \nn \\
	       & = \frac{1}{12} \times \left(\frac{9}{2} - \frac{1}{2}\right) \nn \\
	       & = \frac{1}{12} \times \frac{8}{2} \nn \\
	       & = \frac{1}{3} \,. 
\end{align}
	
%------------------------------------------------------------------------------
% 4 c
%------------------------------------------------------------------------------

\subquestion

Using \rdef{mod2:defn:ContinuousRV:Expectation}, we are able to find that,
\begin{align}
	E(X) & = \int_{-\infty}^{\infty} xf(x)\mathrm{d}x \nn \\
		 & = \int_{-\infty}^{1} xf(x)\mathrm{d}x + \int_{1}^{5} xf(x)\mathrm{d}x + \int_{5}^{\infty} xf(x)\mathrm{d}x \nn \\
		 & = 0 + \int_{1}^{5} x\left( \frac{x}{12}\right)\mathrm{d}x + 0\nn \\
		 & = \frac{1}{12} \times \int_{1}^{5} x^2\mathrm{d}x \nn \\ 
		 & = \frac{1}{12} \times \left[\frac{x^3}{3}\right]^5_1 \nn \\
		 & = \frac{1}{12} \times \left(\frac{5^3}{3}-\frac{1^3}{3}\right) \nn \\
		 & = \frac{1}{12} \times \frac{124}{3} \nn \\
		 & = \frac{124}{36} \,.
\end{align}

From \rdef{mod2:defn:ContinuousRV:Variance}, we get that,
\begin{equation}
	\var(X) = E(X^2) - E(X)^2 \,.\label{2008M:q4:CRVar}
\end{equation}

Using \req{mod2:eq:ContinuousRV:Variance}, we can calculate,
\begin{align}
	E(X^2) & = \int_{-\infty}^{\infty} x^2f(x)\mathrm{d}x \nn \\
	 & = \int_{-\infty}^{1} x^2f(x)\mathrm{d}x+\int_{1}^{5} x^2f(x)\mathrm{d}x+\int_{5}^{\infty} x^2f(x)\mathrm{d}x \nn \\
	& = 0 + \int_{1}^{5} x^2\left( \frac{x}{12}\right)\mathrm{d}x + 0 \nn \\
	& = \frac{1}{12} \times \int_{1}^{5} x^3\mathrm{d}x \nn \\ 
	& = \frac{1}{12} \times \left[\frac{x^4}{4}\right]^5_1 \nn \\
	& = \frac{1}{12} \times \left(\frac{5^4}{4}-\frac{1^4}{4}\right) \nn \\
	& = \frac{1}{12} \times \frac{624}{4} \nn \\
	& = \frac{624}{48} \,. 
\end{align}

Thus, substituting these values into \req{2008M:q4:CRVar}, we can find that,
\begin{align}
	\var(X) & = E(X^2) - E(X)^2 \nn \\
	& = \frac{624}{48} - \left(\frac{124}{36}\right)^2 \nn \\
	& = \frac{624}{48}-\frac{961}{81} \nn \\
	& = \frac{92}{81} \,.
\end{align}

%------------------------------------------------------------------------------
% 4 d
%------------------------------------------------------------------------------

\subquestion

From \rdef{mod2:defn:ContinuousRV:CDF}, we can get that,
\begin{align}
	F(a) & = \int_{-\infty}^{a} f(x)\mathrm{d}x \nn \\
		 & = \int_{-\infty}^{1} f(x)\mathrm{d}x+\int_{1}^{a} f(x)\mathrm{d}x \nn \\
	     & = 0 + \int_{1}^{a} \left(\frac{x}{12} \right)\mathrm{d}x \nn \\
	     & = \frac{1}{12} \times \left[ \frac{x^2}{2} \right]^a_1 \nn \\
	     & = \frac{1}{12} \times \left(\frac{a^2}{2} - \frac{1}{2}\right) \nn \\
	     & = \frac{a^2}{24} - \frac{1}{24} \,.
\end{align}

Therefore, we can define,
\[  F(x) = \begin{cases}
	 0, & x \leq 1 \\
	\frac{x^2}{24}-\frac{1}{24}, & 1\leq x\leq 5 \\
	1, & x\geq 5 \\ 
\end{cases}
\]

%------------------------------------------------------------------------------
% 4 e
%------------------------------------------------------------------------------

\subquestion

Using \rprop{mod2:prop:ContinuousRV:CDF}, we see that,
\begin{align}
	F(M) & = \frac{1}{2} \nn \\
	\frac{M^2}{24}-\frac{1}{24} & = \frac{1}{2} \nn \\
	\text{(Multiplying by 24)} \implies M^2 - 1 & = 12 \nn \\
	M^2 & = 13 \nn \\
	\implies M & = \sqrt{13} \,.
\end{align}

%------------------------------------------------------------------------------
% 4 f
%------------------------------------------------------------------------------

\subquestion

Using Section ~\ref{mod2:section:ExpectationVariance}, we can find that,
\begin{align}
	E(3X+2)	& = 3\times E(X) +2 \nn \\
	        & = 3 \times \frac{124}{36} + 2 \nn \\
	        & = \frac{37}{3} \,. \\ \nn \\
	 \text{Var}(3X+2) & = 3^2 \times \text{Var}(X) \nn \\
	                  & = 9  \times \frac{92}{81} \nn \\
	                  & = \frac{92}{9} \,.
\end{align}


\end{subquestions}
