%------------------------------------------------------------------------------
% Author(s):
% Varaun Ramgoolie
% Copyright:
%  Copyright (C) 2020 Brad Bachu, Arjun Mohammed, Nicholas Sammy, Kerry Singh
%
%  This file is part of Applied-Mathematics-Unit2 and is distributed under the
%  terms of the MIT License. See the LICENSE file for details.
%
%  Description:
%     Year: 2013
%     Module: 1
%     Question: 1 
%------------------------------------------------------------------------------

\begin{subquestions}

%----------------------------------------------------------------------------
% 1 a 
%----------------------------------------------------------------------------

\subquestion

See \rtab{2013:q1:tab:TruthTab1}.

\begin{table}[ht]
	\centering
	\begin{tabular}{|c|c|c|c|c|c|}
		\hline
		p & q & p $\land$ q & p $\lor$ q & $\sim$ (p $\lor$ q) & (p $\land$ q) $\land$ $\sim$ (p $\lor$ q) \\
		\hline
		0 & 0 & 0 & 0 & 1 & 0 \\
		0 & 1 & 0 & 1 & 0 & 0 \\
		1 & 0 & 0 & 1 & 0 & 0 \\
		1 & 1 & 1 & 1 & 0 & 0 \\
		\hline
	\end{tabular}
	\caption{\label{2013:q1:tab:TruthTab1} Truth Table of $(p \land q) \land \sim (p \lor q)$\,.}
\end{table}

%------------------------------------------------------------------------------
% 1 b
%------------------------------------------------------------------------------

\subquestion

From \rdef{mod1:defn:Contradiction}, we see that $(p \land q)\, \land \sim (p \lor q)$ is a contradiction.

%------------------------------------------------------------------------------
% 1 c
%------------------------------------------------------------------------------

\subquestion

Using \rdef{mod1:law:DeMorgan}, we can see that,

\begin{equation}
	\sim (p \lor q) \equiv \, \sim p \, \land \sim q \,.
\end{equation}

%------------------------------------------------------------------------------
% 1 d
%------------------------------------------------------------------------------

\subquestion

Let $p$="It is sunny" and let $q$="It is hot". \\ 
Therefore, "It is not true that it is hot and sunny" $\equiv$ $\sim (p \land q) \,.$ \\
Using \rdef{mod1:law:DeMorgan},

\begin{equation}
	\sim (p \land q) \equiv \, \sim p \, \lor \sim q \,.
\end{equation}

Thus, the equivalent statement of 	$(\sim p \, \lor \sim q)$ is "It is not hot or it is not sunny."

%------------------------------------------------------------------------------
% 1 e
%------------------------------------------------------------------------------

\subquestion

\begin{subsubquestions}
	 
\subsubquestion

The switching circuit is shown.

\begin{center}

\begin{circuitikz}
		
	\draw (0,0) to[normal open switch, *-*](3,0);
	\path (0,0) -- (3,0) node[pos=0.5,below]{p};
	
	\draw (3,0) to[normal open switch, *-*](6,0);
	\path (3,0) -- (6,0) node[pos=0.5,below]{q};
	
	\draw [color=black, thin] (6,0) -- (6,1);
	\draw [color=black, thin] (6,0) -- (6,-1);
	
	\draw (6,1) to[normal open switch, *-*](9,1);
	\path (6,1) -- (9,1) node[pos=0.5, below]{p};
	
	\draw (6,-1) to[normal open switch, *-*](9,-1);
	\path (6,-1) -- (9,-1) node[pos=0.5, below]{r};
	
	\draw [color=black, thin] (6,1) -- (6,0);
    \draw [color=black, thin] (6,-1) -- (6,0);	
    
  	\draw [color=black, thin] (9,0) -- (9,1);
    \draw [color=black, thin] (9,0) -- (9,-1);

	\draw [color=black, thin] (9,0) -- (11,0);
	\draw [color=black, thin] (11,0) -- (11,1);
	\draw [color=black, thin] (11,0) -- (11,-1);
		
	\draw (11,0) to[normal open switch, *-*](14,0);
	\draw (11,1) to[normal open switch, *-*](14,1);
	\draw (11,-1) to[normal open switch, *-*](12.5,-1);
	\draw (12.5,-1) to[normal open switch, *-*](14,-1);
	
	\path (11,0) -- (14,0) node[pos=0.5,below]{s};
	\path (11,1) -- (14,1) node[pos=0.5,below]{q};
	\path (11,-1) -- (12.5,-1) node[pos=0.5,below]{r};
	\path (12.5,-1) -- (14,-1) node[pos=0.5,below]{p};
	
	\draw [color=black, thin] (14,1) -- (14,0);
	\draw [color=black, thin] (14,-1) -- (14,0);
	
	\draw [color=black, thin] (14,0) -- (15,0);	
\end{circuitikz}
	
\end{center}

%-----------------------------------------------------------------------------

\subsubquestion

Using the laws from ~\ref{mod1:section:BooleanAlgebraLaws}, we can simplify the expression as follows,

\begin{align}
	p \land q \land (p \lor r) \land (q \lor (r \land p) \lor s)
	& \equiv 	p \land \underline{q \land (p \lor r)} \land (q \lor (r \land p) \lor s)\,, \nn \\
	...\text{using ~\ref{mod1:law:Commutative}}
	& \equiv 	\underline{p \land (p \lor r)} \land q \land (q \lor (r \land p) \lor s)\,, \nn \\
	...\text{using ~\ref{mod1:law:Absorptive}}
	& \equiv 	p \land q \land (\underline{q \lor (r \land p) \lor s})\,, \nn \\
	...\text{using ~\ref{mod1:law:Associative}}
	& \equiv 	p \land \underline{q \land (q \lor ((r \land p) \lor s))}\,, \nn \\
	...\text{using ~\ref{mod1:law:Absorptive}}
	& \equiv 	p \land q \,. 
\end{align}

\end{subsubquestions}

%------------------------------------------------------------------------------
% 1 f
%------------------------------------------------------------------------------

\subquestion

See \rfig{2013:q1:fig:LogicGates}.

\begin{figure}
	\begin{center}
		\includegraphics{../2013/figures/2013q1LogicGate}
		\caption{\label{2013:q1:fig:LogicGates} Showing the logic gate equivalent of $\sim((p \land q) \lor r)$\,.}
	\end{center}
\end{figure}

\end{subquestions}

