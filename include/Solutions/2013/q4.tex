%------------------------------------------------------------------------------
% Author(s):
% Varaun Ramgoolie
% Copyright:
%  Copyright (C) 2020 Brad Bachu, Arjun Mohammed, Varaun Ramgoolie, Nicholas Sammy
%
%  This file is part of Applied-Mathematics-Unit2 and is distributed under the
%  terms of the MIT License. See the LICENSE file for details.
%
%  Description:
%     Year: 2013
%     Module: 2
%     Question: 4 
%------------------------------------------------------------------------------

\begin{subquestions}
	
%------------------------------------------------------------------------------
% 4 a
%------------------------------------------------------------------------------

\subquestion

It is given that accidents occur on a certain highway at an average rate of 1 per week. This can be modeled using a Poisson distribution.

Let $X$ be the discrete random variable representing the number of accidents which occur on that highway per week. We can express this as $X \sim $ Pois$(1)$.

\begin{subsubquestions}
	
\subsubquestion

We want to find $P(X=2)$. Using \req{mod2:eq:PoissonDist}, we know that,
\begin{equation}
	P(X=k)= \frac{1^k \times e^{-1}}{k!} \,.
\end{equation}
	
Therefore, we can find $P(X=2)$ as follows,
\begin{align}
	P(X=2) & = \frac{1^2 \times e^{-1}}{2!} \nn \\
	       & = \frac{1 \times e^{-1}}{2} \nn \\
	       & = \frac{1}{2 \times e} \nn \\
	       & = 0.184 \,.
\end{align}
	
%------------------------------------------------------------------------------

\subsubquestion

We are now given that an average of 4 accidents occur in a 4-week period. 

Let $Y$ be the discrete random variable representing the number of accidents which occur on that highway during a 4-week period. We can express this as $Y \sim $ Pois$(4)$.

We want to show that $P(Y \geq 4) = 0.567$. Using \req{mod2:eq:PoissonDist}, we know that,
\begin{equation}
	P(Y=k)= \frac{4^k \times e^{-4}}{k!} \,. \label{2013:q4:eq:Pois1}
\end{equation}

We can notice that,
\begin{align}
	P(Y \geq 4) & = 1 - P(Y \leq 3) \nn \\
	            & = 1 - \left[P(Y=0)+P(Y=1)+P(Y=2)+P(Y=3) \right] \,.
\end{align}

Using \req{2013:q4:eq:Pois1}, we can calculate $P(Y \leq 3)$ as follows,
\begin{align}
	P(Y=0) & = \frac{4^0 \times e^{-4}}{0!} 
	         = \frac{1 \times e^{-4}}{1} \nn \\
	       & = \frac{1}{e^4} \,. \\	\nn \\
	P(Y=1) & = \frac{4^1 \times e^{-4}}{1!} 
	         = \frac{4 \times e^{-4}}{1} \nn \\
	       & = \frac{4}{e^4} \,. \\	\nn \\       
	P(Y=2) & = \frac{4^2 \times e^{-4}}{2!} 
		     = \frac{16 \times e^{-4}}{2} \nn \\
           & = \frac{8}{e^4} \,. \\ \nn \\            
	P(Y=3) & = \frac{4^3 \times e^{-4}}{3!} 
             = \frac{64 \times e^{-4}}{6} \nn \\
           & = \frac{64}{6 \times e^4} \,. \\ \nn \\         
    P(Y \leq 3) & = P(Y=0)+P(Y=1)+P(Y=2)+P(Y=3) \nn \\
                & = \frac{1}{e^4} + \frac{4}{e^4} + \frac{8}{e^4} + \frac{64}{6 \times e^4} \nn \\
                & = \frac{1}{e^4} \times \left(1+4+8+\frac{64}{6} \right) \nn \\
                & = \frac{1}{e^4} \times \left(\frac{142}{6} \right) \nn \\
                & = \frac{142}{6 \times e^4} \nn \\
                & = 0.433 \,.
\end{align}

Thus,
\begin{align}
	P(Y \geq 4) & = 1 - P(Y \leq 3) \nn \\
	            & = 1 - 0.433 \nn \\
	            & = 0.567 \,.
\end{align}

%------------------------------------------------------------------------------

\subsubquestion


\end{subsubquestions}





\end{subquestions}

