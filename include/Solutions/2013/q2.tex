%------------------------------------------------------------------------------
% Author(s):
% Varaun Ramgoolie
% Copyright:
%  Copyright (C) 2020 Brad Bachu, Arjun Mohammed, Varaun Ramgoolie, Nicholas Sammy
%
%  This file is part of Applied-Mathematics-Unit2 and is distributed under the
%  terms of the MIT License. See the LICENSE file for details.
%
%  Description:
%     Year: 2013
%     Module: 1
%     Question: 2
%------------------------------------------------------------------------------

\begin{subquestions}

%---------------------------------------------------------
% 2 a 
%---------------------------------------------------------

\subquestion

We can modify the given activity network to obtain the desired information.
\begin{figure}[H]
	\begin{center}
		\includegraphics[scale=0.75]{../2013/figures/2013q2CriticalPath}
		\caption{\label{2013:q2:fig:ActNet} Activity network of the project.}
	\end{center}
\end{figure}



%----------------------------------------------------------
% Lots of subsubquestions
%----------------------------------------------------------

\begin{subsubquestions}

\subsubquestion 

The earliest start time of $S$ is 4 days. The earliest start time for $X$ is 7 days.

%-----------------------------------------------------------------------------

\subsubquestion

The minimum completion time of the project if 13 days.

%-----------------------------------------------------------------------------

\subsubquestion

The latest start time of $Q$ is 0 days. The latest start time of $X$ is 7 days.

%-----------------------------------------------------------------------------

\subsubquestion

From \rdef{mod1:defn:CritPath}, there are three critical paths for the project. They are,
\begin{align}
	\text{Start} \rightarrow Q \rightarrow S \rightarrow X \rightarrow \text{End}\,, \nn \\
	\text{Start} \rightarrow Q \rightarrow T \rightarrow X \rightarrow \text{End}\,, \nn \\
	\text{Start} \rightarrow R \rightarrow T \rightarrow X \rightarrow \text{End}\,.
\end{align}

%-----------------------------------------------------------------------------

\subsubquestion

The float time of activity $T$ is 0 days, using \rdef{mod1:defn:FloatTime}.

\end{subsubquestions}

%--------------------------------------------------------
% 2 b 
%--------------------------------------------------------

\subquestion

From \rdef{mod1:defn:Degree}, we see that the degree of $A$ is 4 and the degree of $C$ is 2.

%--------------------------------------------------------
% 2 c 
%--------------------------------------------------------

\subquestion

\begin{subsubquestions}

\subsubquestion 

We can determine the supermarkets to which each warehouse must be assigned in order to minimise the cost of delivery by applying the Hungarian algorithm as follows.
\begin{table}[H]
	\begin{minipage}{0.3\textwidth}
		\centering
		\begin{tabular}{cccc}
			6 & 5 & 7 & 9 \\
			2 & 6 & 5 & 8 \\
			10 & 5 & 1 & 9 \\
			11 & 6 & 3 & 8 \\
		\end{tabular}
		\captionsetup{width=1.1\linewidth}
		\caption*{Matrix From question}
	\end{minipage}
	\hspace{20pt}
	%-----------------------------------------------------------------------------
	\begin{minipage}{0.3\textwidth}
		\centering
		\begin{tabular}{cccc}
			1 & 0 & 2 & 4 \\
			0 & 4 & 3 & 6 \\
			9 & 4 & 0 & 8 \\
			8 & 3 & 0 & 5 \\
		\end{tabular}
		\captionsetup{width=1.1\linewidth}
		\caption*{Reducing Rows}
	\end{minipage}
	\hspace{20pt}
	%-----------------------------------------------------------------------------
	\begin{minipage}{0.3\textwidth}
		\centering
		\begin{tabular}{cccc}
			1 & 0 & 2 & 0 \\
			0 & 4 & 3 & 2 \\
			9 & 4 & 0 & 4 \\
			8 & 3 & 0 & 1 \\
		\end{tabular}
		\captionsetup{width=1.1\linewidth}
		\caption*{Reducing Columns} 
	\end{minipage}
	
	\vspace{20pt} 
	%-----------------------------------------------------------------------------
	\begin{minipage}{0.3\textwidth}
		\centering
		\begin{tabular} {cccccc}
			&   &        & \hspace{-3.25mm} \hvs{v1}       &   &                \\ 
   \hhs{h1} & 1 &      0 &                               2 & 0 & \hhe[blue]{h1} \\
   \hhs{h2} & 0 &      4 &                               3 & 2 & \hhe[blue]{h2} \\
			& 9 &      4 &                               0 & 4 &                \\
			& 8 &      3 &                               0 & 1 &                \\
			&   &        & \hspace{-3.25mm} \hve[blue]{v1} &   &                \\
		\end{tabular}
		\captionsetup{width=1.1\linewidth}
		\caption*{Shading 0's}
	\end{minipage}
	\hspace{20pt}
	%-----------------------------------------------------------------------------
	\begin{minipage}{0.3\textwidth}
		\centering
		\begin{tabular}{cccc}
		      &   &   &   \\
	    	1 & 0 & 4 & 0 \\
			0 & 4 & 5 & 2 \\
			8 & 3 & 0 & 3 \\
			7 & 2 & 0 & 0 \\
		      &   &   &   \\	 
		\end{tabular}
		\captionsetup{width=1.1\linewidth}
		\caption*{Applying Step 5, ~\ref{mod1:defn:HungAlgStep4} \\ \hspace{0pt}} % The reference here is a placeholder until that note is put in%
	\end{minipage}
	\hspace{20pt}
	%-----------------------------------------------------------------------------
	\begin{minipage}{0.3\textwidth}
		\centering
		\begin{tabular} {cccccc}
			&   &        & \hspace{-3.25mm} \hvs{v3}      & \hspace{-3.25mm} \hvs{v4}       &               \\ 
   \hhs{h3} & 1 &      0 &                         4      &                          0      & \hhe[red]{h3} \\
   \hhs{h4}	& 0 &      4 &                         5      &                          2      & \hhe[red]{h4} \\ 
            & 8 &      3 &                         0      &                          3      &               \\
			& 7 &      2 &                         0      &                          0      &               \\ 
			&   &        & \hspace{-3.25mm} \hve[red]{v3} & \hspace{-3.25mm} \hve[red]{v4}  &               \\
		\end{tabular}
		\captionsetup{width=1.1\linewidth}
		\caption*{Shading 0's}
	\end{minipage}
		
	\caption{\label{2013:q2:tab:HungAlgo} Steps of the Hungarian Algorithm.}
\end{table}	

From \rtab{2013:q2:tab:HungAlgo}, the possible pairings of warehouses and supermarkets are as follows,

\begin{align}
	& W_1 \rightarrow S_2 \,, S_4 \,, \nn \\
	& W_2 \rightarrow S_1 \,, \nn \\
	& W_3 \rightarrow S_3 \,, \nn \\
	& W_4 \rightarrow S_3 \,, S_4 \,.
\end{align}

Therefore, the optimum pairings to minimize the cost of delivery are,
\begin{align}
	& W_1 \rightarrow S_2\,, \nn \\
	& W_2 \rightarrow S_1\,, \nn \\
	& W_3 \rightarrow S_3\,, \nn \\
	& W_4 \rightarrow S_4\,.
\end{align}

%-----------------------------------------------------------------------------

\subsubquestion

The total cost of transportation for the 4 warehouses is,
\begin{align}
	\text{Total cost} = 5+2+1+8=16 \, \text{dollars} \,.
\end{align}

\end{subsubquestions}

\end{subquestions}