%------------------------------------------------------------------------------
% Author(s):
% Varaun Ramgoolie
% Copyright:
%  Copyright (C) 2020 Brad Bachu, Arjun Mohammed, Varaun Ramgoolie, Nicholas Sammy
%
%  This file is part of Applied-Mathematics-Unit2 and is distributed under the
%  terms of the MIT License. See the LICENSE file for details.
%
%  Description:
%     Year: 2005 C
%     Module: 1
%     Question: 1 
%------------------------------------------------------------------------------

\begin{subquestions}

%------------------------------------------------------------------------------
% 1 a -------------------------------------------------------------------------
%------------------------------------------------------------------------------

\subquestion

\begin{subsubquestions}
	
\subsubquestion

We can construct the following truth table for the statement $\boldsymbol{p \lor q}$.
\begin{table}[ht]
	\centering
	\begin{tabular}{|c|c|c|}
		\hline
		$\boldsymbol{p}$ & $\boldsymbol{q}$ & $\boldsymbol{p \lor q}$ \\
		\hline
		0 & 0 & 0 \\
		0 & 1 & 1 \\
		1 & 0 & 1 \\
		1 & 1 & 1 \\
		\hline
	\end{tabular}
	\caption{\label{2005:q1:tab:TruthTab1} Truth table of $\boldsymbol{p \lor q}$.}
\end{table}

%-----------------------------------------------------------------------------

\subsubquestion

We can complete the truth table in the following way.
\begin{table}[ht]
	\centering
	\begin{tabular}{|c|c|c|c|c|c|}
		\hline
		$\boldsymbol{p}$ & $\boldsymbol{q}$ & $\boldsymbol{\sim p}$ & $\boldsymbol{p \implies q}$ & $\boldsymbol{\sim p \land q}$ & $\boldsymbol{(p \implies q) \iff (\sim p \land q)}$ \\
		\hline
		0 & 0 & 1 & 1 & 0 & 0 \\
		0 & 1 & 1 & 1 & 1 & 1 \\
		1 & 0 & 0 & 0 & 0 & 1 \\
		1 & 1 & 0 & 1 & 0 & 0 \\
		\hline
	\end{tabular}
	\caption{\label{2005:q1:tab:TruthTab2} Truth table of $\boldsymbol{(p \implies q) \iff (\sim p \land q)}$.}
\end{table}

%-----------------------------------------------------------------------------

\subsubquestion

\begin{subsubsubquestions}
	
\subsubsubquestion

Since the truth values of \rtab{2005:q1:tab:TruthTab1} and \rtab{2005:q1:tab:TruthTab2} are not the same, the expressions $\boldsymbol{p \lor q}$ and $\boldsymbol{(p \implies q) \iff (\sim p \land q)}$ are not logically equivalent.

%-----------------------------------------------------------------------------

\subsubsubquestion

From \rdef{mod1:defn:Tautology}, we can see that ($\boldsymbol{(p \implies q) \iff (\sim p ~\land q)}$) is \textbf{not a tautology} because the truth values are not all 1.

\end{subsubsubquestions}

\end{subsubquestions}

%------------------------------------------------------------------------------
% 1 b -------------------------------------------------------------------------
%------------------------------------------------------------------------------

\subquestion

\begin{subsubquestions}
	
\subsubquestion

If $\boldsymbol{x}$ is the proposition "I do Statistical Analysis" and $\boldsymbol{y}$ is the proposition "I do Applied Mathematics", then $\boldsymbol{x ~\land \sim y}$ can be expressed as "I do Statistical Analysis and I do not do Applied Mathematics."

%-----------------------------------------------------------------------------
\subsubquestion

The proposition "I do Statistical Analysis so I will do Applied Mathematics" can be interpreted in logical language as "If I do Statistical Analysis, then I do Applied Mathematics also". This can be expressed as:
\begin{equation}
	\boldsymbol{x \implies y}\,.
\end{equation}

\end{subsubquestions}

%------------------------------------------------------------------------------
% 1 c -------------------------------------------------------------------------
%------------------------------------------------------------------------------

\subquestion

Let $\boldsymbol{A} = \boldsymbol{(l \land m \land n) \lor (l \land m ~\land \sim n)}$. \\
Let $\boldsymbol{B} = \boldsymbol{l \land m}$. \\
$A$ then becomes,
\begin{equation}
	\boldsymbol{A \equiv (B \land n) \lor (B \land \sim n)} \,.
\end{equation}

Using the Distributive Law (\rdef{mod1:law:Distributive}) in reverse, $A$ becomes,
\begin{align}
	\boldsymbol{A} & \boldsymbol{\equiv B \land (n \lor \sim n)} \,.
\end{align}

Using the Complement Law (\rdef{mod1:law:Complement}) and Identity Law (\rdef{mod1:law:Identity}), $A$ is now,
\begin{align}
	\boldsymbol{A} & \boldsymbol{\equiv B \land T} \nn \\
					& \equiv \boldsymbol{B} \,.
\end{align}

Therefore,
\begin{equation}
	\boldsymbol{(l \land m \land n) \lor (l \land m ~\land \sim n) \equiv l \land m} \,.
\end{equation}

%------------------------------------------------------------------------------
% 1 d -------------------------------------------------------------------------
%------------------------------------------------------------------------------

\subquestion

See the switching circuit of $\boldsymbol{((a \land b) \lor (a \land c)) \land (a \lor b)}$.
\begin{center}
	\begin{circuitikz}
		
	\draw [color=black, thick] (0,0) -- (2,0);
	\draw [color=black, thick] (2,0) -- (2,1);
	\draw [color=black, thick] (2,0) -- (2,-1);
	
	\draw (2,1) to[normal open switch, *-*](4,1);	
	\path (2,1) -- (4,1) node[pos=0.5,below]{a};
	
	\draw (4,1) to[normal open switch, *-*](6,1);
	\path (4,1) -- (6,1) node[pos=0.5,below]{b};
		
	\draw (2,-1) to[normal open switch, *-*](4,-1);
	\path (2,-1) -- (4,-1) node[pos=0.5,below]{a};
	
	\draw (4,-1) to[normal open switch, *-*](6,-1);
	\path (4,-1) -- (6,-1) node[pos=0.5,below]{c};
	
	\draw [color=black, thick] (6,1) -- (6,0);
	\draw [color=black, thick] (6,-1) -- (6,0);
	\draw [color=black, thick] (6,0) -- (7,0);
	
	\draw [color=black, thick] (7,0) -- (7,1);
	\draw [color=black, thick] (7,0) -- (7,-1);
	
	\draw (7,1) to[normal open switch, *-*](9,1);	
	\path (7,1) -- (9,1) node[pos=0.5,below]{a};
	
	\draw (7,-1) to[normal open switch, *-*](9,-1);	
	\path (7,-1) -- (9,-1) node[pos=0.5,below]{b};
	
	\draw [color=black, thick] (9,1) -- (9,0);
	\draw [color=black, thick] (9,-1) -- (9,0);
	\draw [color=black, thick] (9,0) -- (11,0);
	
	\end{circuitikz}
\end{center}

\end{subquestions}


