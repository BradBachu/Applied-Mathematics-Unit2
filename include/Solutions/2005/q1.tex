%------------------------------------------------------------------------------
% Author(s):
% Varaun Ramgoolie
% Copyright:
%  Copyright (C) 2020 Brad Bachu, Arjun Mohammed, Varaun Ramgoolie, Nicholas Sammy
%
%  This file is part of Applied-Mathematics-Unit2 and is distributed under the
%  terms of the MIT License. See the LICENSE file for details.
%
%  Description:
%     Year: 2005 C
%     Module: 1
%     Question: 1 
%------------------------------------------------------------------------------

\begin{subquestions}

%------------------------------------------------------------------------------
% 1 a -------------------------------------------------------------------------
%------------------------------------------------------------------------------

\subquestion

\begin{subsubquestions}
	
\subsubquestion

See \rtab{2005:q1:tab:TruthTab1}.
\begin{table}[ht]
	\centering
	\begin{tabular}{|c|c|c|}
		\hline
		p & q & p $\lor$ q \\
		\hline
		0 & 0 & 0 \\
		0 & 1 & 1 \\
		1 & 0 & 1 \\
		1 & 1 & 1 \\
		\hline
	\end{tabular}
	\caption{\label{2005:q1:tab:TruthTab1} Truth table of $p \lor q$.}
\end{table}

%-----------------------------------------------------------------------------

\subsubquestion

See \rtab{2005:q1:tab:TruthTab2}
\begin{table}[ht]
	\centering
	\begin{tabular}{|c|c|c|c|c|c|}
		\hline
		p & q & $\sim$ p & p $\implies$ q & $\sim$ p $\land$ q & (p $\implies$ q) $\iff$ ($\sim$ p $\land$ q) \\
		\hline
		0 & 0 & 1 & 1 & 0 & 0 \\
		0 & 1 & 1 & 1 & 1 & 1 \\
		1 & 0 & 0 & 0 & 0 & 1 \\
		1 & 1 & 0 & 1 & 0 & 0 \\
		\hline
	\end{tabular}
	\caption{\label{2005:q1:tab:TruthTab2} Truth table of $(p \implies q) \iff (\sim p \land q)$.}
\end{table}

%-----------------------------------------------------------------------------

\subsubquestion

\begin{subsubsubquestions}
	
\subsubsubquestion

Since the truth values of \rtab{2005:q1:tab:TruthTab1} and \rtab{2005:q1:tab:TruthTab2} are not the same, the expressions ($p \lor q$) and ($(p \implies q) \iff (\sim p \land q)$) are not logically equivalent.

%-----------------------------------------------------------------------------

\subsubsubquestion

From \rdef{mod1:defn:Tautology}, we can see that ($(p \implies q) \iff (\sim p ~\land q)$) is \textbf{not a tautology}.

\end{subsubsubquestions}

\end{subsubquestions}

%------------------------------------------------------------------------------
% 1 b -------------------------------------------------------------------------
%------------------------------------------------------------------------------

\subquestion

\begin{subsubquestions}
	
\subsubquestion

$(x ~\land \sim y)$ can be expressed as "I do Statistical Analysis and I do not do Applied Mathematics."

%-----------------------------------------------------------------------------
\subsubquestion

"I do Statistical Analysis so I will do Applied Mathematics" can be expressed as,
\begin{equation}
	x \implies y\,.
\end{equation}

\end{subsubquestions}

%------------------------------------------------------------------------------
% 1 c -------------------------------------------------------------------------
%------------------------------------------------------------------------------

\subquestion

Let $A$="$(l \land m \land n) \lor (l \land m ~\land \sim n)$". \\
Let $B$="$(l \land m)$". \\
$A$ then becomes,
\begin{equation}
	A \equiv (B \land n) \lor (B \land \sim n) \,.
\end{equation}

Using \rdef{mod1:law:Distributive} and \rdef{mod1:law:Absorptive}, $A$ becomes,
\begin{align}
	A & \equiv ((B \land n) \lor B) \land ((B \land n) \lor \sim n) \nn \\
	  & \equiv B \land ((B \land n) \lor \sim n) \,.
\end{align}

Let $C$="$(B \land n) \lor \sim n$". \\
Using \rdef{mod1:law:Distributive}, \rdef{mod1:law:Complement} and \rdef{mod1:law:Identity} on $C$, it becomes,
\begin{align}
	C & \equiv (B \land n) \lor \sim n \nn \\
	  & \equiv (\sim n \lor B) \land (\sim n \lor n) \nn \\
	  & \equiv (\sim n \lor B) \land T \nn \\
	  & \equiv (\sim n \lor B) \,.
\end{align}

From this, $A$ now becomes,
\begin{equation}
	A \equiv B \land (\sim n \lor B) \,.
\end{equation}

Using \rdef{mod1:law:Absorptive}, $A$ becomes,
\begin{align}
	A & \equiv B \land (\sim n \lor B) \nn \\
	  & \equiv B \,.
\end{align}

Therefore,
\begin{equation}
	(l \land m \land n) \lor (l \land m ~\land \sim n) \equiv l \land m \,.
\end{equation}

%------------------------------------------------------------------------------
% 1 d -------------------------------------------------------------------------
%------------------------------------------------------------------------------

\subquestion

See the switching circuit of $((a \land b) \lor (a \land c)) \land (a \lor b)$.
\begin{center}
	\begin{circuitikz}
		
	\draw [color=black, thick] (0,0) -- (2,0);
	\draw [color=black, thick] (2,0) -- (2,1);
	\draw [color=black, thick] (2,0) -- (2,-1);
	
	\draw (2,1) to[normal open switch, *-*](4,1);	
	\path (2,1) -- (4,1) node[pos=0.5,below]{a};
	
	\draw (4,1) to[normal open switch, *-*](6,1);
	\path (4,1) -- (6,1) node[pos=0.5,below]{b};
		
	\draw (2,-1) to[normal open switch, *-*](4,-1);
	\path (2,-1) -- (4,-1) node[pos=0.5,below]{a};
	
	\draw (4,-1) to[normal open switch, *-*](6,-1);
	\path (4,-1) -- (6,-1) node[pos=0.5,below]{c};
	
	\draw [color=black, thick] (6,1) -- (6,0);
	\draw [color=black, thick] (6,-1) -- (6,0);
	\draw [color=black, thick] (6,0) -- (7,0);
	
	\draw [color=black, thick] (7,0) -- (7,1);
	\draw [color=black, thick] (7,0) -- (7,-1);
	
	\draw (7,1) to[normal open switch, *-*](9,1);	
	\path (7,1) -- (9,1) node[pos=0.5,below]{a};
	
	\draw (7,-1) to[normal open switch, *-*](9,-1);	
	\path (7,-1) -- (9,-1) node[pos=0.5,below]{b};
	
	\draw [color=black, thick] (9,1) -- (9,0);
	\draw [color=black, thick] (9,-1) -- (9,0);
	\draw [color=black, thick] (9,0) -- (11,0);
	
	\end{circuitikz}
\end{center}

\end{subquestions}


