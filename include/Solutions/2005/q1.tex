%------------------------------------------------------------------------------
% Author(s):
% Varaun Ramgoolie
% Copyright:
%  Copyright (C) 2020 Brad Bachu, Arjun Mohammed, Nicholas Sammy, Kerry Singh
%
%  This file is part of Applied-Mathematics-Unit2 and is distributed under the
%  terms of the MIT License. See the LICENSE file for details.
%
%  Description:
%     Year: 2005 C
%     Module: 1
%     Question: 1 
%------------------------------------------------------------------------------

\begin{subquestions}

%------------------------------------------------------------------------------
% 1 a -------------------------------------------------------------------------
%------------------------------------------------------------------------------

\subquestion

\begin{subsubquestions}
	
\subsubquestion

See \rtab{2005:q1:tab:TruthTab1}.

\begin{table}[ht]
	\centering
	\begin{tabular}{|c|c|c|}
		\hline
		p & q & p $\lor$ q \\
		\hline
		0 & 0 & 0 \\
		0 & 1 & 1 \\
		1 & 0 & 1 \\
		1 & 1 & 1 \\
		\hline
	\end{tabular}
	\caption{\label{2005:q1:tab:TruthTab1} Truth table of $p \lor q$.}
\end{table}

\subsubquestion

See \rtab{2005:q1:tab:TruthTab2}

\begin{table}[ht]
	\centering
	\begin{tabular}{|c|c|c|c|c|c|}
		\hline
		p & q & $\sim$ p & p $\implies$ q & $\sim$ p $\land$ q & (p $\implies$ q) $\iff$ ($\sim$ p $\land$ q) \\
		\hline
		0 & 0 & 1 & 1 & 0 & 0 \\
		0 & 1 & 1 & 1 & 1 & 1 \\
		1 & 0 & 0 & 0 & 0 & 1 \\
		1 & 1 & 0 & 1 & 0 & 0 \\
		\hline
	\end{tabular}
	\caption{\label{2005:q1:tab:TruthTab2} Truth table of $(p \implies q) \iff (\sim p \land q)$.}
\end{table}

\subsubquestion

\begin{subsubsubquestions}
	
\subsubsubquestion

Since the truth values of \rtab{2005:q1:tab:TruthTab1} and \rtab{2005:q1:tab:TruthTab2} are not the same, the expressions ($p \lor q$) and ($(p \implies q) \iff (\sim p \land q)$) are not logically equivalent.

\subsubsubquestion

From \rdef{mod1:defn:Tautology}, we can see that ($(p \implies q) \iff (\sim p ~\land q)$) is \textbf{not a tautology}.

\end{subsubsubquestions}

\end{subsubquestions}

%------------------------------------------------------------------------------
% 1 b -------------------------------------------------------------------------
%------------------------------------------------------------------------------

\subquestion

\begin{subsubquestions}
	
\subsubquestion

$(x ~\land \sim y)$ can be expressed as "I do Statistical Analysis and I do not do Applied Mathematics."

\subsubquestion

"I do Statistical Analysis so I will do Applied Mathematics" can be expressed as,

\begin{equation}
	x \implies y\,.
\end{equation}

\end{subsubquestions}

%------------------------------------------------------------------------------
% 1 c -------------------------------------------------------------------------
%------------------------------------------------------------------------------

\subquestion

Using Laws ~\ref{mod1:section:BooleanAlgebraLaws}, we can see that

\begin{align*}
	\hspace{-10mm}
	(l \land m \land n) \lor (l \land m ~\land \sim n) & \equiv \underline{(l \land m \land n) \lor (l \land m ~\land \sim n)}
	& \text{using ~\ref{mod1:law:Distributive}}\,, \nn \\ 
	 												  & \equiv (\underline{\textcolor{red}{(l \land m \land n) \lor l)}} \land (\underline{\textcolor{blue}{(l \land m \land n) \lor m})} \land ((l \land m    \land n) \lor \sim n)
    & \text{using ~\ref{mod1:law:Absorptive}}\,, \nn \\ 
													  & \equiv \textcolor{red}{l} \land \textcolor{blue}{m} \land (\underline{(l \land m \land n) ~\lor \sim n})
	& \text{using ~\ref{mod1:law:Commutative}}\,, \nn \\ 
													  & \equiv l \land m \land (\underline{\textcolor{red}{\sim n \lor (l \land m \land n)}})
  	& \text{using ~\ref{mod1:law:Distributive}}\,, \nn \\
  													  & \equiv l \land m \land (\textcolor{red}{(\sim n \lor l)} \land \textcolor{red}{(\sim n \lor m)} \land \underline{\textcolor{red}{(\sim n \lor n)}})
    & \text{using ~\ref{mod1:law:Complement}}\,, \nn \\
    										    	  & \equiv l \land m \land (\underline{(\sim n \lor l) \land (\sim n \lor m) \land (\textcolor{red}{T})})
    & \text{using ~\ref{mod1:law:Identity}}\,, \nn \\    
    										    	  & \equiv \underline{l \land m \land (\textcolor{red}{(\sim n \lor l) \land (\sim n \lor m)})}
    & \text{using ~\ref{mod1:law:Associative}}\,, \nn \\  
    												  & \equiv \textcolor{red}{l \land \underline{(\sim n \lor l)}} \land \textcolor{blue}{m \land \underline{(\sim n \lor m)}}
    & \text{using ~\ref{mod1:law:Commutative}}\,, \nn \\     
    				    	    					  & \equiv \underline{l \land (\textcolor{red}{l ~\lor \sim n})} \land \underline{m \land (\textcolor{blue}{m ~\lor \sim n})}
    & \text{using ~\ref{mod1:law:Absorptive}}\,, \nn \\ 	
    												  & \equiv \textcolor{red}{l} \land \textcolor{blue}{m}\,, \\	
													  & \equiv l \land m\,. \\							  			    	    					  				 											
\end{align*}

%------------------------------------------------------------------------------
% 1 d -------------------------------------------------------------------------
%------------------------------------------------------------------------------

\subquestion

See the switching circuit of $((a \land b) \lor (a \land c)) \land (a \lor b)$.

\begin{circuitikz}
	\draw [color=black, thick] (0,0) -- (2,0);
	\draw [color=black, thick] (2,0) -- (2,1);
	\draw [color=black, thick] (2,0) -- (2,-1);
	
	\draw (2,1) to[normal open switch, *-*](4,1);	
	\path (2,1) -- (4,1) node[pos=0.5,below]{a};
	
	\draw (4,1) to[normal open switch, *-*](6,1);
	\path (4,1) -- (6,1) node[pos=0.5,below]{b};
		
	\draw (2,-1) to[normal open switch, *-*](4,-1);
	\path (2,-1) -- (4,-1) node[pos=0.5,below]{a};
	
	\draw (4,-1) to[normal open switch, *-*](6,-1);
	\path (4,-1) -- (6,-1) node[pos=0.5,below]{c};
	
	\draw [color=black, thick] (6,1) -- (6,0);
	\draw [color=black, thick] (6,-1) -- (6,0);
	\draw [color=black, thick] (6,0) -- (7,0);
	
	\draw [color=black, thick] (7,0) -- (7,1);
	\draw [color=black, thick] (7,0) -- (7,-1);
	
	\draw (7,1) to[normal open switch, *-*](9,1);	
	\path (7,1) -- (9,1) node[pos=0.5,below]{a};
	
	\draw (7,-1) to[normal open switch, *-*](9,-1);	
	\path (7,-1) -- (9,-1) node[pos=0.5,below]{b};
	
	\draw [color=black, thick] (9,1) -- (9,0);
	\draw [color=black, thick] (9,-1) -- (9,0);
	\draw [color=black, thick] (9,0) -- (11,0);
\end{circuitikz}

\end{subquestions}


