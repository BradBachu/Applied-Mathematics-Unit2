%------------------------------------------------------------------------------
% Author(s):
% Varaun Ramgoolie
% Copyright:
%  Copyright (C) 2020 Brad Bachu, Arjun Mohammed, Varaun Ramgoolie, Nicholas Sammy
%
%  This file is part of Applied-Mathematics-Unit2 and is distributed under the
%  terms of the MIT License. See the LICENSE file for details.
%
%  Description:
%     Year: 2005 C
%     Module: 2
%     Question: 3
%------------------------------------------------------------------------------

%------------------------------------------------------------------------------
% 3 a
%------------------------------------------------------------------------------

\begin{subquestions}

\subquestion
	
We are given the probability density function, $f(x)$, of a continuous random variable $X$.

We can use the property that probabilities must sum to one, \rprop{mod2:prop:ContinuousRV:1} for continuous random variables,
\begin{align}
	\int_{-\infty}^{\infty} f(x)\mathrm{d}x & = 1 \,. 
\end{align}
Performing the integral requires us to split the integral into three regions,
\begin{align}
	\int_{-\infty}^{1} f(x)\mathrm{d}x + \int_{1}^{4} f(x)\mathrm{d}x + \int_{4}^{\infty} f(x)\mathrm{d}x & = 1 \,. 
\end{align}
Since we know how the functions are defined in each region, we can now solve for $k$,
\begin{align}
	0 + \int_{1}^{4} kx^3\mathrm{d}x + 0  & = 1 \nn \\
	k \times \left[\frac{x^4}{4}\right]^4_1 & = 1 \nn \\
	k \times \left[\frac{4^4}{4} - \frac{1^4}{4} \right] & = 1 \nn \\
	k \times \left[\frac{256}{4} - \frac{1}{4} \right] & = 1 \nn \\
	\frac{255k}{4} & = 1 \nn \\
	\implies k & = \frac{4}{255} \,.
\end{align}

%------------------------------------------------------------------------------
% 3 b
%------------------------------------------------------------------------------

\subquestion

\begin{subsubquestions}
	
\subsubquestion

From Note ~\ref{mod2:note:ContinuousRV:Note1}, we know that we must integrate the probabilities in the given region,
\begin{align}
	P(2 \leq X) &= P(2\leq X \leq \infty) \nn \\
	           & = \int_2^{\infty} f(x)~\mathrm{d}x \,.
\end{align}
Again, we must now split up the integral over the regions which $f$ is defined,
\begin{align}
	P(2 \leq X)  & = \int_2^{4} f(x)~\mathrm{d}x+\int_4^{\infty} f(x)~\mathrm{d}x \,.
\end{align}
Finally, we can evaluate,
\begin{align}
                                    & = \int_{2}^{4} \frac{4x^3}{255} \mathrm{d}x + 0 \nn \\
	                                 & = \frac{4}{255} \times \int_2^4 x^3\mathrm{d}x \nn \\
	                                 & = \frac{4}{255} \times \left[\frac{x^4}{4}\right]^4_2 \nn \\
	                                 & = \frac{4}{255} \times \left[\frac{4^4}{4}-\frac{2^4}{4}\right] \nn \\
	                                 & = \frac{4}{255} \times \left[\frac{256}{4}-\frac{16}{4}\right] \nn \\
	                                 & = \frac{4}{255} \times \frac{240}{4} \nn \\
	                                 & = \frac{240}{255} = \frac{16}{17} \,.
\end{align}
	
%------------------------------------------------------------------------------

\subsubquestion

From \rdef{mod2:defn:ContinuousRV:Expectation}, we know that the expectation of a continuous random variable $X$ is given by,
\begin{align}
	E(X) & = \int_{-\infty}^{\infty} x f(x)\mathrm{d}x \,.
\end{align}
Splitting up the integral as before,
\begin{align}
E(X) & = \int_{-\infty}^{1}xf(x)\mathrm{d}x + \int_{1}^{4} xf(x)\mathrm{d}x + \int_{4}^{\infty} xf(x)\mathrm{d}x \,,
\end{align}
allows us to evaluate,
\begin{align}
E(X) & = 0 + \int_{1}^{4} \frac{4x^4}{255} \mathrm{d}x + 0 \nn \\
	     & = \frac{4}{255} \times \int_{1}^{4} x^4 \mathrm{d}x \nn \\
	     & = \frac{4}{255} \times \left[\frac{x^5}{5}\right]^4_1 \nn \\
	     & = \frac{4}{255} \times \left[\frac{4^5}{5} - \frac{1^5}{5}\right] \nn \\
	     & = \frac{4}{255} \times \left[\frac{1024}{5} - \frac{1}{5}\right] \nn \\
	     & = \frac{4}{255} \times \frac{1023}{5} \nn \\
	     & = \frac{4092}{1275} \approx 3.21 \,. 
\end{align}

From \rdef{mod2:defn:ContinuousRV:Variance}, we know that the variance is defined via,
\begin{equation}
	\text{Var}(X) = E(X^2) - {E(X)}^2 \,. \label{2005:q3:CRVar}
\end{equation}
Hence, all that is left to compute is $E(X^2)$. We can do this by using \req{mod2:eq:ContinuousRV:Variance}, 
\begin{align}
	E(X^2) & = \int_{-\infty}^{\infty} x^2 f(x)\mathrm{d}x \,.
\end{align}
Again, the integral must be split up appropriately,
\begin{align} 
E(X^2) & = \int_{-\infty}^{1} x^2 f(x)\mathrm{d}x+\int_{1}^{4} x^2 f(x)\mathrm{d}x+\int_{4}^{\infty} x^2 f(x)\mathrm{d}x \,,
\end{align}
so that we can evaluate as follows,
\begin{align}
E(X^2) & = 0 + \int_{1}^{4} \frac{4x^5}{255} \mathrm{d}x + 0 \nn \\
	& = \frac{4}{255} \times \int_{1}^{4} x^5 \mathrm{d}x \nn \\
	& = \frac{4}{255} \times \left[\frac{x^6}{6}\right]^4_1 \nn \\
	& = \frac{4}{255} \times \left[\frac{4^6}{6} - \frac{1^6}{6}\right] \nn \\
	& = \frac{4}{255} \times \left[\frac{4096}{6} - \frac{1}{6}\right] \nn \\
	& = \frac{4}{255} \times \frac{4095}{6} \nn \\
	& = \frac{16380}{1530} \approx 10.71 \,.
\end{align}	

Substituting these values into \req{2005:q3:CRVar}, we have,
\begin{align}
\text{Var}(X) & = E(X^2) - {E(X)}^2 \nn \\
	& = 10.71 - 3.21^2 \nn \\
	& = 0.406 \,.	
\end{align}

\end{subsubquestions}
	
%------------------------------------------------------------------------------
% 3 b
%------------------------------------------------------------------------------

\subquestion

\begin{subsubquestions}
	
\subsubquestion
Using \rdef{mod2:defn:ContinuousRV:CDF}, we know that the cumulative distribution function $F(x)$ is defined by,
\begin{align}
F(x) &= \int_{-\infty}^{x} f(t) ~\dd{t} \,.
\end{align}
Now, we must evaluate this integral over different values of $x$, corresponding to the intervals which $f$ is defined.

For $x<1$, 
\begin{align}
F(x) &=  \int_{-\infty}^{x} f(t) ~\dd{t} \nn \\
&= 0 \,.
\end{align}

For $1 \leq x < 4$,
\begin{align}
F(x) &=  \int_{-\infty}^{x} f(t) ~\dd{t} \nn \\
      &=  \int_{-\infty}^{1} f(t) ~\dd{t} +  \int_{1}^{x} f(t) ~\dd{t} \nn \\ 
      &= 0 + \int_1^x t^3 \dd t \nn \\
      & = \frac{4}{255} \times \left[\frac{t^4}{4}\right]^x_1 \nn \\
      & = \frac{4}{255} \times \left[\frac{x^4}{4} - \frac{1^4}{4} \right] \nn \\
      & = \frac{1}{255}\left(x^4 - 1 \right) \,.
\end{align}

For $x \geq 4$,
\begin{align}
F(x) &=  \int_{-\infty}^{x} f(t) ~\dd{t} \nn \\
      &= \int_{-\infty}^{1} f(t) ~\dd{t} + \int_{1}^{4} f(t) ~\dd{t} + \int_{4}^{x} f(t) ~\dd{t} \nn \\
      &= 0 + \frac{4^4-1}{255} + 0 \nn \\
      &= 1 \,.
\end{align}

Putting this information together, we have,
\begin{equation}
F(x) = \begin{cases} 
	0 , & x\leq 1 \\
	\frac{1}{255}\left(x^4 - 1 \right), & 1 \leq x < 4 \\
	1, & x \geq 4 \,. \label{eq:2005:q3:CDF}
\end{cases}
\end{equation}

%------------------------------------------------------------------------------

\subsubquestion

From \rprop{mod2:prop:ContinuousRV:CDF}, know that the lower quartile, $LQ$, is the value of $x$ such that
\begin{align}
	F(x =LQ) & = 0.25 \,.
\end{align}
Since the value of $F(LQ)$ is between $0$ and $1$, we know from our definition of the c.d.f above, \req{eq:2005:q3:CDF}, that $1 \leq LQ < 4$. Hence, we can find $LQ$ by equating the following,
\begin{align}
	\frac{1}{255}\left((LQ)^4 - 1 \right) & = 0.25 \nn \\
	 (LQ)^4 - 1 & = \frac{255}{4} \nn \\
	(LQ)^4 & = \frac{259}{4} \nn \\
	\implies LQ & = \sqrt[4]{\frac{259}{4}} \nn \\
	           & \approx 2.84 \,. 
\end{align}	

\end{subsubquestions}
	
\end{subquestions}
