%------------------------------------------------------------------------------
% Author(s):
% Varaun Ramgoolie
% Copyright:
%  Copyright (C) 2020 Brad Bachu, Arjun Mohammed, Varaun Ramgoolie, Nicholas Sammy
%
%  This file is part of Applied-Mathematics-Unit2 and is distributed under the
%  terms of the MIT License. See the LICENSE file for details.
%
%  Description:
%     Year: 2005 C
%     Module: 2
%     Question: 3
%------------------------------------------------------------------------------

%------------------------------------------------------------------------------
% 3 a
%------------------------------------------------------------------------------

\begin{subquestions}
	
\subquestion

We are given the probability density function, $f(x)$, of a continuous random variable $X$.

From \rprop{mod2:prop:ContinuousRV:1}, we know that,
\begin{align}
	\int_{-\infty}^{\infty} f(x)\mathrm{d}x & = 1 \nn \\
	\int_{-\infty}^{1} f(x)\mathrm{d}x + \int_{1}^{4} f(x)\mathrm{d}x + \int_{4}^{\infty} f(x)\mathrm{d}x & = 1 \nn \\
	0 + \int_{1}^{4} kx^3\mathrm{d}x + 0  & = 1 \nn \\
	k \times \left[\frac{x^4}{4}\right]^4_1 & = 1 \nn \\
	k \times \left[\frac{4^4}{4} - \frac{1^4}{4} \right] & = 1 \nn \\
	k \times \left[\frac{256}{4} - \frac{1}{4} \right] & = 1 \nn \\
	\frac{255k}{4} & = 1 \nn \\
	\implies k & = \frac{4}{255} \,.
\end{align}

%------------------------------------------------------------------------------
% 3 b
%------------------------------------------------------------------------------

\subquestion

\begin{subsubquestions}
	
\subsubquestion

From Note ~\ref{mod2:note:ContinuousRV:Note1}, we can get that,
\begin{align}
	P(2 \leq X) &= P(2\leq X \leq \infty) \nn \\
	                                 & = \int_2^{\infty} f(x)~\mathrm{d}x \nn \\
	                                 & = \int_2^{4} f(x)~\mathrm{d}x+\int_4^{\infty} f(x)~\mathrm{d}x \nn \\
	                                 & = \int_{2}^{4} \frac{4x^3}{255} \mathrm{d}x + 0 \nn \\
	                                 & = \frac{4}{255} \times \int_2^4 x^3\mathrm{d}x \nn \\
	                                 & = \frac{4}{255} \times \left[\frac{x^4}{4}\right]^4_2 \nn \\
	                                 & = \frac{4}{255} \times \left[\frac{4^4}{4}-\frac{2^4}{4}\right] \nn \\
	                                 & = \frac{4}{255} \times \left[\frac{256}{4}-\frac{16}{4}\right] \nn \\
	                                 & = \frac{4}{255} \times \frac{240}{4} \nn \\
	                                 & = \frac{240}{255} = \frac{16}{17} \,.
\end{align}
	
%------------------------------------------------------------------------------

\subsubquestion

From \rdef{mod2:defn:ContinuousRV:Expectation}, we get that,
\begin{align}
	E(X) & = \int_{-\infty}^{\infty} x f(x)\mathrm{d}x \nn \\
		 & = \int_{-\infty}^{1}xf(x)\mathrm{d}x + \int_{1}^{4} xf(x)\mathrm{d}x + \int_{4}^{\infty} xf(x)\mathrm{d}x \nn \\
		 & = 0 + \int_{1}^{4} \frac{4x^4}{255} \mathrm{d}x + 0 \nn \\
	     & = \frac{4}{255} \times \int_{1}^{4} x^4 \mathrm{d}x \nn \\
	     & = \frac{4}{255} \times \left[\frac{x^5}{5}\right]^4_1 \nn \\
	     & = \frac{4}{255} \times \left[\frac{4^5}{5} - \frac{1^5}{5}\right] \nn \\
	     & = \frac{4}{255} \times \left[\frac{1024}{5} - \frac{1}{5}\right] \nn \\
	     & = \frac{4}{255} \times \frac{1023}{5} \nn \\
	     & = \frac{4092}{1275} \approx 3.21 \,. 
\end{align}

From \rdef{mod2:defn:ContinuousRV:Variance}, we know that,
\begin{equation}
	\text{Var}(X) = E(X^2) - {E(X)}^2 \,. \label{2005:q3:CRVar}
\end{equation}

By using \req{mod2:eq:ContinuousRV:Variance}, we can calculate,
\begin{align}
	E(X^2) & = \int_{-\infty}^{\infty} x^2 f(x)\mathrm{d}x \nn \\
	& = \int_{-\infty}^{1} x^2 f(x)\mathrm{d}x+\int_{1}^{4} x^2 f(x)\mathrm{d}x+\int_{4}^{\infty} x^2 f(x)\mathrm{d}x \nn \\
	& = 0 + \int_{1}^{4} \frac{4x^5}{255} \mathrm{d}x + 0 \nn \\
	& = \frac{4}{255} \times \int_{1}^{4} x^5 \mathrm{d}x \nn \\
	& = \frac{4}{255} \times \left[\frac{x^6}{6}\right]^4_1 \nn \\
	& = \frac{4}{255} \times \left[\frac{4^6}{6} - \frac{1^6}{6}\right] \nn \\
	& = \frac{4}{255} \times \left[\frac{4096}{6} - \frac{1}{6}\right] \nn \\
	& = \frac{4}{255} \times \frac{4095}{6} \nn \\
	& = \frac{16380}{1530} \approx 10.71 \,.
\end{align}	

Thus, substituting these values into \req{2005:q3:CRVar}, we can find that,
\begin{align}
\text{Var}(X) & = E(X^2) - {E(X)}^2 \nn \\
	& = 10.71 - 3.21^2 \nn \\
	& = 0.406 \,.	
\end{align}

\end{subsubquestions}
	
%------------------------------------------------------------------------------
% 3 b
%------------------------------------------------------------------------------

\subquestion

\begin{subsubquestions}
	
\subsubquestion

Using \rdef{mod2:defn:ContinuousRV:CDF}, we are able to get that,
\begin{align}
	F(a) & = \int_{1}^{a} f(x)\mathrm{d}x \nn \\
	     & = \frac{4}{255} \times \int_{1}^{a} x^3\mathrm{d}x \nn \\
	     & = \frac{4}{255} \times \left[\frac{x^4}{4}\right]^a_1 \nn \\
	     & = \frac{4}{255} \times \left[\frac{a^4}{4} - \frac{1^4}{4} \right] \nn \\
	     & = \frac{1}{255}\left(a^4 - 1 \right) \,.
\end{align}
	
We will define,
\[ F(x) = \begin{cases} 
	0 , & x\leq 1 \\
	\frac{1}{255}\left(x^4 - 1 \right), & 1 \leq x \leq 4 \\
	1, & x \geq 4 \,.
\end{cases}
\]

%------------------------------------------------------------------------------

\subsubquestion

From \rprop{mod2:prop:ContinuousRV:CDF}, we can find that,
\begin{align}
	F(LQ) & = 0.25 \nn \\
	\frac{1}{255}\left((LQ)^4 - 1 \right) & = 0.25 \nn \\
	\text{Multiplying by 255} \implies (LQ)^4 - 1 & = \frac{255}{4} \nn \\
	(LQ)^4 & = \frac{259}{4} \nn \\
	\implies LQ & = \sqrt[4]{\frac{259}{4}} \nn \\
	           & \approx 2.84 \,. 
\end{align}	

\end{subsubquestions}
	
\end{subquestions}
