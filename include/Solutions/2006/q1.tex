%------------------------------------------------------------------------------
% Author(s):
% Varaun Ramgoolie
% Copyright:
%  Copyright (C) 2020 Brad Bachu, Arjun Mohammed, Varaun Ramgoolie, Nicholas Sammy
%
%  This file is part of Applied-Mathematics-Unit2 and is distributed under the
%  terms of the MIT License. See the LICENSE file for details.
%
%  Description:
%     Year: 2006 C
%     Module: 1
%     Question: 1 
%------------------------------------------------------------------------------

\begin{subquestions}

%------------------------------------------------------------------------------
% 1 a--------------------------------------------------------------------------
%------------------------------------------------------------------------------
 
\subquestion

\begin{subsubquestions}

%------------------------------------------------------------------------------

\subsubquestion

Inverse.

%------------------------------------------------------------------------------

\subsubquestion

Converse.

%------------------------------------------------------------------------------

\subsubquestion

Contrapositive.

\end{subsubquestions}
	
%------------------------------------------------------------------------------
% 1 b--------------------------------------------------------------------------
%------------------------------------------------------------------------------

\subquestion

\begin{subsubquestions}

%-----------------------------------------------------------------------------

\subsubquestion

See \rtab{2006:q1:tab:TruthTab1}.

\begin{table}[ht]
	\centering
	\begin{tabular}{|c|c|c|c|c|c|}
		\hline
		x & y & $\sim$ x & $\sim$ y & x $\implies$ y & $\sim$ y $\implies$ $\sim$ x \\
		\hline
		0 & 0 & 1 & 1 & 1 & 1 \\
		0 & 1 & 1 & 0 & 1 & 1 \\
		1 & 0 & 0 & 1 & 0 & 0 \\
		1 & 1 & 0 & 0 & 1 & 1 \\
		\hline
	\end{tabular}
	\caption{\label{2006:q1:tab:TruthTab1} Truth Table of $x \implies y$ and $\sim y \implies \sim x$.}
\end{table}

%-----------------------------------------------------------------------------
	
\subsubquestion

From \rtab{2006:q1:tab:TruthTab1}, we can see that the truth values of $x \implies y$ and $\sim y \implies \sim x$ are the same. Therefore, both expressions are logically equivalent.

\end{subsubquestions}

%------------------------------------------------------------------------------
% 1 c--------------------------------------------------------------------------
%------------------------------------------------------------------------------

\subquestion

\begin{subsubquestions}

%------------------------------------------------------------------------------

\subsubquestion

"You have your cake and eat it" can be expressed as,
\begin{equation}
	p ~\land \sim p \,.
\end{equation}

%------------------------------------------------------------------------------

\subsubquestion

From \rtab{2006:q1:tab:TruthTab2} and \rdef{mod1:defn:Contradiction}, we can see that $p ~\land \sim p$ is a contradiction.

\begin{table}[ht]
	\centering
	\begin{tabular}{|c|c|c|}
		\hline
		p & $\sim$ p & p $\land$ $\sim$ p \\
		\hline
		0 & 1 & 0 \\
		1 & 0 & 0 \\ 
		\hline
	\end{tabular}
	\caption{\label{2006:q1:tab:TruthTab2} Truth Table of $x \implies y$ and $\sim y \implies \sim x$.}
\end{table}

\end{subsubquestions}

%------------------------------------------------------------------------------
% 1 d--------------------------------------------------------------------------
%------------------------------------------------------------------------------

\subquestion

See \rfig{2006:q1:fig:ActNet}.
\begin{figure}
	\begin{center}
		\includegraphics{../2006/figures/2006q1ActNet}
		\caption{\label{2006:q1:fig:ActNet} Activity Network of the project.}
	\end{center}
\end{figure}

\end{subquestions}

