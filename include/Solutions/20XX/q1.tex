%%%%%%%%%%%%%%%%%%%%%%%%%%%%%%%%%%%%%%%%%%%%%%%%%%%%%%%%%%%%%%%%%%%%%%%%%
% 20XX Q1
%%%%%%%%%%%%%%%%%%%%%%%%%%%%%%%%%%%%%%%%%%%%%%%%%%%%%%%%%%%%%%%%%%%%%%%%%
\begin{enumerate}[label = \bfseries (\alph*)]

%------------------------------------------------------------------------
% 1 (a)
%------------------------------------------------------------------------
\item What we want to do.

Given a single equation, use
\begin{equation} \label{20XX:q1:name}
	F = m \times a
\end{equation}

For multiple equations use
\begin{align}
	F &= m \times a \\
	  &= 3\text{ kg} \times 2 \text{ ms}^{-1} \\
	  &= 6 \text{ N}
\end{align}

Most times, we do not need numbering on everything, in such
cases, use `nolabel' on all lines but the last one.
\begin{align}
	F &= m \times a \nonumber \\
	  &= 3\text{ kg} \times 2 \text{ ms}^{-1} \nonumber \\
	  &= 6 \text{ N}
\end{align}

%------------------------------------------------------------------------
% 1 (b)
%------------------------------------------------------------------------
\item What we want to do.

Reference equations via Eq.~(\ref{20XX:q1:name}).

Reference Notes and Definitions without parenthesis using the full word,
such as: see Note~\ref{mod1:note:name} or Theorem~\ref{mod1:thm:name}

\end{enumerate}

