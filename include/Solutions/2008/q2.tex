%------------------------------------------------------------------------------
% Author(s):
% Varaun Ramgoolie
% Copyright:
%  Copyright (C) 2020 Brad Bachu, Arjun Mohammed, Varaun Ramgoolie, Nicholas Sammy
%
%  This file is part of Applied-Mathematics-Unit2 and is distributed under the
%  terms of the MIT License. See the LICENSE file for details.
%
%  Description:
%     Year: 2008 June
%     Module: 1
%     Question: 2 
%------------------------------------------------------------------------------

\begin{subquestions}
	
%------------------------------------------------------------------------------
% 2 a -------------------------------------------------------------------------
%------------------------------------------------------------------------------	
	
\subquestion
	
In words, $(\sim p ~\land \sim q)$, is equivalent to "He is not tall and he is not happy."
	
%------------------------------------------------------------------------------
% 2 b -------------------------------------------------------------------------
%------------------------------------------------------------------------------	
	
\subquestion
	
Let $p$ = "London is in England," and let $q$ = "2 $\times$ 3 = 5". 

The statement in the question is equivalent to $(p \lor q)$. 
	
We know that $p$ has a truth value of 1 and $q$ has a truth value of 0. 
	
Thus, we get that the truth value of the statement is equal to,
\begin{align}
		p \lor q &= 1 \lor 0\,, \nn \\
		&= 1\,.
\end{align}
	
%------------------------------------------------------------------------------
% 2 c -------------------------------------------------------------------------
%------------------------------------------------------------------------------	
	
\subquestion
	
From \rtab{2008:q2:tab:TruthTab1} and \rdef{mod1:defn:Tautology}, wee see that $(p \land q) \implies (p \lor q)$ is a tautology.
\begin{table}[ht]
		\centering
		\begin{tabular}{|c|c|c|c|c|}
			\hline
			p & q & p $\land$ q & p $\lor$ q & (p $\land$ q) $\implies$ (p $\lor$ q) \\
			\hline
			0 & 0 & 0 & 0 & 1 \\
			0 & 1 & 0 & 1 & 1 \\
			1 & 0 & 0 & 1 & 1 \\
			1 & 1 & 1 & 1 & 1 \\
			\hline
		\end{tabular}
		\caption{\label{2008:q2:tab:TruthTab1} Showing the truth values of $(p \land q) \implies (p \lor q)$.}
\end{table}
	
%------------------------------------------------------------------------------
% 2 d -------------------------------------------------------------------------
%------------------------------------------------------------------------------	
	
\subquestion
	
\begin{subsubquestions}
		
\subsubquestion
		
See the switching circuit below.
\begin{center}
\begin{circuitikz}
			\draw [color=black, thin] (0,0) -- (2,0);
			\draw [color=black, thin] (2,0) -- (2,2);
			\draw [color=black, thin] (2,0) -- (2,-2);
			
			\draw (2,2) to[normal open switch, *-*](6,2);
			\draw (6,2) to[normal open switch, *-*](10,2);
			
			\path (2,2) -- (6,2) node[pos=0.5,below]{A};
			\path (6,2) -- (10,2) node[pos=0.5,below]{$\sim$ B};
			
			\draw [color=black, thin] (2,-2) -- (3,-2);
			\draw [color=black, thin] (3,-2) -- (3,-1);
			\draw [color=black, thin] (3,-2) -- (3,-3);
			
			\draw (3,-1) to[normal open switch, *-*](5,-1);
			\draw (3,-3) to[normal open switch, *-*](5,-3);	
			
			\path (3,-1) -- (5,-1) node[pos=0.5,below]{$\sim$ A};
			\path (3,-3) -- (5,-3) node[pos=0.5,below]{C};
			
			\draw [color=black, thin] (5,-1) -- (5,-2);
			\draw [color=black, thin] (5,-3) -- (5,-2);
			\draw [color=black, thin] (5,-2) -- (6,-2);
			
			\draw (6,-2) to[normal open switch, *-*](10,-2);
			\path (6,-2) -- (10,-2) node[pos=0.5,below]{B};
			
			\draw [color=black, thin] (10,-2) -- (10,0);
			\draw [color=black, thin] (10,2) -- (10,0);
			\draw [color=black, thin] (10,0) -- (12,0);
\end{circuitikz}	
\end{center}

%------------------------------------------------------------------------------

\subsubquestion
		
See \rfig{2008:q2:fig:LogicGate}.
\begin{figure}
		\begin{center}
				\includegraphics{../2008/figures/2008q2LogicGate}
				\caption{\label{2008:q2:fig:LogicGate} Showing the logic circuit of $\sim a \land (a \lor b)$.}
		\end{center}
\end{figure}
	
\end{subsubquestions}

%------------------------------------------------------------------------------
% 2 e -------------------------------------------------------------------------
%------------------------------------------------------------------------------

\subquestion

Let $Q$="$(A \land B) \lor (A ~\land \sim B) \lor (\sim A ~\land \sim B)$". \\

Let $R$="$(A \land B) \lor (A ~\land \sim B)$". \\

From this, we see that,
\begin{equation}
	Q = R \lor (\sim A ~\land \sim B). \label{2008:q2:BooleanMainEqn}
\end{equation}

Using \rdef{mod1:law:Distributive} and \rdef{mod1:law:Absorptive} on $R$, it becomes
\begin{align}
	R & = (A \land B) \lor (A ~\land \sim B) \nn \\
	  & = ((A \land B) \lor A) \land ((A \land B) ~\lor \sim B) \nn \\
	  & = A \land ((A \land B) ~\lor \sim B) \,.
\end{align}

Let $S$="$(A \land B) \lor \sim B$". \\

We can notice that,
\begin{equation}
	R = A \land S \,.
\end{equation}

Using \rdef{mod1:law:Distributive}, \rdef{mod1:law:Complement} and \rdef{mod1:law:Identity} on $S$, we can see that,
\begin{align}
	S & = (A \land B) ~\lor \sim B \nn \\
	  & = (\sim B \lor A) \land (\sim B \lor B) \nn \\
	  & = (\sim B \lor A) \land T \nn \\
	  & = (\sim B \lor A) \,.
\end{align}

Therefore, $R$ now becomes,
\begin{align}
	R & = A \land S \nn \\
	  & = A \land (\sim B \lor A) \,.
\end{align}

Using \rdef{mod1:law:Absorptive}, $R$ becomes,
\begin{align}
	R & = A \land (\sim B \lor A) \nn \\
	  & = A \,.
\end{align}

\req{2008:q2:BooleanMainEqn} now becomes,
\begin{equation}
	Q = A \lor (\sim A ~\land \sim B).
\end{equation}

Using \rdef{mod1:law:Distributive}, \rdef{mod1:law:Complement} and \rdef{mod1:law:Identity}, $Q$ becomes,
\begin{align}
	Q & = A \lor (\sim A ~\land \sim B) \nn \\
	  & = (A ~\lor \sim A) \land (A ~\lor \sim B) \nn \\
	  & = T \land (A ~\lor \sim B) \nn \\
	  & = (A ~\lor \sim B) \,. 
\end{align}

Thus,
\begin{equation}
	(A \land B) \lor (A ~\land \sim B) \lor (\sim A ~\land \sim B) \equiv (A ~\lor \sim B) \,.
\end{equation}

\end{subquestions}