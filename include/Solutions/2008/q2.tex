%------------------------------------------------------------------------------
% Author(s):
% Varaun Ramgoolie
% Copyright:
%  Copyright (C) 2020 Brad Bachu, Arjun Mohammed, Varaun Ramgoolie, Nicholas Sammy
%
%  This file is part of Applied-Mathematics-Unit2 and is distributed under the
%  terms of the MIT License. See the LICENSE file for details.
%
%  Description:
%     Year: 2008 June
%     Module: 1
%     Question: 2 
%------------------------------------------------------------------------------

\begin{subquestions}
	
%------------------------------------------------------------------------------
% 2 a -------------------------------------------------------------------------
%------------------------------------------------------------------------------	
	
\subquestion
	
The proposition $\boldsymbol{(\sim p ~\land \sim q)}$ can be expressed in words as "He is not tall and he is not happy."
	
%------------------------------------------------------------------------------
% 2 b -------------------------------------------------------------------------
%------------------------------------------------------------------------------	
	
\subquestion
	
Let $\boldsymbol{p}$ = "London is in England," and let $\boldsymbol{q}$ = "2 $\times$ 3 = 5". 

"London is in England or 2 x 3 = 5" can be expressed as the proposition $\boldsymbol{p \lor q}$. 
	
We know that $\boldsymbol{p}$ has a truth value of 1 and $\boldsymbol{q}$ has a truth value of 0. 
	
Thus, by referring to \rdef{mod1:tab:Conjunction} we get that the truth value of the statement is equal to,
\begin{align}
		\boldsymbol{p \lor q} &= \boldsymbol{1 \lor 0} \nn \\
		&= \boldsymbol{1} \,.
\end{align}
	
%------------------------------------------------------------------------------
% 2 c -------------------------------------------------------------------------
%------------------------------------------------------------------------------	
	
\subquestion
	
We can construct the following truth table for the proposition $\boldsymbol{(p \land q) \implies (p \lor q)}$
\begin{table}[ht]
		\centering
		\begin{tabular}{|c|c|c|c|c|}
			\hline
			$\boldsymbol{p}$ & $\boldsymbol{q}$ & $\boldsymbol{p \land q}$ & $\boldsymbol{p \lor q}$ & $\boldsymbol{(p \land q) \implies (p \lor q)}$ \\
			\hline
			0 & 0 & 0 & 0 & 1 \\
			0 & 1 & 0 & 1 & 1 \\
			1 & 0 & 0 & 1 & 1 \\
			1 & 1 & 1 & 1 & 1 \\
			\hline
		\end{tabular}
		\caption{\label{2008:q2:tab:TruthTab1} Showing the truth values of $\boldsymbol{(p \land q) \implies (p \lor q)}$.}
\end{table}

We can see that the proposition $\boldsymbol{(p \land q) \implies (p \lor q)}$ is a tautology (\rdef{mod1:defn:Tautology}) since it always has a truth value of 1.
%------------------------------------------------------------------------------
% 2 d -------------------------------------------------------------------------
%------------------------------------------------------------------------------	
	
\subquestion
	
\begin{subsubquestions}
		
\subsubquestion
		
We can represent the boolean expression $\boldsymbol{(A \ \land \sim B) \lor ((\sim A \lor C) \land B)}$ as the following switching circuit.
\begin{center}
\begin{circuitikz}
			\draw [color=black, thin] (0,0) -- (2,0);
			\draw [color=black, thin] (2,0) -- (2,2);
			\draw [color=black, thin] (2,0) -- (2,-2);
			
			\draw (2,2) to[normal open switch, *-*](6,2);
			\draw (6,2) to[normal open switch, *-*](10,2);
			
			\path (2,2) -- (6,2) node[pos=0.5,below]{A};
			\path (6,2) -- (10,2) node[pos=0.5,below]{$\sim$ B};
			
			\draw [color=black, thin] (2,-2) -- (3,-2);
			\draw [color=black, thin] (3,-2) -- (3,-1);
			\draw [color=black, thin] (3,-2) -- (3,-3);
			
			\draw (3,-1) to[normal open switch, *-*](5,-1);
			\draw (3,-3) to[normal open switch, *-*](5,-3);	
			
			\path (3,-1) -- (5,-1) node[pos=0.5,below]{$\sim$ A};
			\path (3,-3) -- (5,-3) node[pos=0.5,below]{C};
			
			\draw [color=black, thin] (5,-1) -- (5,-2);
			\draw [color=black, thin] (5,-3) -- (5,-2);
			\draw [color=black, thin] (5,-2) -- (6,-2);
			
			\draw (6,-2) to[normal open switch, *-*](10,-2);
			\path (6,-2) -- (10,-2) node[pos=0.5,below]{B};
			
			\draw [color=black, thin] (10,-2) -- (10,0);
			\draw [color=black, thin] (10,2) -- (10,0);
			\draw [color=black, thin] (10,0) -- (12,0);
\end{circuitikz}	
\end{center}

%------------------------------------------------------------------------------

\subsubquestion
		
We can represent the Boolean expression $\boldsymbol{\sim a \land ( a \lor b)}$ as the following logic circuit.
\begin{figure}[H]
		\begin{center}
				\includegraphics{../2008/figures/2008q2LogicGate}
				\caption{\label{2008:q2:fig:LogicGate} Showing the logic circuit of $\boldsymbol{\sim a \land (a \lor b)}$.}
		\end{center}
\end{figure}
	
\end{subsubquestions}

%------------------------------------------------------------------------------
% 2 e -------------------------------------------------------------------------
%------------------------------------------------------------------------------

\subquestion

Let $\boldsymbol{Q} = \boldsymbol{(A \land B) \lor (A ~\land \sim B) \lor (\sim A ~\land \sim B)}$. \\

Let $\boldsymbol{R} = \boldsymbol{(A \land B) \lor (A ~\land \sim B)}$ \\ 

By using the Distributive Law (\rdef{mod1:law:Distributive}) in reverse and the Complement Law (\rdef{mod1:law:Complement}), $\boldsymbol{R}$ becomes
\begin{align}
	\boldsymbol{R} & \equiv \boldsymbol{A \land (B \ \lor \sim B)} \nn \\ 
					& \equiv \boldsymbol{A \land T} \,.
\end{align}

Therefore, we can rewrite $\boldsymbol{Q}$ 
\begin{equation}
	\boldsymbol{Q} \equiv \boldsymbol{(A \land T) \lor (\sim A \ \land \sim B)} \,.
\end{equation}

By using the Identity Law (\rdef{mod1:law:Identity}) and Distributive Law (\rdef{mod1:law:Distributive}), $\boldsymbol{Q}$ becomes 
\begin{align}
	\boldsymbol{Q} & \equiv \boldsymbol{A \lor (\sim A \ \land \sim B)} \nn \\
					& \equiv \boldsymbol{(A \ \lor \sim A) \land (A \ \lor \sim B)} \,.
\end{align}

Finally by using the Complement (\rdef{mod1:law:Complement}) and Identity (\rdef{mod1:law:Identity}) Laws,
\begin{align}
	\boldsymbol{Q} & \equiv \boldsymbol{T \land (A \ \lor \sim B)} \nn \\
					&  \equiv \boldsymbol{A \ \lor \sim B} \,.
\end{align}

Thus,
\begin{equation}
	\boldsymbol{(A \land B) \lor (A ~\land \sim B) \lor (\sim A ~\land \sim B) \equiv (A ~\lor \sim B)} \,.
\end{equation}

\end{subquestions}