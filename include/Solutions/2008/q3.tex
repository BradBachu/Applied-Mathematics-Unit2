%------------------------------------------------------------------------------
% Author(s):
% Varaun Ramgoolie
% Copyright:
%  Copyright (C) 2020 Brad Bachu, Arjun Mohammed, Varaun Ramgoolie, Nicholas Sammy
%
%  This file is part of Applied-Mathematics-Unit2 and is distributed under the
%  terms of the MIT License. See the LICENSE file for details.
%
%  Description:
%     Year: 2008 June
%     Module: 2
%     Question: 3 
%------------------------------------------------------------------------------

%------------------------------------------------------------------------------
% 3 a
%------------------------------------------------------------------------------

\begin{subquestions}
	
\subquestion

We being by observing that the first digit must be greater than or equal to 5 in order for our number to be greater than 500,000. Thus, the first digit must be in the set \{5, 6, 7, 8, 9\}. Similarly, for our number to be odd, the final digit must be odd (from the set \{5,7,9\}). Since repetition is allowed when choosing our digits, we find that we have 5 choices for the first digit, 6 choices for the second, third, fourth and fifth digit and, 3 choices for our last digit. Thus, by \rdef{mod2:defn:MultiplicationRule}, we get that,
\begin{align}
	\text{Number of odd numbers greater than 500,000} & = 5 \times 6 \times 6 \times 6 \times 6 \times 3 \nn \\
	                                                  & = 19440 \,.
\end{align}

%------------------------------------------------------------------------------
% 3 b
%------------------------------------------------------------------------------

\subquestion

\begin{subsubquestions}
	
\subsubquestion

Given that there are no restrictions, we need to choose 4 people from a group of 13. From \rdef{mod2:defn:CombinationEqn}, we get that,
\begin{align}
	^{13}C_4 & = \frac{13!}{4! \times (13-4)!} \nn \\
	         & = \frac{13!}{4! \times 9!} \nn \\
	         & = 715 \,.
\end{align}
	
%------------------------------------------------------------------------------

\subsubquestion

If the team contains 2 girls, it must also contain 2 boys. Therefore, we must choose 2 girls and 2 boys from the group as follows,
\begin{align}
	\text{Team with exactly 2 girls} & = ~^6C_2 \times ~^7C_2 \nn \\
	                                 & = \frac{6!}{2! \times (6-2)!} \times \frac{7!}{2! \times (7-2)!} \nn \\
	                                 & = \frac{6!}{2! \times 4!} \times \frac{7!}{2! \times 5!} \nn \\
	                                 & = 15 \times 21 \nn \\
	                                 & = 315 \,.
\end{align}

%------------------------------------------------------------------------------

\subsubquestion

In order for the team to have more girls than boys, there must either be 4 girls chosen or 3 girls chosen and 1 boy. Using \rdef{mod2:defn:AdditionRule} and \rdef{mod2:defn:MultiplicationRule}, we see that this can be given as,
\begin{align}
	\text{Teams with more girls} & = \text{Teams with 4 girls} + \text{Teams with 3 girls, 1 boy} \nn \\
	                                       & = ~^6C_4 + \left(^6C_3 \times ~^7C_1 \right) \nn \\
	                                       & = \frac{6!}{4! \times (6-4)!} + \left(\frac{6!}{3! \times (6-3)!} \times \frac{7!}{1! \times (7-1)!} \right) \nn \\
	                                       & = \frac{6!}{4! \times 2!} + \left(\frac{6!}{3! \times 3!} \times \frac{7!}{1! \times 6!} \right) \nn \\
	                                       & = 15 + (20 \times 7) \nn \\
	                                       & = 155 \,.
\end{align}

\end{subsubquestions}

%------------------------------------------------------------------------------
% 3 c
%------------------------------------------------------------------------------

\subquestion

\begin{subsubquestions}
	
\subsubquestion
	
We are given that $A$ and $B$ are independent. From \rdef{mod2:defn:Independent}, we know,
\begin{equation}
	P(A \land B) = P (A) \times P(B) \,.
\end{equation}

Using \rdef{mod2:defn:Conditional}, we can express $P(A|B)$ as,
\begin{equation}
P(A|B) = \frac{P(A \land B)}{P(B)} \,.
\end{equation}

Combining this with the definition of independence, we see that,
\begin{align}
	P(A|B) &=  \frac{P(A \land B)}{P(B)} \nn\\
	&= \frac{P(A) \times P(B)}{P(B)} \nn\\
	&= P(A) \,.
\end{align}

Thus, 
\begin{align}
	P(A) &= P(A|B) \nn\\
		  &= \frac{2}{3}\,.
\end{align}
	
%------------------------------------------------------------------------------

\subsubquestion

Using $P(A)=\frac{2}{3}$, we can find $P(B)$ by using the independence,
\begin{align}
	P(A \land B) = P (A) \times P(B)  \,.
\end{align}

Substituting the values
\begin{align}
\frac{1}{3} &= \frac{2}{3} \times P(B) \nn \\
	                \implies P(B) & = \frac{1}{3} \times \frac{3}{2} \nn \\
	                              & = \frac{1}{2} \,.
\end{align}

%------------------------------------------------------------------------------

\subsubquestion

We should note that,
\begin{equation}
	P(A') = 1 - P(A) \,.
\end{equation}

Using this, we can get that,
\begin{align}
	P(A' \land B') & = P(A') \times P(B') \nn \\
	               & = (1-P(A)) \times (1-P(B)) \nn \\
	               & = \left(1 - \frac{2}{3}\right) \times \left(1 - \frac{1}{2}\right) \nn \\
	               & = \frac{1}{3} \times \frac{1}{2} \nn \\
	               & = \frac{1}{6} \,.
\end{align}

\end{subsubquestions}

%------------------------------------------------------------------------------
% 3 d
%------------------------------------------------------------------------------

\subquestion

From \rdef{mod2:defn:MutuallyExclusive}, we know that $A$ and $B$ are mutually exclusive if and only if,
\begin{equation}
	P(A \land B) = 0 \,.
\end{equation}

Given that we know $P(A \land B) = \frac{1}{3}$, we can conclude that $A$ and $B$ are not mutually exclusive.

\end{subquestions}