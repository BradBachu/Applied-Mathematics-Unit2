%------------------------------------------------------------------------------
% Author(s):
% Varaun Ramgoolie
% Copyright:
%  Copyright (C) 2020 Brad Bachu, Arjun Mohammed, Varaun Ramgoolie, Nicholas Sammy
%
%  This file is part of Applied-Mathematics-Unit2 and is distributed under the
%  terms of the MIT License. See the LICENSE file for details.
%
%  Description:
%     Year: 2008 June
%     Module: 1
%     Question: 1 
%------------------------------------------------------------------------------

\begin{subquestions}
	
%------------------------------------------------------------------------------
% 1 a -------------------------------------------------------------------------
%------------------------------------------------------------------------------

\subquestion

The objective function $P$, profit, that we want to maximize is,
\begin{equation}
	P=15x + 27y \,. 
\end{equation}

Using $x$ as the number of units of product X and $y$ as the number of units of product Y, our inequalities are,
\begin{align}
	\text{Minimum constraint:} \ & x \geq 0 \,, \nn \\
	\text{Minimum constraint:} \ & y \geq 0 \,, \nn \\
	\text{Time constraint:} \ & 6x + 9y \leq 360 \implies 2x + 3y \leq 120 \,, \nn \\
	\text{Cost constraint:} \ & 15x + 9y \leq 675 \implies 5x + 3y \leq 225 \,.
\end{align}

%------------------------------------------------------------------------------
% 1 b -------------------------------------------------------------------------
%------------------------------------------------------------------------------

\subquestion

The feasible region of this linear programming problem is shaded in \rfig{2008J:q1:fig:Graph}.

\begin{center}

\begin{figure}[H]
	\begin{center}
		\includegraphics{../2008/figures/2008Jq1Graph}
		\caption{\label{2008J:q1:fig:Graph} Linear Programming Graph.}
	\end{center}
\end{figure}

\end{center}

%------------------------------------------------------------------------------
% 1 c -------------------------------------------------------------------------
%------------------------------------------------------------------------------

\subquestion

We can perform a tour of the vertices (\rdef{mod1:defn:TourOfVertices}) to find the values of $x$ and $y$ that maximize the objective function $P$

\begin{table}[H]
	\centering
	\begin{tabular}{|c|c|}
		\hline
		Vertice & $P = 15x + 27y$ \\
		\hline
		(0, 0) & 0 \\
		(0, 40) & 1080 \\
		(35, $\frac{150}{9}$) & 705 \\
		(45, 0) & 675 \\
		\hline
	\end{tabular}
	\caption{\label{2008:q1:tab:Profit} Tour of Vertices}
\end{table}

Thus, from \rtab{2008:q1:tab:Profit}, the maximum value of $P$ is $1080$, when $x = 0$ and $y = 40$.

\end{subquestions}

