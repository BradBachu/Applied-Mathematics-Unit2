%------------------------------------------------------------------------------
% Author(s):
% Varaun Ramgoolie
% Copyright:
%  Copyright (C) 2020 Brad Bachu, Arjun Mohammed, Varaun Ramgoolie, Nicholas Sammy
%
%  This file is part of Applied-Mathematics-Unit2 and is distributed under the
%  terms of the MIT License. See the LICENSE file for details.
%
%  Description:
%     Year: 2011
%     Module: 1
%     Question: 2 
%------------------------------------------------------------------------------

\begin{subquestions}

%-----------------------------------------------------------------------------
% 2 a
%-----------------------------------------------------------------------------

\subquestion

The Hungarian algorithm is shown in \rtab{2011:q2:tab:HungAlgo}.
\begin{table}[!hbt]
	\begin{minipage}{0.3\textwidth}
		\centering
		\begin{tabular}{cccc}
			13 & 15 & 14 & 17 \\
			18 & 16 & 13 & 13 \\
			12 & 14 & 11 & 16 \\
			13 & 13 & 15 & 13 \\
		\end{tabular}
		\captionsetup{width=1.1\linewidth}
		\caption*{Matrix From question}
	\end{minipage}
	\hspace{20pt}
	%-----------------------------------------------------------------------------
	\begin{minipage}{0.3\textwidth}
		\centering
		\begin{tabular}{cccc}
			0 & 2 & 1 & 4 \\
			5 & 3 & 0 & 0 \\
			1 & 3 & 0 & 5 \\
			0 & 0 & 2 & 0 \\
		\end{tabular}
		\captionsetup{width=1.1\linewidth}
		\caption*{Matrix after Reducing Rows}
	\end{minipage}
	\hspace{20pt}
	%-----------------------------------------------------------------------------
	\begin{minipage}{0.3\textwidth}
		\centering
		\begin{tabular}{cccc}
			0 & 2 & 1 & 4 \\
			5 & 3 & 0 & 0 \\
			1 & 3 & 0 & 5 \\
			0 & 0 & 2 & 0 \\
		\end{tabular}
		\captionsetup{width=1.1\linewidth}
		\caption*{Matrix after Reducing Columns} 
	\end{minipage}
	\vspace{20pt} 
	%-----------------------------------------------------------------------------
	\begin{minipage}{0.3\textwidth}
		\centering
		\begin{tabular} {cccccc}
			&   &        &\hspace{-3.25mm} \hvs{v1}      &   &                       \\ 
   \hhs{h1} & 0 &      2 &                             1 & 4 & \hhe[red]{h1}         \\
   \hhs{h2} & 5 &      3 &                             0 & 0 & \hhe[red]{h2}         \\
			& 1 &      3 &                             0 & 5 &                       \\
   \hhs{h3}	& 0 &      0 &                             2 & 0 & \hhe[red]{h3}         \\
			&   &        &\hspace{-3.25mm} \hve[red]{v1} &   &                       \\
		\end{tabular}
		\captionsetup{width=1.1\linewidth}
		\caption*{Shading 0's using the least \\ \centering number of lines}
	\end{minipage}
	\caption{\label{2011:q2:tab:HungAlgo} Showing the steps of the Hungarian Algorithm.}
\end{table}

From \rtab{2011:q2:tab:HungAlgo}, we can see that the possible pairings for the runners are as follows:
\begin{align}
	&\text{Person A} \rightarrow \text{Position 1}\,, \nn \\
	&\text{Person B} \rightarrow \text{Position 3, Position 4}\,, \nn \\
	&\text{Person C} \rightarrow \text{Position 3}\,, \nn \\
	&\text{Person D} \rightarrow \text{Position 1, Position 2, Position 4}\,. 
\end{align}

Therefore, the matchings to minimize the total time are,
\begin{align}
	&\text{Person A} \rightarrow \text{Position 1}\,, \nn \\
	&\text{Person B} \rightarrow \text{Position 4}\,, \nn \\
	&\text{Person C} \rightarrow \text{Position 3}\,, \nn \\
	&\text{Person D} \rightarrow \text{Position 2}\,.  
\end{align}

Therefore, the minimum time for the entire race is,
\begin{equation}
	\text{Total time} = 13+13+11+13=50 \, \text{seconds}\,.	
\end{equation}

%-----------------------------------------------------------------------------
% 2 b
%-----------------------------------------------------------------------------

\subquestion

\begin{subsubquestions}
	
\subsubquestion

See \rfig{2011:q2:fig:ActNet}.
\begin{figure}
	\begin{center}
		\includegraphics{../2011/figures/2011q2ActNet}
		\caption{\label{2011:q2:fig:ActNet} Activity Network of the operation.}
	\end{center}
\end{figure}

%-----------------------------------------------------------------------------

\subsubquestion

See \rtab{2011:q2:tab:Table1}.
\begin{table}[ht]
	\centering
	\begin{tabular}{|c|c|c|c|}
		\hline
		Activity & Earliest Start Time & Latest Start Time & Float Time \\
		\hline
		A & 0 & 0 & 0 \\
		B & 0 & 21 & 21 \\
		C & 18 & 18 & 0 \\
		D & 7 & 7 & 0 \\
		E & 18 & 22 & 4 \\
		\hline
	\end{tabular}
	\caption{\label{2011:q2:tab:Table1} Float times of the activities.}
\end{table}
	
%-----------------------------------------------------------------------------

\subsubquestion

\begin{subsubsubquestions}
	
\subsubsubquestion

From \rdef{mod1:defn:CritPath}, the critical path is,
\begin{equation}
	\text{Start} \rightarrow A \rightarrow D \rightarrow C \rightarrow \text{End}\,.
\end{equation}

%-----------------------------------------------------------------------------

\subsubsubquestion

The minimum completion time of this project is 24 hours.

\end{subsubsubquestions}

\end{subsubquestions}

\end{subquestions}


