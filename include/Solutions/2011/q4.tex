%------------------------------------------------------------------------------
% Author(s):
% Varaun Ramgoolie
% Copyright:
%  Copyright (C) 2020 Brad Bachu, Arjun Mohammed, Varaun Ramgoolie, Nicholas Sammy
%
%  This file is part of Applied-Mathematics-Unit2 and is distributed under the
%  terms of the MIT License. See the LICENSE file for details.
%
%  Description:
%     Year: 2011
%     Module: 2
%     Question: 4 
%------------------------------------------------------------------------------

%------------------------------------------------------------------------------
% 4 a
%------------------------------------------------------------------------------

\begin{subquestions}
	
\subquestion

We are given that accidents occur at an average rate of 3 per week on a particular stretch of road. We will let $X$ represent the amount of accidents which occur in one week. Thus, we can see that $X$ follows a Poisson distribution, given as,
\begin{equation}
	X \sim \text{Pois}(3) \,.
\end{equation}
	
\begin{subsubquestions}
	
\subsubquestion

We want to find $P(X=0)$. From \rdef{mod2:defn:Poisson}, we know that,
\begin{equation}
	P(X = x) =\frac{ 3 ^ x \times e^{-3}}{x!} \,. \label{2011:q4:Poisson1}
\end{equation}
	
Thus, we can calculate $P(X=0)$ as,
\begin{align}
	P(X=0) & = \frac{ 3 ^0 \times e^{-3}}{0!} \nn \\
	       & = \frac{ 1 \times e^{-3}}{1} \nn \\
	       & = \frac{1}{e^3} \,.
\end{align}
	
%------------------------------------------------------------------------------

\subsubquestion

We want to find $P(X > 3)$. Since the Poisson Distribution is discrete, we get that,
\begin{align}
	P(X>3) & =1-P(X \leq 3) \nn \\
	       & = 1 - [P(X=0)+P(X=1)+P(X=2)+P(X=3)] \,.
\end{align}

Using \req{2011:q4:Poisson1}, we can calculate $P(X>3)$ as follows,
\begin{align}
	P(X=0) & = \frac{1}{e^3} \,. \\ \nn \\
	P(X=1) & = \frac{ 3^1 \times e^{-3}}{1!} \nn \\
	       & = \frac{3 \times e^{-3}}{1} \nn \\
	       & = \frac{3}{e^3} \,. \\ \nn \\
	P(X=2) & = \frac{ 3^2 \times e^{-3}}{2!} \nn \\
	       & = \frac{9 \times e^{-3}}{2} \nn \\
	       & = \frac{9}{2e^3} \,. \\ \nn \\
	P(X=3) & = \frac{ 3^3 \times e^{-3}}{3!} \nn \\
	       & = \frac{27 \times e^{-3}}{6} \nn \\
	       & = \frac{27}{6e^3} \,. \\ \nn \\
	P(X>3) & =1-P(X \leq 3) \nn \\
		   & = 1 - [P(X=0)+P(X=1)+P(X=2)+P(X=3)] \nn \\
		   & = 1 - \left(\frac{1}{e^3} + \frac{3}{e^3} + \frac{9}{2e^3} + \frac{27}{6e^3} \right) \nn \\
		   & = 1 - \frac{14}{e^3} \nn \\
		   & = 0.303 \,.
\end{align}

%------------------------------------------------------------------------------

Given that there are 3 accidents in 1 week, there are therefore 6 accidents on average during a fortnight (2 week period). Thus, we will let $Y$ represent the number of accidents in a 2 week period. Since we know that $Y$ also follows a Poisson distribution, we can denote it as,
\begin{equation}
	Y \sim \text{Pois}(6) \,.
\end{equation} 

From \rdef{mod2:defn:Poisson}, we know that,
\begin{equation}
	P(Y = y) =\frac{ 6 ^ y \times e^{-6}}{y!} \,.
\end{equation}

Thus, we can find $P(Y=6)$ as follows,
\begin{align}
	P(Y = 6) & = \frac{ 6 ^ 6 \times e^{-6}}{6!} \nn \\	
	         & = \frac{46656 \times e^{-6}}{720} \nn \\
	         & = \frac{324}{5 \times e^6} \nn \\
	         & = 0.161 \,.
\end{align}

\end{subsubquestions}
	
%------------------------------------------------------------------------------
% 4 b
%------------------------------------------------------------------------------
	
\subquestion

It is given that the probability of a student passing the Statistics examination is $0.84$. If we let $X$ be the number of students that pass an examination, we can see that $X$ follows a Binomial distribution denoted as,
\begin{equation}
	X \sim \text{Bin}(0.84) \,.
\end{equation} 

From \rdef{mod2:defn:Binomial}, we know that,
\begin{equation}
	P(X = x) = { n \choose x} (0.84)^x (1-0.84)^{n-x} \,. \label{2011:q4:Bin1}
\end{equation}

In a group of 12 students, we want to find the probability that more than two thirds of the group passes the examination. That is, we want to find $P(X > 8)$. Since X is discrete, we know that,
\begin{equation}
	P(X > 8) = P(X=9)+P(X=10)+P(X=11)+P(X=12) \,.
\end{equation}

Using \req{2011:q4:Bin1}, we can calculate $P(X > 8)$ as follows,
\begin{align}
	P(X=9) & = { 12 \choose 9} \times (0.84)^9 \times (1-0.84)^{12-9} \nn \\
	       & = 220 \times (0.84^9) \times (0.16)^3 \nn \\
	       & = 0.188 \,. \\ \nn \\
	P(X=10) & = { 12 \choose 10} \times (0.84)^{10} \times (1-0.84)^{12-10} \nn \\
	        & = 66 \times (0.84)^{10} \times (0.16)^2 \nn \\
	        & = 0.296 \,. \\ \nn \\
	P(X=11) & = { 12 \choose 11} \times (0.84)^{11} \times (1-0.84)^{12-11} \nn \\
	        & = 12 \times (0.84)^{11} \times 0.16 \nn \\
	        & = 0.282 \,. \\ \nn \\
	P(X=12) & = { 12 \choose 12} \times (0.84)^{12} \times (1-0.84)^{12-12} \nn \\
	        & = 1 \times (0.84)^{12} \times 1 \nn \\
	        & = 0.123 \,. \\ \nn \\
	P(X > 8) & = P(X=9)+P(X=10)+P(X=11)+P(X=12) \nn \\
	         & = 0.188+0.296+0.282+0.123 \nn \\
	         & = 0.889 \,.        
\end{align}

%------------------------------------------------------------------------------
% 4 c
%------------------------------------------------------------------------------

\subquestion

In a box, there are 4 white marbles, 2 red marbles, and 3 black marbles. Three marbles are drawn from the 9 marbles without replacement.  

\begin{subsubquestions}
	
\subsubquestion

We want to find the probability that the marbles will be drawn in the order of white $\rightarrow$ black $\rightarrow$ red. The probability that the first marble drawn is white is $\frac{4}{9}$. The probability that the second marble drawn is black is $\frac{3}{8}$. The probability that the third marble drawn is red is $\frac{2}{7}$. Thus, from \rdef{mod2:defn:MultiplicationRule}, 
\begin{align}
	P(white \rightarrow black \rightarrow red) & = \frac{4}{9} \times \frac{3}{8} \times \frac{2}{7} \nn \\
	                                           & = \frac{1}{21} \,.
\end{align}
	
\subsubquestion


\end{subsubquestions}





\end{subquestions}