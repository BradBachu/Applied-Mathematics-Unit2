%------------------------------------------------------------------------------
% Author(s):
% Varaun Ramgoolie
% Copyright:
%  Copyright (C) 2020 Brad Bachu, Arjun Mohammed, Varaun Ramgoolie, Nicholas Sammy
%
%  This file is part of Applied-Mathematics-Unit2 and is distributed under the
%  terms of the MIT License. See the LICENSE file for details.
%
%  Description:
%     Year: 2011
%     Module: 2
%     Question: 3 
%------------------------------------------------------------------------------

%------------------------------------------------------------------------------
% 3 a
%------------------------------------------------------------------------------

\begin{subquestions}
	
\subquestion

We should make some observations about the word MAXIMUM. There are 5 letters which make up the word, \{A, I , M, U, X\}, and all of the letters occur exactly once, except for M which is present 3 times in the word. The number of letters in the word, $n$, is equal to 7.
	
\begin{subsubquestions}
	
\subsubquestion

From \rdef{mod2:defn:PermutationEqn2}, we can see that,
\begin{align}
	\text{Number of Arrangements} & = \frac{n!}{3!} \nn \\
	                              & = \frac{7!}{3!} \nn \\
	                              & = 7 \times 6 \times 5 \times 4 \nn \\
	                              & = 840 \,.
\end{align}	
	
We should note that the $3!$ comes about because there are 3 identical M's in MAXIMUM.

%------------------------------------------------------------------------------

\subsubquestion

Let us first remove 1 arbitrary vowel from the letters.
We now have a group of 6 letters (3 M's, 1 X, and 2 vowels) and 1 single vowel that we have removed. From \rdef{mod2:defn:MultiplicationRule},
\begin{align}
	\hspace{-50pt}
	\text{$\#$ of arrangements ending in a vowel} = \text{$\#$ of 6 letter arrangements} \times \text{$\#$ of ways to remove 1 vowel}. \label{2011:q3:NoVowel}
\end{align}

Since there are 3 vowels, there are only 3 ways to remove 1 vowel from MAXIMUM. Moving forward,
\begin{align}
	\text{Number of Arrangements of \{3 M's, 1 X and 2 vowels\}} & = \frac{n!}{3!} \nn \\
																 & = \frac{6!}{3!} \nn \\
																 & = 4 \times 5 \times 6 \nn \\
																 & = 120 \,.
\end{align}
 
Thus, from \req{2011:q3:NoVowel},
\begin{align}
	\text{$\#$ of arrangements ending in a vowel} & = 120 \times 3 \nn \\
	                                            & = 360 \,.
\end{align}

%------------------------------------------------------------------------------

\subsubquestion

It should be noted that the three M's should be treated as a single item. Now, there are only 5 items to sort \{A, I, U, X, and 1 set of 3 M's\}. Thus, from \rdef{mod2:defn:PermutationEqn},
\begin{align}
	^{5}P_5 & = \frac{5!}{(5-5)!} \nn \\
	        & = \frac{5!}{0!} \nn \\
	        & = 5! \nn \\
	        & = 120 \,.
\end{align}

\end{subsubquestions}

%------------------------------------------------------------------------------
% 3 b
%------------------------------------------------------------------------------

\subquestion

We should notice that a committee can only have more that 3 women in it if ALL 4 women are chosen. Therefore, there is only 1 possible committee with all 4 women. 

Thus, we can find the total number of committees that can be formed among all the persons, and minus the 1 committee with 4 women, in order to find the number of committees with AT MOST 3 women. Thus,
\begin{equation}
  	\text{Number of committees with AT MOST 3 Women} = \text{Total number of committees} - 1 \,.
\end{equation}

Since every possible person out of the 10 is distinct, we can use \rdef{mod2:defn:CombinationEqn} as follows,
\begin{align}
	\text{Number of Committees} & = ~ ^{10}C_5 \nn \\
	                            & = \frac{10!}{5!\times (10-5)!} \nn \\
	                            & = \frac{10!}{5! \times 5!} \nn \\
	                            & = 252 \,.
\end{align}

Thus,
\begin{equation}
	\text{Number of committees with at most 3 Women} = 252 - 1 = 251 \,.
\end{equation}

%------------------------------------------------------------------------------
% 3 c
%------------------------------------------------------------------------------

\subquestion

We define $X$ as the discrete random variable which represents the number attempts needed to draw a green ball.

\begin{subsubquestions}
	
\subsubquestion

From Section ~\ref{mod2:section:Geometric}, we can see that
\begin{equation}
	X \sim \text{Geo} \left(\frac{3}{9}\right) \text{or} ~ X \sim \text{Geo} \left(\frac{1}{3}\right) \,.
\end{equation}

%------------------------------------------------------------------------------

\subsubquestion

\begin{subsubsubquestions}
	
\subsubsubquestion

From \rdef{mod2:defn:Geomtric}, we know that,
\begin{align}
	P(X = x) & = \frac{1}{3} \times \left(1-\frac{1}{3} \right)^{x-1} \nn \\
	         & = \frac{1}{3} \times \left(\frac{2}{3} \right)^{x-1} \,.
\end{align}	
	
Thus, we can find $P(X=2)$ as follows,
\begin{align}
	P(X=2) & = \frac{1}{3} \times \left(\frac{2}{3} \right)^{2-1} \nn \\
	       & = \frac{1}{3} \times \frac{2}{3}  \nn \\
	       & = \frac{2}{9} \,.
\end{align}

%------------------------------------------------------------------------------

\subsubsubquestion

From \rdef{mod2:eq:Geometric:Prop},
\begin{align}
	P(X > 2) & = (1-p)^2 \nn \\
	            & = \left(\frac{2}{3}\right)^2 \nn \\
	            & = \frac{4}{9} \,.
\end{align}

%------------------------------------------------------------------------------

\subsubsubquestion

Since $X$ is discrete, we know that,
\begin{equation}
	P(X < 4) = 1 - P(X > 3) \,.
\end{equation}

Using \rdef{mod2:eq:Geometric:Prop},
\begin{align}
	P(X > 3) & = (1-p)^3 \nn \\
		   	 & = \left(\frac{2}{3}\right)^3 \nn \\
			 & = \frac{8}{27} \,.
\end{align}

Thus, $P(X<4)$ can be calculated as follows,
\begin{align}
	P(X < 4) & = 1 - P(X > 3) \nn \\
	         & = 1 - \frac{8}{27} \nn \\
	         & = \frac{19}{27} \,.
\end{align}

\end{subsubsubquestions}

%------------------------------------------------------------------------------
% 3 d
%------------------------------------------------------------------------------

\subsubquestion

From \req{mod2:eq:Geometric:Mean},
\begin{align}
	E(X) & = \frac{1}{p} \nn \\
	     & = \frac{1}{\frac{1}{3}} \nn \\
	     & = 3 \,.
\end{align}


\end{subsubquestions}

\end{subquestions}
