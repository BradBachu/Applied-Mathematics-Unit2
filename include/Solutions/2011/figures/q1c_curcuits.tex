%------------------------------------------------------------------------------
% Author(s):
%  Brad Bachu
%
% Copyright:
%  Copyright (C) 2020 Brad Bachu, Arjun Mohammed, Nicholas Sammy, Kerry Singh
%
%  This file is part of Applied-Mathematics-Unit2 and is distributed under the
%  terms of the MIT License. See the LICENSE file for details.
%
%  Description: TikZ switching diagram 
%     Year: 2011 
%     Module: 1
%     Question: 1 
%------------------------------------------------------------------------------

\documentclass[crop,tikz]{standalone}

\usepackage{../../../../src/tikzappmath}

\usetikzlibrary{scopes}
\usetikzlibrary{arrows,shapes.gates.logic.US,shapes.gates.logic.IEC,calc}

\begin{document}

   \begin{tikzpicture}

      % make these defs here so that I can change all after
      % note that you do not have to keep them consistent
      % I just defined this to make initial coding easier
      \def\xspace{3} % set the horizonal distance between gates
      \def\yspace{2} % set the vertical distance between gates
      \def\shift{.7cm} % sets some spacing to draw connecting lines

      % initial node
      \node at (0,0) [circuitdot] (in) {}; 
      % draw the top wires
      \draw (in)--++ (\shift,0) node[above] (t1){}|-++ (\shift,\shift) node[circuitdot] {} ++(.5*\shift,0) node (a1) {$a$} ++(0.5*\shift,0)  node[circuitdot]{} --++(1.5*\shift,0) node[circuitdot]{} ++(.5*\shift,0) node(b){$b$} ++(.5*\shift,0) node[circuitdot]{}  -|++(1.5*\shift,-\shift) --+(\shift,0)node[circuitdot]{} ;
      % \draw the bottom wires
      \draw (t1)|-++ (\shift,-\shift) node[circuitdot] {} ++(.5*\shift,0) node (a2) {$a$} ++(0.5*\shift,0)  node[circuitdot]{} --++ (\shift,0) node[above] (t2) {} |-++(0.5*\shift,\shift) node[circuitdot]{} ++ (0.5*\shift,0) node(nb) {$\lnot b$} ++ (.5*\shift,0) node[circuitdot]{}-|++(.5*\shift,-\shift) node(t3){} -|++(\shift,\shift) node(t4){};
      \draw (t2) |-++ (0.5*\shift,-\shift)node[circuitdot]{} ++ (0.5*\shift,0) node(c) {$c$} ++ (.5*\shift,0) node[circuitdot]{}-|++(.5*\shift,\shift);

      % -|++(\shift,-\shift)  --++ (\shift,0) |-++(\shift,\shift) node[circuitdot] {} ++ (.5*\shift,0) node (np) {$\lnot p$} ++ (.5*\shift,0) node[circuitdot] {} -|++ (\shift,-\shift) --++ (\shift,0) node[circuitdot] (end) {} ;

   \end{tikzpicture}
   
\end{document} 