%------------------------------------------------------------------------------
% Author(s):
% Varaun Ramgoolie
% Copyright:
%  Copyright (C) 2020 Brad Bachu, Arjun Mohammed, Varaun Ramgoolie, Nicholas Sammy
%
%  This file is part of Applied-Mathematics-Unit2 and is distributed under the
%  terms of the MIT License. See the LICENSE file for details.
%
%  Description:
%     Year: 2009
%     Module: 2
%     Question: 3 
%------------------------------------------------------------------------------

%------------------------------------------------------------------------------
% 3 a
%------------------------------------------------------------------------------

\begin{subquestions}
	
\subquestion

We are given 7 distinct letters.

\begin{subsubquestions}

\subsubquestion

From \rdef{mod2:defn:PermutationEqn}, we can find that the number of total arrangements using all 7 letters is,
\begin{align}
	^7P_7 & = \frac{7!}{(7-7)!} \nn \\
	      & = \frac{7!}{(0)!} \nn \\
	      & = 7! \nn \\
	      & = 5040 \,.
\end{align}
	
%------------------------------------------------------------------------------

\subsubquestion

There are 4 consonants and 3 vowels. In order to find the number or arrangements of interchanging consonants and vowels, we can form a line denoted with $C_x$ and $V_y$, where C and V represent consonants and vowels respectively, and $x$ and $y$ represent the number of vowels and consonants remaining as follows,
\begin{equation}
	C_4:V_3:C_3:V_2:C_2:V_1:C_1 \,.
\end{equation}

Thus, from \rdef{mod2:defn:MultiplicationRule}, we can find the number of arrangements with alternating consonants and vowels as follows,
\begin{align}
	\text{Arrangements with alternating pattern} & = 4 \times 3\times 3\times 2\times 2\times 1\times 1 \nn \\
	                                             & = 144 \,.
\end{align}

Thus, the probability that an alternating arrangemnet is chosen at random is,
\begin{equation}
	P(\text{Alternating Sequence}) = \frac{144}{5040} = \frac{1}{35} \,.
\end{equation}

\end{subsubquestions}	
	
%------------------------------------------------------------------------------
% 3 b
%------------------------------------------------------------------------------

\subquestion

\begin{subsubquestions}
	
\subsubquestion

Let $L$ be the event that a taxi arrives late. Let $X$ be the event of calling a taxi. We want to find
\begin{equation}
	P(L)\,.
\end{equation}
We are given that
\begin{align}
	P(X=F) &= 0.3 \nn\\ 
	P(X=G) &= 0.25 \nn\\
	P(X=T) &= 0.45 
\end{align}
and
\begin{align}
	P(L|X=F) &= 0.03 \nn\\
	P(L|X=G) &= 0.05\nn\\
	P(L|X=T) &= 0.10
\end{align}

As a taxi must be called, the probability that a taxi arrives late is,
\begin{equation}
	P(L) = P({(X=F) \land L}) + P((X=G) \land L) + P({(X=T) \land L}) \,.
\end{equation}
	
Thus, we can find $P({L})$ as follows,
\begin{align}
	P({(X=F) \land L}) & = 0.3 \times 0.03 \nn \\
	                       & = 0.009 \,. \\ \nn 
\end{align}
\begin{align}
	P({(X=G) \land L}) & = 0.25 \times 0.05 \nn \\
	 					   & = 0.0125 \, \\ \nn 
\end{align}
\begin{align}
	P({(X=T) \land L}) & = 0.45 \times 0.1 \nn \\
	 					   & = 0.045 \,.  \\ \nn    
\end{align}
\begin{align}	 					   
	P(L) & = P({(X=F) \land L}) + P({(X=G) \land L}) + P({(X=T) \land L}) \nn \\
	                    & = 0.009+0.0125+0.045 \nn \\
	                    & = 0.0665 = 6.65\% \,.					              
\end{align}

Alternatively, we know that,
\begin{align}
	P(L)&= \sum_{X} P(L|X) P(X)\,.
\end{align}

Thus, we can compute $P(L)$ as,
\begin{align}
	P(L)&= P(L|X=F) P(X=F) + P(L|X=G) P(X=G)+ P(L|X=T) P(X=T) \nn\\
	&=0.3 \times 0.03 +  0.25 \times 0.05 + 0.45 \times 0.1 \nn\\
	&=  0.0665 = 6.65\% \,.		
\end{align}

%------------------------------------------------------------------------------

\subsubquestion

We want to compute $P(X=T| L)$. From \rdef{mod2:defn:Conditional}, we have,
\begin{align}
	P(X=T| L) & = \frac{P(T \land L)}{P(L)} \nn \\
	                      & = \frac{0.045}{0.0665} \nn \\
	                      & = \frac{90}{133} \nn \\ 
	                      & \approx 67.7\%
\end{align}

\end{subsubquestions}

%------------------------------------------------------------------------------
% 3 c
%------------------------------------------------------------------------------

\subquestion

The probability density function, $f$, of a continuous random variable is given,
\begin{subsubquestions}
	
\subsubquestion

We want to compute $P(X<1)$. Note that,
\begin{equation}
	P(1<X) = P(1<X<\infty) \,.
\end{equation}

From Note ~\ref{mod2:note:ContinuousRV:Note1},
\begin{align}
	 P(1<X<\infty) & = \int_{1}^{\infty}f(x)\,\dd x
\end{align}

Next, we must split up the integral over the different intervals $f$ is defined,
\begin{align}
P(1<X) & = \int_{1}^{2}f(x)\,\dd x+\int_{2}^{\infty}f(x)\,\dd x \,.
\end{align}

Substituting, 
\begin{align}
P(1<X) & = \int_1^2 \left(\frac{3}{8}(4-4x+x^2) \right)\dd x + 0\nn \\
	    & = \frac{3}{8} \times \left[4x-2x^2+\frac{x^3}{3} \right]_1^2 \nn \\
	    & = \frac{3}{8} \times \left(\left[4(2)-2(2)^2+\frac{(2)^3}{3} \right]-\left[4(1)-2(1)^2+\frac{(1)^3}{3} \right] \right) \nn \\
	    & = \frac{3}{8} \times \left( \frac{8}{3} - \frac{7}{3}\right) \nn \\
	    & = \frac{3}{8} \times \frac{1}{3} \nn \\
	    & = \frac{1}{8} \,.	
\end{align}

%------------------------------------------------------------------------------
\subsubquestion

Using \rdef{mod2:defn:ContinuousRV:Expectation}, $E(X)$ is given by,
\begin{align}
	E(X) & = \int_{-\infty}^{\infty}x f(x)\,\dd x \,.
\end{align}

Again, we must split up the integral over the different regions,
\begin{align}
	E(X) & = \int_{-\infty}^{0}x f(x)\mathrm{d}x + \int_{0}^{2}x f(x)\mathrm{d}x + \int_{2}^{\infty}x f(x)\,\dd x\,,
\end{align}
before we can substitute and calculate,
\begin{align}
	E(X) & = 0 + \int_{0}^{2}x f(x)\,\dd x + 0 \nn \\
	     & = \int_{0}^{2} \left(x \times \frac{3}{8}(4-4x+x^2)\right)\,\dd x \nn \\
	     & = \frac{3}{8} \times \int_{0}^{2} \left(4x-4x^2+x^3\right)\,\dd x \nn \\
	     & = \frac{3}{8} \times \left[ 2x^2 - \frac{4x^3}{3} + \frac{x^4}{4}\right]_0^2 \nn \\
	     & = \frac{3}{8} \times \frac{4}{3} \nn \\
	     & = \frac{1}{2} \,. \\
\end{align}

Similarly, using \rdef{mod2:defn:ContinuousRV:Variance}, we can find $\var(X)$ as,
\begin{equation}
  \var(X) = E(X^2)- E(X)^2 \,.
\end{equation}

We proceed by finding $E(X^2)$ as,
\begin{align}
  E(X^2) & = \int_{-\infty}^{\infty}x^2 f(x)\,\dd x \,.
\end{align}

Again, we must split up the integral over the different regions,
\begin{align}
  E(X^2) & = \int_{-\infty}^{0}x^2 f(x)\,\dd x+\int_{0}^{2}x^2 f(x)\,\dd x+\int_{2}^{\infty}x^2 f(x)\,\dd x \,,
\end{align}
before we can substitute and calculate,
\begin{align}
  E(X^2) & = 0 + \int_{0}^{2} \left(x^2 \times \frac{3}{8}(4-4x+x^2)\right)\mathrm{d}x + 0 \nn \\
  		 & = \frac{3}{8} \times \int_{0}^{2} \left(4x^2-4x^3+x^4\right)\mathrm{d}x  \nn \\
  		 & = \frac{3}{8} \times \left[ \frac{4x^3}{3} - x^4 + \frac{x^5}{5}\right]_0^2  \nn \\
  		 & = \frac{3}{8} \times \frac{16}{15}\nn \\
  		 & = \frac{2}{5}\,.
\end{align}
  
Hence, we can compute $\var(X)$ by substituting as follows,
\begin{align}
  		\var(X) & = E(X^2)- E(X)^2 \nn \\
         & = \frac{2}{5} - \left(\frac{1}{2}\right)^2 \nn \\
         & = \frac{2}{5} - \frac{1}{4} \nn \\
         & = \frac{3}{20} \,.
\end{align}

%------------------------------------------------------------------------------

\subsubquestion

From \rdef{mod2:defn:ContinuousRV:CDF}, we can compute the cumulative density function, $F(a)$, which is defined as $P(X<a)$, as follows,
\begin{align}
	F(a) & = \int_{-\infty}^{a} f(x)\,\dd x \,.
\end{align}

Since $F(q) = \frac{1}{4}$, we know $0\leq q \leq 2$. Hence, we must split up the integral as
\begin{align}
	F(a) & = \int_{-\infty}^{0} f(x)\,\dd x + \int_{0}^{a} f(x)\,\dd x \,,
\end{align}
before we can substitute and evaluate,
\begin{align}
	F(a) & = 0 + \int_{0}^{a} f(x)\mathrm{d}x \nn \\
	     & = \frac{3}{8} \times \int_{0}^{a} (2-x)^2\mathrm{d}x \nn \\
	     & = \frac{3}{8} \times \left[\left(\frac{(2-x)^3}{3} \times -1 \right) \right]_0^a \nn \\
	     & = \frac{3}{8} \times \left(\frac{-(2-(a))^3}{3} - \frac{-(2-(0))^3}{3}\right) \nn \\
	     & = \frac{3}{8} \times \left(\frac{-(2-a)^3}{3} - \frac{-8}{3}\right) \nn \\
	     & = \frac{-(2-a)^3 + 8}{8} \nn \\
	     & = 1 - \frac{(2-a)^3}{8}
\end{align}

Now, we can find $q$ using the fact that
\begin{equation}
	F(q) = \frac{1}{4}\,.
\end{equation}

Hence
\begin{align}
	1 - \frac{(2-q)^3}{8} & = \frac{1}{4} \nn \\
	            8 - (2-q)^3 & = 2 \nn \\
	                (2-q)^3 & = 6 \nn \\
	                2-q & = \sqrt[3]{6} \nn \\
	                \implies q & = 2 - \sqrt[3]{6} \nn \\
	                & \approx 0.183 \,.
\end{align}

\end{subsubquestions}

\end{subquestions}



