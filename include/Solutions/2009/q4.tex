%------------------------------------------------------------------------------
% Author(s):
% Varaun Ramgoolie
% Copyright:
%  Copyright (C) 2020 Brad Bachu, Arjun Mohammed, Varaun Ramgoolie, Nicholas Sammy
%
%  This file is part of Applied-Mathematics-Unit2 and is distributed under the
%  terms of the MIT License. See the LICENSE file for details.
%
%  Description:
%     Year: 2009
%     Module: 2
%     Question: 4
%------------------------------------------------------------------------------

%------------------------------------------------------------------------------
% 4 a
%------------------------------------------------------------------------------

\begin{subquestions}
	
\subquestion

We are given a scenario which illustrates a Binomial distribution with parameters, $p = 0.65$ and $n=12$. Let $X$ be the discrete random variable which represents the number of people that pass the test. We can express this as,
\begin{equation}
	X \sim \text{Bin}(12,0.65) \,.
\end{equation}

\begin{subsubquestions}
	
\subsubquestion

\begin{subsubsubquestions}
	
\subsubsubquestion

From, \rdef{mod2:defn:Binomial}, we know that,
\begin{equation}
	P(X = x) = { 12 \choose x} (0.65)^x (1-0.65)^{12-x} \,. \label{2009:q4:BinEqn1}
\end{equation}
	
Thus, we get that,
\begin{align}
	P(X=8) & = { 12 \choose 8} (0.65)^8 \times (1-0.65)^{12-8} \nn \\
		   & = { 12 \choose 8} (0.65)^8 \times (0.35)^{4} \nn \\
		   & = 495 \times (0.65)^8 \times (0.35)^{4} \nn \\
		   & = 0.237 \,.
\end{align}

%------------------------------------------------------------------------------

\subsubsubquestion

Since $X$ is discrete, we notice that,
\begin{equation}
	P(8\leq X \leq 10) = P(8)+P(9)+P(10) \,.
\end{equation}

Therefore, using \req{2009:q4:BinEqn1}, we find that,
\begin{align}
	P(9) & = { 12 \choose 9} (0.65)^9 \times (1-0.65)^{12-9} \nn \\
	     & = { 12 \choose 9} (0.65)^9 \times (0.35)^{3} \nn \\
	     & = 220 \times (0.65)^9 \times (0.35)^{3} \nn \\
	     & = 0.195 \,. \\ \nn \\
	P(10) & = { 12 \choose 10} (0.65)^8 \times (1-0.65)^{12-10} \nn \\
         & = { 12 \choose 10} (0.65)^8 \times (0.35)^{2} \nn \\
	     & = 66 \times (0.65)^{10} \times (0.35)^{2} \nn \\
	     & = 0.109 \,. \\ \nn \\
	P(8\leq X \leq 10) & = P(8)+P(9)+P(10) \nn \\
	                   & = 0.237+0.195+0.109 \nn \\
	                   & = 0.541 \,.   
\end{align}

\end{subsubsubquestions}

%------------------------------------------------------------------------------

\subsubquestion

Using \req{mod2:eq:Binomial:Mean}, we find that,
\begin{align}
	E(X) & = np \nn \\
	     & = 12 \times 0.65 \nn \\
	     & = 7.8 \,.
\end{align}

Similarly, using \req{mod2:eq:Binomial:Variance}, we can find that,
\begin{align}
	\text{Standard Deviation} & = \sqrt{\text{Variance}} \nn \\
	                          & = \sqrt{np(1-p)} \nn \\
	                          & = \sqrt{12\times0.65\times0.35} \nn \\
	                          & = \sqrt{2.73} \nn \\
	                          & = 1.65 \,.
\end{align}

%------------------------------------------------------------------------------

\subsubquestion

We will let $Y$ be the discrete random variable which represents the number of days in which at least 8 but no more than 10 persons pass the test, over a 5 day period. We get that,
\begin{equation}
	Y \sim \text{Bin}(5,0.541) \,.
\end{equation}

From, \rdef{mod2:defn:Binomial}, we know that,
\begin{equation}
	P(Y = y) = { 5 \choose y} (0.541)^y (1-0.541)^{5-y} \,. \label{2009:q4:BinEqn2}
\end{equation}

Using \req{2009:q4:BinEqn2}, we can find that,
\begin{align}
	P(Y=3) & = { 5 \choose 3} (0.541)^3 (1-0.541)^{5-3} \nn \\
	       & = 10 \times (0.541)^3 \times (0.459)^{2} \nn \\
	       & = 0.334 \,.
\end{align}

\end{subsubquestions}

%------------------------------------------------------------------------------
% 4 b
%------------------------------------------------------------------------------	
	
\subquestion

We will let $X$ represent the number of accidents which occur in a week. We can see that $X$ follows a Poisson distribution with parameter, $\lambda =2$. This can be given as,
\begin{equation}
	X \sim \text{Pois}(2) \,.
\end{equation}

From \rdef{mod2:defn:Poisson}, we know that,
\begin{equation}
	P(X = x) =\frac{2^x \times e^{-2}}{x!} \,. \label{2009:q4:PoisEqn1}
\end{equation}

Since $X$ is discrete, we know that,
\begin{equation}
	P(X \geq 1) = 1 - P(X=0) \,.
\end{equation}

Using \req{2009:q4:PoisEqn1} we can find that,
\begin{align}
	P(X=0) & = \frac{2^0 \times e^{-2}}{0!} \nn \\
	     & = \frac{1 \times e^{-2}}{1} \nn \\
	     & = e^{-2} \,. \\ \nn \\ 
	 P(X \geq 1) & = 1 - P(X=0) \nn \\
	             & = 1 - e^{-2} \nn \\
	             & = 0.865 \,.  
\end{align}

%------------------------------------------------------------------------------
% 4 c
%------------------------------------------------------------------------------

\subquestion 

Let $X$ represent the number of people that have a certain disease. We see that $X$ follows a Binomial distribution with parameters, $p=0.04$ and $n=100$. From \rdef{mod2:defn:PoissonApproxBinomial}, we can notice that,
\begin{align}
	& n=100>30 \,, \nn \\
	\text{and} ~ & np=100 \times 0.04 = 4<5 \,. \nn 
\end{align}

Thus, we can see that $X$ can be approximated by a Poisson distribution with parameter, $\lambda=np=4$. We get that,
\begin{equation}
	X \sim \text{Pois}(4) \,.
\end{equation}

From \rdef{mod2:defn:Poisson}, we get that,
\begin{equation}
	P(X = x) =\frac{4^x \times e^{-4}}{x!} \,. \label{2009:q4:PoisEqn2}
\end{equation}

Since $X$ is discrete, we get that,
\begin{equation}
	P(X \leq 3) = P(X=0)+P(X=1)+P(X=2)+P(X=3) \,.
\end{equation}

Therefore, using \req{2009:q4:PoisEqn2}, we get that,
\begin{align}
	P(X=0) & = \frac{4^0 \times e^{-4}}{0!} \nn \\
	       & = \frac{1 \times e^{-4}}{1} \nn \\
	       & = e^{-4} \\ \nn \\
	P(X=1) & = \frac{4^1 \times e^{-4}}{1!} \nn \\
	       & = \frac{4 \times e^{-4}}{1} \nn \\
	       & = 4e^{-4} \\ \nn \\
	P(X=2) & = \frac{4^2 \times e^{-4}}{2!} \nn \\
	       & = \frac{16 \times e^{-4}}{2} \nn \\
	       & = 8e^{-4} \\ \nn \\
	P(X=3) & = \frac{4^3 \times e^{-4}}{3!} \nn \\
	       & = \frac{64 \times e^{-4}}{6} \nn \\
	       & = \frac{32e^{-4}}{3} \\ \nn \\
	P(X \leq 3) & = P(X=0)+P(X=1)+P(X=2)+P(X=3) \nn \\
	            & = e^{-4} \times \left(1+4+8+\frac{32}{3}\right) \nn \\
	            & = \frac{71e^{-4}}{3} \nn \\
	            & = 0.433 \,.
\end{align}


\end{subquestions}