%------------------------------------------------------------------------------
% Author(s):
% Varaun Ramgoolie
% Copyright:
%  Copyright (C) 2020 Brad Bachu, Arjun Mohammed, Nicholas Sammy, Kerry Singh
%
%  This file is part of Applied-Mathematics-Unit2 and is distributed under the
%  terms of the MIT License. See the LICENSE file for details.
%
%  Description:
%     Year: 2016
%     Module: 1
%     Question: 2 
%------------------------------------------------------------------------------
\usetikzlibrary{patterns}

\begin{subquestions}
	
%------------------------------------------------------------------------------
% 2 a--------------------------------------------------------------------------
%------------------------------------------------------------------------------

\subquestion

We must maximize the expression, $P=6x+4y$, given the following inequalities:
\begin{align}
	4x+6y & \leq 48\,, \nn \\
	4x+2y & \leq 32\,, \nn \\
	0 \leq & ~ y \leq 5\,, \nn \\
	x & \geq 0\,.
\end{align}

\begin{subsubquestions}

%------------------------------------------------------------------------------
% 2 a i------------------------------------------------------------------------
%------------------------------------------------------------------------------
	
\subsubquestion

We can draw the inequalities for the linear programming question below.
\begin{figure}[H]
	\begin{center}
		\includegraphics{../2016/figures/2016q2LinGraph}
		\caption{\label{2016:q2:graph:Graph1} Linear Programming Graph.}
	\end{center}
\end{figure}


%------------------------------------------------------------------------------
% 2 a ii------------------------------------------------------------------------
%------------------------------------------------------------------------------

\subsubquestion

The feasible region is shaded in Graph ~\ref{2016:q2:graph:Graph1}.

%------------------------------------------------------------------------------
% 2 a iii----------------------------------------------------------------------
%------------------------------------------------------------------------------

\subsubquestion

From ~\rdef{mod1:defn:TourOfVertices}, we can calculate $P$ by performaing a tour of the vertices,

\begin{table}[H]
	\centering
	\begin{tabular}{|c|c|}
		\hline
		Vertice & $P = 6x + 4y$ \\
		\hline
		(0, 0) & 0 \\
		(0, 5) & 20 \\
		(4.5, 5) & 47 \\
		(6, 4) & 52 \\
		(8, 0) & 48 \\
		\hline
	\end{tabular}
	\caption{\label{2016:q2:tab:P} Tour of Vertices}
\end{table}

From \rtab{2016:q2:tab:P}, we see that the maximum value of $P$ is 52, when $x=6$ and $y=4$.

\end{subsubquestions}

%------------------------------------------------------------------------------
% 2 b--------------------------------------------------------------------------
%------------------------------------------------------------------------------

\subquestion

\begin{subsubquestions}
	
\subsubquestion

We wish to assign four drivers to take passengers to four towns in order to minimise travel time. We can use the Hungarian algorithm on the data given.
\begin{table}[H]
	\begin{minipage}{0.3\textwidth}
		\centering
		\begin{tabular}{cccc}
			23 & 31 & 25 & 28 \\
			30 & 20 & 33 & 29 \\
			20 & 25 & 28 & 25 \\
			35 & 19 & 20 & 30 \\
		\end{tabular}
		\captionsetup{width=1.1\linewidth}
		\caption*{Matrix From question}
	\end{minipage}
	\hspace{20pt}
	%------------------------------------------------------------------------------
	\begin{minipage}{0.3\textwidth}
		\centering
		\begin{tabular}{cccc}
			0 & 8 & 2 & 5 \\
		   10 & 0 & 13 & 9 \\
			0 & 5 & 8 & 5 \\
		   16 & 0 & 1 & 11 \\
		\end{tabular}
		\captionsetup{width=1.1\linewidth}
		\caption*{Matrix after Reducing Rows}
	\end{minipage}
	\hspace{20pt}
	%------------------------------------------------------------------------------
	\begin{minipage}{0.3\textwidth}
		\centering
		\begin{tabular}{cccc}
			0 & 8 & 1 & 0 \\
		   10 & 0 & 12 & 4 \\
			0 & 5 & 7 & 0 \\
		   16 & 0 & 0 & 6 \\
		\end{tabular}
		\captionsetup{width=1.1\linewidth}
		\caption*{Matrix after Reducing Columns} 
	\end{minipage}
	
	\vspace{20pt} 
	%------------------------------------------------------------------------------
	\begin{minipage}{0.3\textwidth}
		\centering
		\begin{tabular} {cccccc}
			&   &        & 							 &   &                       \\ 
   \hhs{h1} & 0 &      8 &                         1 & 0 & \hhe[red]{h1}         \\
   \hhs{h2} &10 &      0 &                        12 & 4 & \hhe[red]{h2}         \\
   \hhs{h3}	& 0 &      5 &                         7 & 0 & \hhe[red]{h3}         \\
   \hhs{h4}	&16 &      0 &                         0 & 6 & \hhe[red]{h4}         \\
			&   &        & 							 &   &                       \\
		\end{tabular}
		\captionsetup{width=1.1\linewidth}
		\caption*{Shading 0's using the least \\ \centering number of lines}
	\end{minipage}
	\caption{\label{2016:q2:tab:HungAlgo} Showing the steps of the Hungarian Algorithm.}
\end{table}

From \rtab{2016:q2:tab:HungAlgo}, the possible matchings of the passengers and drivers are,
\begin{align}
	& A \rightarrow R,P \,, \nn \\
	& B \rightarrow H \,, \nn \\
	& C \rightarrow R,P \,, \nn \\
	& D \rightarrow H,S \,.
\end{align} 

Thus, the matchings that minimize the total travel time is,
\begin{align}
	& A \rightarrow R ~\text{or} ~P \,, \nn \\
	& B \rightarrow H \,, \nn \\
	& C \rightarrow R ~\text{or} ~P \,, \nn \\
	& D \rightarrow S \,.
\end{align} 

%------------------------------------------------------------------------------

\subsubquestion

Therefore, the minimum total travel time is,
\begin{equation}
	\text{Minimum Time} = 23+20+25+20 = 88\, \text{minutes}\,.
\end{equation}


\end{subsubquestions}

\end{subquestions}

