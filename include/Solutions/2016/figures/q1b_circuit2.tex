\documentclass[crop,tikz]{standalone}

\usepackage{../../../../src/tikzappmath}

\usetikzlibrary{scopes}
\usetikzlibrary{arrows,shapes.gates.logic.US,shapes.gates.logic.IEC,calc}

\begin{document}

   \begin{tikzpicture}

      % make these defs here so that I can change all after
      % note that you do not have to keep them consistent
      % I just defined this to make initial coding easier
      \def\xspace{3} % set the horizonal distance between gates
      \def\yspace{2} % set the vertical distance between gates
      \def\shift{.8cm} % sets some spacing to draw connecting lines

      % initial node
      \node at (0,0) [circuitdot] (in) {}; 
      % draw the top wires
      \draw (in)--++ (\shift,0) {}|-++ (\shift,\shift) node[circuitdot] {} ++(.5*\shift,0) node (p) {$p$} ++(0.5*\shift,0)  node[circuitdot]{} -|++(\shift,-\shift)  --++ (\shift,0) |-++(\shift,\shift) node[circuitdot] {} ++ (.5*\shift,0) node (np) {$\lnot p$} ++ (.5*\shift,0) node[circuitdot] {} -|++ (\shift,-\shift) --++ (\shift,0) node[circuitdot] (end) {} ;
      % now draw bottom half
      \draw (in)--++ (\shift,0) {}|-++ (\shift,-\shift) node[circuitdot] {} ++(.5*\shift,0) node (q) {$q$} ++(0.5*\shift,0)  node[circuitdot]{} -|++(\shift,\shift)  --++ (\shift,0) |-++(\shift,-\shift) node[circuitdot] {} ++ (.5*\shift,0) node (nq) {$\lnot q$} ++ (.5*\shift,0) node[circuitdot] {} -|++ (\shift,\shift) --++ (\shift,0) node[circuitdot] (end) {} ;
   \end{tikzpicture}

\end{document} 