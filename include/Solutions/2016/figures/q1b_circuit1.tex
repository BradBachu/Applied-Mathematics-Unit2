\documentclass[crop,tikz]{standalone}

\usepackage{../../../../src/tikzappmath}

\usetikzlibrary{scopes}
\usetikzlibrary{arrows,shapes.gates.logic.US,shapes.gates.logic.IEC,calc}


\begin{document}
   \begin{tikzpicture}

      % make these defs here so that I can change all after
      % note that you do not have to keep them consistent
      % I just defined this to make initial coding easier
      \def\xspace{3} % set the horizonal distance between gates
      \def\yspace{2} % set the vertical distance between gates
      \def\shift{.8cm} % sets some spacing to draw connecting lines

      % initial node
      \node at (0,0) [circuitdot] (in) {}; 
      % from the t draw the top wires
      \draw (in)--++ (\shift,0) {}--++ (0,\shift) --++(\shift,0) node[circuitdot] {} ++(.5*\shift,0) node (p) {$p$} ++(0.5*\shift,0)  node[circuitdot]{} --++(\shift,0) node[circuitdot] {} ++(.5*\shift,0) node (nq) {$\lnot q$} ++(0.5*\shift,0)  node[circuitdot] {} --++(\shift,0)  --++(0,-\shift) node (t2) {} --++ (\shift,0) node[circuitdot] (out) {};
      % from the t draw the bottom wire (just negate the y values)
      \draw (in)--++ (\shift,0) {}--++ (0,-\shift) --++(\shift,0) node[circuitdot] {} ++(.5*\shift,0) node (np) {$\lnot p$} ++(0.5*\shift,0)  node[circuitdot]{} --++(\shift,0) node[circuitdot] {} ++(.5*\shift,0) node (q) {$q$} ++(0.5*\shift,0)  node[circuitdot] {} --++(\shift,0)  --++(0,\shift) node (t2) {} --++ (\shift,0) node[circuitdot] (out) {};

   \end{tikzpicture}

\end{document} 