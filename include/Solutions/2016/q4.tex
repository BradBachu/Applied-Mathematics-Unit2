%------------------------------------------------------------------------------
% Author(s):
% Varaun Ramgoolie
% Copyright:
%  Copyright (C) 2020 Brad Bachu, Arjun Mohammed, Varaun Ramgoolie, Nicholas Sammy
%
%  This file is part of Applied-Mathematics-Unit2 and is distributed under the
%  terms of the MIT License. See the LICENSE file for details.
%
%  Description:
%     Year: 2016
%     Module: 2
%     Question: 4
%------------------------------------------------------------------------------

%------------------------------------------------------------------------------
% 4 a 
%------------------------------------------------------------------------------

\begin{subquestions}
	
\subquestion

We need to perform a $\chi^2$ test at the 5\% level of significance.

\begin{subsubquestions}
	
\subsubquestion

The hypotheses can be given as
\begin{itemize}
	\item $H_0$: ``The data is equally distributed.''
	
	\item $H_1$: ``The data is not equally distributed.''
\end{itemize}

%------------------------------------------------------------------------------

\subsubquestion

A frequency table is given and we want to find whether there is an equal distribution of grades. Therefore, we get that our Expected Value, $E$, for all of the Grades is $20$.

We get,
\begin{table}[H]
	\centering
	\begin{tabular}{|c|c|c|c|}
		\hline
		Grade & Observed,O & Expected,O & $\frac{(O-E)^2}{E}$ \\
		\hline
		A & 14 & 20 & 1.8 \\
		B & 22 & 20 & 0.2 \\
		C & 30 & 20 & 5 \\ 
		D & 18 & 20 & 0.2 \\ 
		F & 16 & 20 & 0.8 \\
		\hline
		Total, $\chi^2_\text{calc.}$ & & & 8 \\
		\hline
	\end{tabular}
	\caption{\label{2016:q4:ChiTable} $\chi^2$ Observed and Expected Values.}
\end{table}
	
Since none of the $\frac{(O-E)^2}{E}$ values is less than 5, we get that the number of degrees of freedom, $\nu$, is,
\begin{equation}
	\nu = 5-1=4 \,.
\end{equation}

Thus, the critical value for this test will be,
\begin{equation}
	\chi^{2}_{0.05} (\nu) = \chi^{2}_{0.05} (4) = 9.488 \,.
\end{equation}

We will therefore reject $H_0$ if,
\begin{equation}
	\chi^{2}_{\text{calc.}} > 9.488 \,. \label{2016:q4:CritReg}
\end{equation}

%------------------------------------------------------------------------------

\subsubquestion

From \rtab{2016:q4:ChiTable}, we found that the test statistic is equal to 8.

%------------------------------------------------------------------------------

\subsubquestion

From \req{2016:q4:CritReg}, we see that $\left( \chi^{2}_{\text{calc.}}=8 \right)$ is not greater than $\left(\chi^{2}_{0.05} (4) = 9.488 \right)$. Thus, we do not reject $H_0$.

\end{subsubquestions}
	
%------------------------------------------------------------------------------
% 4 b
%------------------------------------------------------------------------------
	
\subquestion

We are given the probability density function, $f$, of the continuous random variable $X$.

\begin{subsubquestions}
	
\subsubquestion
We use \rdef{mod2:prop:ContinuousRV:1},
\begin{equation}
	\int_\infty^\infty f(x) \,\dd x = 1\,.
\end{equation}

Splitting up the integral over the different regions,
\begin{align}
	\int_{-\infty}^{\infty} f(x)\,\dd x = \int_{-\infty}^{0} f(x)\,\dd x + \int_{0}^{3} f(x)\mathrm{d}x + \int_{3}^{\infty} f(x)\,\dd x& = 1\,,
\end{align}
we can then substitute and evaluate as,
\begin{align}
\int_\infty^\infty f(x) \,\dd x  &=0+ \int_{0}^{3} (k(4-x))\, \mathrm{d}x +0  \nn \\
	&=\left[ 4kx- \frac{kx^2}{2} \right]^3_0  \nn \\
	&=\left[ 4k(3)- \frac{k(3)^2}{2} \right] - \left[ 4k(0)- \frac{k(0)^2}{2} \right] \nn \\
	&=12k - \frac{9k}{2} \,.
\end{align}
Solving for $k$,
\begin{align}
	k \left(12 - \frac{9}{2}\right) &= 1 \nn \\
	k & = \frac{1}{\left(12 - \frac{9}{2}\right)} \nn \\
	k & = \frac{1}{\left(\frac{15}{2}\right)} \nn \\
	\implies k & = \frac{2}{15} \,.
\end{align}
	
%------------------------------------------------------------------------------

\subsubquestion

Using Note ~\ref{mod2:note:ContinuousRV:Note1}, 
\begin{equation}
	P(X>1) = \int_{1}^\infty f(x) \,\dd x \,.
\end{equation}

Splitting up the integral in the appropriate regions, we have,
\begin{align}
	P(X>1) &= \int_{1}^3 f(x) \,\dd x + \int_{3}^\infty f(x) \,\dd x \,.
\end{align}

Now, substituting $f(x)$ in these regions allows us to evaluate as follows
\begin{align}
	P(1<X)  & =  \int_1^3 \left(\frac{2}{15}(4-x)\right) \,\dd x + 0 \nn \\
	& = \frac{2}{15} \times \int_1^3 \left(4-x\right)\,\dd x \nn \\
	& = \frac{2}{15} \times \left[ 4x- \frac{x^2}{2}\right]^3_1 \nn \\
	& = \frac{2}{15} \times \left( \left[ 4(3)- \frac{(3)^2}{2}\right] -\left[ 4(1)- \frac{(1)^2}{2}\right] \right) \nn \\
	& = \frac{2}{15} \times \left( \left[\frac{(15}{2}\right] -\left[ \frac{7}{2}\right] \right) \nn \\
	& = \frac{2}{15} \times \frac{8}{2} \nn \\
	& = \frac{8}{15} \,.	
\end{align}
	
%------------------------------------------------------------------------------

\subsubquestion

From, \rdef{mod2:defn:ContinuousRV:Expectation}, we note,
\begin{equation}
	E(X) = \int_{-\infty}^{\infty}x f(x)\,\dd x \,.
\end{equation}

Again, we must split up the integral in the appropriate regions,
\begin{align}
	E(X) &= 	\int_{-\infty}^{0}x f(x)\,\dd x +  \int_{0}^{3}x f(x)\,\dd x + \int_{3}^{\infty}x f(x)\,\dd x \,.
\end{align}

Substituting $f(x)$ in the regions, we find that,
\begin{align}
	E(X) & = 0+ \int_{0}^{3} \Big(x \times \left( \frac{2}{15}(4-x)\right) \Big)\,\dd x + 0\nn \\
	                                        & = \frac{2}{15} \times \int_0^3 \left(4x-x^2 \right)\,\dd x \nn \\
	                                        & = \frac{2}{15} \times \left[ 2x^2 - \frac{x^3}{3} \right]_0^3 \nn \\
	                                        & = \frac{2}{15} \times \left( \left[ 2(3)^2 - \frac{(3)^3}{3} \right] - \left[ 2(0)^2 - \frac{(0)^3}{3} \right] \right) \nn \\
	                                        & = \frac{2}{15} \times 9 \nn \\
	                                        & = \frac{18}{15} \,.
\end{align}

%------------------------------------------------------------------------------

\subsubquestion

From Note ~\ref{mod2:note:ContinuousRV:Note1},
\begin{align}
	F(x) = \int_{-\infty}^{x} f(a)\,\dd a
\end{align}

Now, we must consider the three possibilities, that $x\leq0$, $0 < x \leq 3$ and $x>3$.
For $x\leq 0$, we simply have,
\begin{align}
	F(x) &= \int_{-\infty}^{x} f(a)\,\dd a \nn\\
		&= \int_{-\infty}^x0 \,\dd a \nn\\
		&= 0\,.
\end{align}

For $0 < x \leq 3$, we have,
\begin{align}
	F(x) &= \int_{-\infty}^{0} f(a)\,\dd a + \int_0^x f(a) \,\dd a\nn\\
		&= \int_{-\infty}^0 0 \,\dd x + \int_0^x k(4-x) \nn\\
		& = \int_{0}^{x}\left( \frac{2}{15}(4-a)\right)\,\dd a\nn \\
	                                  & = \frac{2}{15} \times \left[4a - \frac{a^2}{2} \right]^x_0 \nn \\
	                                  & = \frac{8x}{15} - \frac{x^2}{15} \,.
\end{align}

Finally, for $x >3$,
\begin{align}
	F(x) &= \int_{-\infty}^{0} f(a)\,\dd a + \int_0^3 f(a) \,\dd a+ \int_3^x f(a) \,\dd a\nn\\
	&= 0 + 1 + 0 \nn\\
	&=1 \,.	
\end{align}


Summarizing,
\begin{equation}
F(x) =
\begin{cases} 
	0, & x \leq 0 \\
	\frac{8x}{15} - \frac{x^2}{15}, & 0 \leq x \leq 3 \\
	1, & x \geq 3 \,.
\end{cases}
\end{equation}
	

%------------------------------------------------------------------------------

\subsubquestion

We can recall from Note ~\ref{mod2:note:ContinuousRV:CDF}, 
\begin{align}
		P( 1.5 < X < 2) & = F(2) - F(1.5) \,.
\end{align}
Substituting for $F$, we can evaluate as,
\begin{align}
		P( 1.5 < X < 2) & = \left(\frac{8(2)}{15} - \frac{(2)^2}{15}\right) - \left(\frac{8(1.5)}{15} - \frac{(1.5)^2}{15}\right) \nn \\
		                & = \left(\frac{16}{15} - \frac{4}{15}\right) - \left(\frac{12}{15} - \frac{2.25}{15}\right) \nn \\
		                & = \left(\frac{12}{15}\right) - \left(\frac{9.75}{15} \right) \nn \\
		                & = \frac{2.25}{15} \nn \\
		                & = 0.15 \,.
\end{align}

\end{subsubquestions}
	
	
\end{subquestions}

