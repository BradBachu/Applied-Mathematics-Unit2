%------------------------------------------------------------------------------
% Author(s):
% Varaun Ramgoolie
% Copyright:
%  Copyright (C) 2020 Brad Bachu, Arjun Mohammed, Varaun Ramgoolie, Nicholas Sammy
%
%  This file is part of Applied-Mathematics-Unit2 and is distributed under the
%  terms of the MIT License. See the LICENSE file for details.
%
%  Description:
%     Year: 2016
%     Module: 2
%     Question: 3
%------------------------------------------------------------------------------

%------------------------------------------------------------------------------
% 3 a
%------------------------------------------------------------------------------

\begin{subquestions}
	
\subquestion
		
\begin{subsubquestions}
	
\subsubquestion

In total, there are 17 paintings available. Thus, using \rdef{mod2:defn:CombinationEqn}, the number of portfolios that can be created is,
\begin{align}
	^{17}C_{12} & = \frac{17!}{(17-12)! \times 12!} \nn \\
	            & = \frac{17!}{5! \times 12!} \nn \\
	            & = 6188 \,.
\end{align}
	
%------------------------------------------------------------------------------
	
\subsubquestion 

In order to find the number of portfolios with 8 water-color and 4 oil paintings, we must first find the number of ways to choose 8 water-color paintings from 10 paintings and the number of ways to choose 4 oil paintings from 7 paintings. \\

For the 8 water-color paintings,
\begin{align}
	^{10}C_{8} & = \frac{10!}{(10-8)! \times 8!} \nn \\
	           & = \frac{10!}{2! \times 8!} \nn \\
	           & = 45 \,.
\end{align}

For the 4 oil paintings,
\begin{align}
	^{7}C_{4} & = \frac{7!}{(7-4)! \times 4!} \nn \\
	          & = \frac{7!}{3! \times 4!} \nn \\
	          & = 35 \,.
\end{align}

Therefore, using \rdef{mod2:defn:MultiplicationRule}, the number of portfolios with 8 water-color \textbf{and} 4 oil paintings is,
\begin{align}
	^{10}C_{8} ~\times ~ ^7C_{4} & = 45 \times 35 \nn \\
	                            & = 1575 \,.
\end{align}

%------------------------------------------------------------------------------

\subsubquestion

Given that the selection of paintings are random, we can see that,
\begin{align}
	P(\text{8 water-color, 4 oil paintings}) & = \frac{\text{Portfolios with 8 water-color, 4 oil paintings}}{\text{Total number of portfolios}} \nn \\
	                                            & = \frac{1575}{6188} \nn \\
	                                            & = \frac{225}{884} \nn \\
	                                            & = 0.255 \,.
\end{align}

\end{subsubquestions}
	
%------------------------------------------------------------------------------
% 3 b
%------------------------------------------------------------------------------

\subquestion

We know that the probability of rolling a 6 on a fair die is $\frac{1}{6}$. Let $X$ be the discrete random variable representing ``the number of rolls a player does before starting a game, up to and including the 6''.

From this, we know that $X$ follows a Geometric Distribution with probability of success, $p=\frac{1}{6}$, as follows
\begin{equation}
	X \sim \text{Geo} \left(\frac{1}{6} \right) \,.
\end{equation}

\begin{subsubquestions}
	
\subquestion

From \rdef{mod2:defn:Geomtric}, the probability mass function, $P(X=x)$ is,
\begin{align}
	P(X = x) & = \frac{1}{6} \times \left(1-\frac{1}{6} \right)^{x-1} \nn \\
	         & = \frac{1}{6} \times \left(\frac{5}{6} \right)^{x-1}
\end{align}
	
Thus, $P(X=3)$ can be calculated as,
\begin{align}
	P(X = 3) & = \frac{1}{6} \times \left(\frac{5}{6} \right)^{3-1} \nn \\
	         & = \frac{1}{6} \times \left(\frac{5}{6} \right)^{2} \nn \\
	         & = \frac{1}{6} \times \left(\frac{25}{36} \right) \nn \\
	         & = \frac{25}{216} \,.
\end{align}

%------------------------------------------------------------------------------

\subsubquestion

The probability that \textbf{at most} 5 rolls are necessary to start the game is the same as $P(X<6)$. Since $X$ is a discrete random variable,
\begin{equation}
	P(X<6) = 1 - P(X>5) \,.
\end{equation}

From \req{mod2:eq:Geometric:Prop}, we know that,
\begin{align}
	P(X>5) & = \left(1 - \frac{1}{6} \right)^5 \nn \\
	       & = \left(\frac{5}{6} \right)^5 \nn \\
	       & = \frac{3125}{7776} \,.
\end{align}

Therefore,
\begin{align}
	P(X<6) & = 1 - P(X>5) \nn \\
	       & = 1 - \frac{3125}{7776} \nn \\
	       & = \frac{4651}{7776} \,.
\end{align}

%------------------------------------------------------------------------------

\subsubquestion

From \rdef{mod2:eq:Geometric:Mean},
\begin{align}
	E(X) & = \frac{1}{p} \nn \\
	     & = \frac{1}{\frac{1}{6}} \nn \\
	     & = 6 \,.
\end{align}

\end{subsubquestions}

%------------------------------------------------------------------------------
% 3 c
%------------------------------------------------------------------------------

\subquestion

Let $X$ be the discrete the random variable representing the number of faulty reports per day. $X$ follows a Poisson Distribution as,
\begin{equation}
	X \sim \text{Pois}(2.5) \,.
\end{equation}

\begin{subsubquestions}
	
\subsubquestion

From \rdef{mod2:defn:Poisson}, we know that the probability mass function, $P(X=x)$ is,
\begin{equation}
	P(X=x) = \frac{2.5^x \times e^{-2.5}}{x!} \,.
\end{equation}

Thus, we can calculate $P(X=4)$ as follows,
\begin{align}
	P(X=4) & = \frac{2.5^4 \times e^{-2.5}}{4!} \nn \\
	       & = \frac{39.065 \times e^{-2.5}}{24} \nn \\
	       & = \frac{39.065}{24 \times e^{2.5}} \nn \\
	       & = 0.134 \,.
\end{align}
	
%------------------------------------------------------------------------------

\subsubquestion

Over a five-day period, the average rate will be 5 times the daily rate ($5 \times 2.5 = 12.5$). If we let $Y$ represent the number of faulty reports per five-day period, we see that $Y$ follows a Poisson Distribution as,
\begin{equation}
	Y \sim \text{Pois}(12.5) \,.
\end{equation}

Similarly, the probability mass function, $P(Y=y)$ is
\begin{equation}
	P(Y=y) = \frac{12.5^y \times e^{-12.5}}{y!} \,.
\end{equation}

Hence, we use this as follows,
\begin{align}
	P(Y=5) & = \frac{12.5^5 \times e^{-12.5}}{5!} \nn \\
	       & = \frac{12.5^5 \times e^{-12.5}}{120} \nn \\
	       & = \frac{12.5^5}{120 \times e^{12.5}} \nn \\
	       & = 0.00948 \,.
\end{align}

\end{subsubquestions}
\end{subquestions}