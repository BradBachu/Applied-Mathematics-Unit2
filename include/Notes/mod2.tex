\section{Counting Principles}\label{mod2:section:CountingPrinciples}

\begin{defn}\label{mod2:defn:MultiplicationRule}
	If event $K_i$ can occur in $p$ ways and event $K_j$ can occur in $q$ ways, the number of ways that $K_i$ \textbf{and} $K_j$ can occur is,
	\begin{equation}
		p \times q.
	\end{equation}
	This can be extended to any number of events. This is known as the \textbf{Multiplication Rule}.
\end{defn}


\begin{defn}\label{mod2:defn:AdditionRule}
	If event $K_i$ can occur in $p$ ways and event $K_j$ can occur in $q$ ways, the number of ways that $K_i$ \textbf{or} $K_j$ can occur is,
	\begin{equation}
		p + q.
	\end{equation}
	This can be extended to any number of events. This is known as the \textbf{Addition Rule}.
\end{defn}


\begin{defn} \label{mod2:defn:Combination}
	A \textbf{combination} is a group of elements chosen from a set of objects. The order of the elements does not matter.
\end{defn}


\begin{defn} \label{mod2:defn:CombinationEqn}
	The number of combinations of $r$ elements from a set of $n$ \textbf{distinct} elements is defined as,
	\begin{equation}
		^nC_r = \frac{n!}{r!~(n-r)!} \,.
	\end{equation}
\end{defn}


\begin{defn}\label{mod2:defn:Permutation}
	A \textbf{permutation} is a group of elements chosen from a set of objects where the order of the elements \textbf{does} matter.	
\end{defn}


\begin{defn}\label{mod2:defn:PermutationEqn}
	The number of permutations of $r$ elements from a set of $n$ \textbf{distinct} elements is defined as,
	\begin{equation}
		^nP_r = \frac{n!}{(n-r)!} \,.
	\end{equation}
\end{defn}


\begin{defn}\label{mod2:defn:PermutationEqn2}
	The number of permutations of $n$ elements, not all distinct, is defined as,
	\begin{equation}
		\frac{n!}{{k_1}!\,{k_2}!\,...\,{k_r}!} \,,
	\end{equation}
	where  $\sum_{i=1}^{r} k_i = n$, and $k_i$ is the number of non-distinct elements of type $i$.
\end{defn}


\begin{defn}\label{mod2:defn:PermutationEqn3}
	The number of permutations of size $r$ out of $n$ elements, with repetition, is defined as,
	\begin{equation}
		n^r
	\end{equation}
\end{defn}




\section{Probability Theory}\label{mod2:section:ProbabilityTheory}


\begin{axiom}  \label{mod2:axiom:ProbabilityAxioms}
	Consider an experiment whose sample space is $S$. The \textbf{probability of the event $E$}, denoted as $P(E)$, is a number that satisfies the following three axioms	
	\begin{enumerate}[label = \bfseries (\arabic*)]
		\item 	$0 \leq P(E) \leq 1 , $ \label{mod2:axiom:ProbabilityAxiom:1}
				
			
		
		\item 	$	P(S) = 1, $ \label{mod2:axiom:ProbabilityAxiom:2}
				
			
		
		\item 	For any sequence of mutually exclusive events $E_1, E_2, ... , E_n$.
		
			
					$P\left( \cup_{i=1}^n E_i \right) = \sum_{i=1}^n P(E_i), $ \label{mod2:axiom:ProbabilityAxiom:3}
			
			
			
				Eg. $P(E_1 \cup E_2) = P(E_1)+P(E_2)$.
		
		
	\end{enumerate}
\end{axiom}


\begin{defn} \label{mod2:defn:Probability}
	Furthermore, we define,
	\begin{equation}
		P(\text{event E occurs}) = \frac{\text{no. of times A can occur}}{\text{total number of outcomes}}. 
	\end{equation}
\end{defn}



\begin{defn} \label{mod2:defn:Complement}
	The probability of the \textbf{complement} of event $A$, denoted as $P(A^c)$ or $P(\bar{A})$ or $P(A^\prime)$, is defined as
	\begin{equation}
		P(\bar{A}) = 1 - P(A).
	\end{equation}
\end{defn}


\begin{prop} \label{mod2:prop:SetPropositions}
Some useful propositions can be easily derived using the above and drawing your own diagrams
\begin{align}
P(A \cup B) &= P(A) + P(B) - P(A \cap B), \label{mod2:eq:SetProp:1}  \\ 
P(A^c \cap B) &= P(B) - P(A \cap B), \label{mod2:eq:SetProp:2} \\ 
P(A^c \cap B^c) &=  P(A \cup B)^c ,\label{mod2:eq:SetProp:3} \\ 
P(A^c \cup B^c) &= P (A \cap B) ^c .\label{mod2:eq:SetProp:4} \ 
\end{align}
\end{prop}


\begin{defn} \label{mod2:defn:MutuallyExclusive}
Two events $A$ and $B$ are said to be \textbf{mutually exclusive} if and only if
\begin{equation}	
	P ( A \cap B) = 0. \label{mod2:eq:MutuallyExclusive:1} \ 
\end{equation}
Applying Eq~[\ref{mod2:eq:SetProp:1}], this implies 
\begin{equation}
	P(A \cup B) = P(A) + P(B).\  \label{mod2:eq:MutuallyExclusive:2}
\end{equation}
\end{defn}


\begin{defn} \label{mod2:defn:Independent}
Two events $A$ and $B$ are said to be \textbf{independent} if and only if
\begin{equation}
P(A \cap B) = P (A) \times P(B). \label{mod2:eq:Independent} \ 
\end{equation}
\end{defn}

\begin{defn} \label{mod2:defn:Conditional}
For events $A$ and $B$, the \textbf{conditional probability} of $A$ given $B$ has occurred, denoted by $P(A|B)$, is defined by
\begin{equation}
P(A|B) = \frac{P(A\cap B)}{P(B)}. \label{mod2:eq:ConditionalProbability} \ 
\end{equation}
\end{defn}




\section{Discrete Random Variables} \label{mod2:section:DiscreteRandomVariables}


\begin{defn} \label{mod2:defn:DiscreteRandomVar}
	A \textbf{discrete random variable}, $X$, can take on at most a countable number of possible values. We define its \textit{probability mass} function, $P(x)$ by
	\begin{equation}
	P(x) =P(X = x).
	\end{equation}
\end{defn}

\begin{prop}\label{mod2:prop:Discrete:Properties}
	If $X$ is a discrete random variable with probability mass function $P(x)$, then we know from Axiom~[\ref{mod2:axiom:ProbabilityAxiom:1}] that
	\begin{equation}
	0 \leq P(X=x) \leq 1 \: \forall x.
	\end{equation}
	and from Axiom~[\ref{mod2:axiom:ProbabilityAxiom:2}] and Axiom~[\ref{mod2:axiom:ProbabilityAxiom:3}] that
	\begin{equation}
	\sum_{\forall i}P(X=x_i) = 1 .
	\end{equation}
\end{prop}

\begin{defn} \label{mod2:defn:Discrete:Expectation}
	If $X$ is a discrete random variable, the \textbf{expectation value}, \textbf{mean} or \textbf{first moment of $X$}, $E[X]$, is defined by
	\begin{equation}
	E[X]= \sum_{\forall i} x_i P(X=x_i). \label{mod2:eq:Discrete:Expectation}\ 
	\end{equation}
\end{defn}

\begin{defn} \label{mod2:defn:Discrete:SecondMoment}
	If $X$ is a discrete random variable, the \textbf{second moment of $X$} is defined by
	\begin{equation}
	E[X^2] = \sum_{\forall i} x_i^2 P(X=x_i). \label{mod2:eq:Discrete:SecondMoment} \ 
	\end{equation}
\end{defn}

\begin{defn}\label{mod2:defn:Discrete:Variance}
	If $X$ is a random variable with mean $\mu$, then the \textbf{variance}, $Var[X]$, is defined by
	\begin{align}
	Var[X] &= E[(X-\mu)^2],  \label{mod2:eq:Discrete:Variance:1}  \\
	&= E[X^2] - (E[X])^2  = E[X^2] - \mu^2 . \label{mod2:eq:Discrete:Variance:2} \
	\end{align}
\end{defn}



\section{Discrete Uniform Distribution} \label{mo2:section:DiscreteUniform}

\begin{defn} \label{mod2:defn:DiscreteUniform}
	A discrete random variable $X$ is said to follow a \textbf{discrete uniform distribution} if $X$ can take a finite number of values which are observed with equal probabilities. Therefore, the probability mass function, $P$, of $X$ which can take $n$ possible values,  is given by
	\begin{equation}
		P(X = x_i) = \frac{1}{n} \label{mod2:eq:DiscreteUniform}.
	\end{equation}
\end{defn}

\begin{note}\label{mod2:note:DiscreteUniform}
	In the discrete case, the notation $X \sim \text{Unif}(a, b)$ or $X \sim \text{U}(a, b)$ implies that $X$ can take \textbf{integer} values $x$ such that $a \leq x \leq b$. Here, the total numer of values $X$ can take is given by
	\begin{equation}
		n = b - a + 1,
	\end{equation}
	Therefore the p.m.f of $X$ can be given by
	\begin{equation}
	P(X = x)= \frac{1}{b - a + 1}\ \quad a \leq x \leq b.
	\end{equation}
		The \textit{expected value (mean)} and \textit{variance} of $X \sim \text{U}(a,b)$ are given by:
	\begin{align}
	E[X] &= \frac{a+b}{2}, \label{mod2:eq:DiscreteUniform:Mean} \\
	Var[X] &= \frac{(b-a+1)^2 - 1}{12}. \label{mod2:eq:DiscreteUniform:Variance} 
	\end{align}
\end{note}




\section{Binomial Distribution} \label{mod2:section:Binomial}
\begin{defn} \label{mod2:defn:Binomial}
	A discrete random variable $X$ is said to follow a \textbf{binomial distribution} with parameters $n$ and $p$, $X \sim \text{Bin}(n,p)$, if its probability mass function, $P$, is given by
	\begin{equation}
	P(X = x) = { n \choose x} p^x (1-p)^{n-x} \quad x \in ( 0, 1, 2, ... , n). \label{mod2:eq:BinomialDist} \ 
	\end{equation} 
\end{defn}

\begin{note}
	The \textit{expected value (mean)} and \textit{variance} of $X$ are given by:
	\begin{align}
	E[X] &= np,  \label{mod2:eq:Binomial:Mean} \\ 
	Var[X] &= np(1-p), \label{mod2:eq:Binomial:Variance} \\
	&= npq  \: \text{ where } q = (1-p).\ 
	\end{align}
\end{note}

\begin{note} \label{mod2:note:Binomial:Conditions}
	There are four conditions that describe a binomial distribution
	\begin{enumerate}[label = (\roman*)]
		\item The experiment consist of a fixed number of trials, $n$.
		\item The trials are independent.
		\item Each trial can be classified as a success or failure.
		\item The probability of success, $p$, is constant
	\end{enumerate}
\end{note}



\section{Geometric Distribution} \label{mod2:section:Geometric}
\begin{defn}\label{mod2:defn:Geomtric}
	A discrete random variable $X$ is said to follow a \textbf{geomteric distribution} with parameter $p$, $X \sim \text{Geo}(p)$, if its probability mass function, $P$, is given by
\begin{equation}
P(X = x) = p(1-p)^{x-1} \quad x \in (1, 2, ...). \label{mod2:eq:GeometricDist} \ 
\end{equation} 
\end{defn}

\begin{note}
	The \textit{expected value (mean)} and \textit{variance} of $X$ are given by:
	\begin{align}
	E[X] &= \frac{1}{p},  \label{mod2:eq:Geometric:Mean} \\ 
	Var[X] &= \frac{1-p}{p^2}, \label{mod2:eq:Geomtric:Variance} \\
	&= \frac{q}{p^2}  \: \text{ where } q = (1-p).\ 
	\end{align}
\end{note}

\begin{note} \label{mod2:eq:Geometric:Prop}
	The probability of $X>k$ is given as
	\begin{equation}
		P(X>k)=q^k
	\end{equation}
\end{note}
\begin{note} \label{mod2:note:Geometric:Conditions}
	There are three conditions that describe a geometric distribution
	\begin{enumerate}[label = (\roman*)]
		\item The trials are independent.
		\item Each trial can be classified as a success or failure.
		\item The probability of success, $p$, is constant
	\end{enumerate}
\end{note}



\section{Poisson Distribution}\label{mod2:section:Poisson}
\begin{defn} \label{mod2:defn:Poisson}
	A discrete random variable $X$ is said to follow a \textbf{poisson distribution} with parameter $\lambda$, $X \sim \text{Pois}(\lambda)$, if its probability mass function, $P$, is given by
\begin{equation}
P(X = x) =\frac{ \lambda ^ x e^{-\lambda}}{x!} \quad x \in (0, 1, 2, ...). \label{mod2:eq:PoissonDist} \ 
\end{equation} 
\end{defn}


\begin{note}
	The \textit{expected value (mean)} and \textit{variance} of $X$ are given by:
	\begin{align}
	E[X] &= \lambda,  \label{mod2:eq:Poisson:Mean} \\ 
	Var[X] &= \lambda. \label{mod2:eq:Poisson:Variance} 
	\end{align}
\end{note}


\begin{note} \label{mod2:note:Poisson:Conditions}
	The Poisson distribution is popular for modeling the number of times an event occurs in an interval of time. There are two conditions that describe a poisson distribution
	\begin{enumerate}[label = (\roman*)]
		\item Events are independent of each other.
		\item The average rate at which events occur, $\lambda$, is independent of any occurrences.
	\end{enumerate}
\end{note}


\section {Poisson Approximation to the Binomial Distribution} \label{mod2:section:PoissonApproxBinomial}
 
 \begin{defn} \label{mod2:defn:PoissonApproxBinomial}
The \textbf{Poisson limit theorem} states that a binomially distributed random variable $X \sim \text{Bin}\{n, p\}$ can be approximated by a poisson distribution if the following hold;
\begin{enumerate}[label = (\roman*)]
	\item $n > 50$,
	\item $np < 5$.
\end{enumerate}

Under the above conditions, $X \sim \text{Bin}(n, p)$ can be approximated by $X \sim \text{Pois}(np)$, $\lambda = E[X]$.

\end{defn}
	


\section{Continuous Random Varibales} \label{mod2:section:ContinuousRandomVar}

\begin{defn} \label{mod2:defn:ContinuousRandomVar}
	We say that $X$ is a \textbf{continuous random variable}, if there exists a non-negative function $f$, defined for all real $x \in (-\infty,\infty)$, having the property that, for any set of real numbers $B$,
	\begin{equation}
	P(X \in B) = \int_B f(x).dx. \
	\end{equation}
	The function $f$ is called the \textit{probability density function} of the random variable $X$.
\end{defn}


\begin{prop} \label{mod2:prop:ContinuousRV:1}
	Since $X$ must assume some value, $f$ must satisfy,
	\begin{equation}
	1 = P(X \in (-\infty,\infty)) = \int_{-\infty}^{\infty} f(x).dx.  \
	\end{equation}
\end{prop}

\begin{note} \label{mod2:note:ContinuousRV:Note1}
	If $B=[a,b]$, then
	\begin{align}
	P(X\in B) &= P(a \leq X \leq b), \\
	&= \int_a^b f(x) .dx. \
	\end{align}
\end{note}

\begin{prop} \label{mod2:prop:ContinuousRV:2}
	For any continuous random variable $X$ and real number $a$,
	\begin{equation}
	P(X=a) = 0.\
	\end{equation}
\end{prop}

\begin{note} 
	This implies
	\begin{equation}
	P( X \leq a) = P(X<a). \
	\end{equation}
\end{note}

\begin{defn} \label{mod2:defn:ContinuousRV:Expectation}
	If $X$ is a continuous random variable, the \textbf{expectation (mean)} of $X$ is given by,
	\begin{equation}
	E[X] = \int_{-\infty}^{\infty}x f(x).dx. \label{mod2:eq:ContinuousRV:Expectation} \
	\end{equation}
\end{defn}

\begin{defn} \label{mod2:defn:ContinuousRV:Variance}
	If $X$ is a continuous random variable, the \textbf{variance} of $X$ is given by,
	\begin{align}
	Var[X] &= E[X^2] - {E[X]}^2, \\
	 &= \int_{-\infty}^{\infty}x^2 f(x)\,\dd x - {\left(\int_{-\infty}^{\infty}x f(x)\,\dd x \right)} ^ 2. \label{mod2:eq:ContinuousRV:Variance} \
	\end{align}
\end{defn}

\begin{law} \label{mod2:law:UnconsciousStat}
	In the above formula for variance of a continuous random variable,  the \textbf{Law of the Unconscious Statiscian (LOTUS)} is used. This law states that if $X$ is a continuous randon varibale with probability density function $f$, then for any function g,
	\begin{equation}
		E[g(X)] = \int_{-\infty}^{\infty}g(x) f(x).dx. \label{mod2:eq:LOTUS:ContinuousRV}
	\end{equation}
	
	The law also extends to the discrete case. If $X$ is a discrete random variable, then by \textbf{LOTUS}, 
	
	\begin{equation}
	E[g(X)] =  \sum_{\forall i} g(x_i) P(X=x_i). \label{mod2:eq:LOTUS:DiscreteRV}
	\end{equation}
\end{law}	


\begin{defn} \label{mod2:defn:ContinuousRV:CDF}
	For a continuous random variable $X$ with p.d.f $f$,  the \textbf{cumulative distribution function}, $F$, is defined as,
	\begin{equation}
		F(x) = P(X \leq x), \
	\end{equation}
	
	Using Eq~[\ref{mod2:note:ContinuousRV:Note1}], where $B = (-\infty, x]$,
	\begin{equation}
		F(x) = \int_{-\infty}^{x} f(t)\,\dd t, \
	\end{equation}
	
	Using the \textbf{Fundamental Law of Calculus}, then
	\begin{equation}
	f(x) = \ddd{}{x}F(x)
.	\end{equation}
	
	
\end{defn}	

\begin{note} \label{mod2:note:ContinuousRV:CDF}
	This implies
	\begin{equation}
		P( a < X \leq b) = F(b) - F(a).
	\end{equation}
\end{note}

\begin{prop} \label{mod2:prop:ContinuousRV:CDF}
	The c.d.f can be used to calculate the median and quartiles of a distribution. If a continuous random variable $X$ has c.d.f $F$,  then 
	\begin{align}
	F(M) &= 0.5, \\
	F(LQ) &= 0.25, \\
	F(UQ) &= 0.75.
	\end{align}	
	where $M$ denotes the median, $LQ$ denotes the lower quartile and $UQ$ denotes the upper quartile.
\end{prop}
		




\section{Normal Distribution}\label{mod2:section:Normal}

\begin{defn}\label{mod2:defn:Normal}
	A continuous random variable $X$ is said to follow a \textbf{normal distribution} with parameters $\mu$ and $\sigma^2$, $X \sim \text{N}(\mu,\sigma^2)$, if its probability density function, $f$, is given by
	\begin{equation}
	f(x) = \frac{1}{\sqrt{2 \pi}\sigma} e^{-(x-\mu)^2/2\sigma^2} \quad -\infty < x < \infty  \label{mod2:eq:Normal}\
	\end{equation}
	The parameters $\mu$ and $\sigma^2$ represent the expected value (mean) and variance respectively.
\end{defn}

\begin{note} \label{mod2:note:Normal:Properties}
	Normal random variables have important properties
	\begin{enumerate}[label = (\roman*)]
		\item If $X$ is normally distributed with parameters $\mu$ and $\sigma^2$, then $Y = aX + b$ is normally distributed with parameters $a\mu + b$ and $a^2\sigma^2$ respectively. \label{mod2:note:Normal:Property1}
		\item The previous property implies that if $X$ is normally distributed with parameters $\mu$ and $\sigma^2$, then $Z=\frac{(X-\mu)}{\sigma}$ is normally distributed with parameters $0$ and $1$. We call $Z$ a \textit{standard} or \textit{unit} normal random variable. \label{mod2:note:Normal:Property2}
	\end{enumerate}
\end{note}


\section{Normal Approximation to the Poisson Distribution} \label{mod2:section:NormalApproxPois}

\begin{defn}\label{mod2:def:NormalApproxToPois:Definition}
	A poisson distributions $X \sim \text{Pois}(\lambda)$ where $\lambda > 15$, can be approximated by a normal distribution, $X \sim \text{N}(\lambda, \sqrt{\lambda})$.
\end{defn}

\begin{note}\label{mod2:note:NormalApproxtoPois:Continuity}
	To use this approximation, we must note that because the poisson is a discrete integer-valued random variable and the normal is a continuous random variable, we must write
	\begin{align}
	P(X=a) &= P \left( a -\frac{1}{2} < X < a + \frac{1}{2} \right), \\
	P(X \leq a) &= P \left( X < a + \frac{1}{2} \right), \\
	P(X \geq a) &= P \left( X > a - \frac{1}{2} \right), \\
	P(X < a) &= P \left( X < a - \frac{1}{2} \right), \\
	P(X > a) &= P \left( X > a + \frac{1}{2} \right), \\
	P(a \leq X < b) &= P \left( a - \frac{1}{2} < X < b - \frac{1}{2} \right), \\
	P(a < X \leq  b) &= P \left( a + \frac{1}{2} < X < b + \frac{1}{2} \right), \\
	P(a  \leq  X \leq  b) &= P \left( a - \frac{1}{2} < X < b + \frac{1}{2} \right).
	\end{align}
	This is called the \textbf{continuity correction}.
\end{note}


\section{Expectation and Variance} \label{mod2:section:ExpectationVariance}

\begin{prop}

Define  functions $g$ and $f$ with $a, b, c \in \mathbb{R}$. For \textbf{ANY} random variables (discrete or cotinuous) $X$ and $Y$, the following properties hold
\begin{align}
E[aX + b] &= aE[X] + b, \\
E[X + Y] &= E[X] + E[Y], \\
E[ag(X) + bf(Y)  + c] &= aE[g(X) ]+ bE[f(Y)] + c, \\
Var[ag(X) + b] &= a^2Var[g(X)].
\end{align}

If $X$ and $Y$ are \textbf{independent} random variables, then the following properties also hold
\begin{align}
E[XY] &= E[X] \ E[Y], \\
E[g(X)f(Y)] &= E[g(X)] \ E[f(Y)], \\
Var[X + Y] &= Var[X]+ Var[Y].
\end{align}

\end{prop}

\begin{note} \label{mod2:note:Independence:Non-implication}
	
	If $E[XY] = E[X] \ E[Y]$ or $Var[X + Y] = Var[X]+ Var[Y]$, this does not imply $X$ and $Y$ are independent.
	
\end{note}










